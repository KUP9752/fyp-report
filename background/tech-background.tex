\section{End-to-End Learning for Robot Control}

  End-to-end refers to a robot learning approach where the robot determines certain \textbf{policies} (actions to perform a certain task) from raw inputs from the action space. The action space can include anything that we think the robot may benefit form knowing. This means the 
  robot learns to map sensory inputs directly to motor commands, bypassing the need for intermediate steps such as feature extraction or state estimation. This approach leverages deep learning techniques \cite{Schmidhuber2015nn}, particularly convolutional neural networks (CNNs) and recurrent neural networks (RNNs), to process high-dimensional sensory data and generate appropriate actions. Which can then be used in techniques like Reinforcement Learning \todo{link section when done} or Imitation Learning. \todo{link section when done} Therefore, recent advancements in machine learning technologies has also reshaped the field of robotics and moved it forward \cite{Pierson18082017,newbury2023graspSynthReview,liu2021DRLminireview}.

  \missingfigure{add 2 figures as a small outline like a blackbox + a few steps}

  This contrasts with the classical approaches\todo{add refs here}. Classical robotics involves separating the behaviour of a robot into smaller tasks, where each task is managed by a distinct module and the system is functional when the pipeline comes together. Although good at precisely executing repetitive tasks, this approach requires complex and often handcrafted solutions for each module. This can lead to difficulties in scaling and adapting to new tasks or environments. 
  
  Therefore, the cutting-edge research in robotics concerns end-to-end systems in making multi-modal robots \todo{is multi-modal right?} that are capable of complex decision-making given some sort of environment.

\section{Reinforcement Learning (RL)}
  One of the tried and tested methods of end-to-end learning approaches is a branch under machine learning called \emph{Reinforcement Learning (RL)}.

  RL's main focus is training robots, which are called \emph{agents} in making decisions by interacting with the environment. The key objective is to teach a \emph{policy} to the agent that maximises the overall reward -usually defined by the task and involves a \emph{teacher}. The agent will explore, the possibly actions it can take through trail and error, while learning from the feedback given to it by the reward signal and its environment.

  One of the differentiating factors of RL from classical machine learning paradigms is that the feedback is not instantaneous and sequences of decisions influence the subsequent data and signals given to agent \cite{silver2015}.

\subsection{Mathematical Foundations}

  In scenarios involving autonomous acting, the capability of a an robot to reason and navigate complex problems or dynamic environments plays a central role. So in any RL system the agent must make a sequence of decisions that impact later outcomes. Markov Decision Processes (MDPs) provide a mathematical framework for modelling decisions in environments where the probabilistic outcomes are influenced by actions of such an agent. A formally defined MDP will have the following parts \cite{silver2015}:

\subsection{Markov Decision Processes}
  \subsubsection{State, S}
    The system must be able to process all different configurations of the environment. So the state will encapsulate the surroundings through raw sensory inputs (e.g. images, force sensors) in a high-level representation and will capture all relevant information available \cite{Sutton1998} which might be needed to make an informed decision at a particular time step.

  \subsubsection{The Markov Property}
    A state $S_t$ is \emph{Markov} if and only if:
    \[
      \mathbb{P} \left[S_{t+1} \mid S_t\right] = \mathbb{P}\left[ S_{t+1} \mid S_1, \ldots, S_t\right]
    \]

   % \todo{maybe make this a lemma type table??}
    This property ensures that all relevant information from the history is captured within the state. So once the state is knows the past states can be discarded. Making the current state a sufficient statistic for the future \cite{silver2015}.

    \subsubsection{State Transition Matrix, P}
    This matrix defines the transition probabilities from all states $s$ to all successor states $s'$, so:
    
    \[ P_{ss'} = \mathbb{P} \left[S_{t+1} = s'  \mid S_t = s\right]\] 
    and 
    
    \[ P =
    \begin{bmatrix}
      P_{11} & \cdots & P_{1n} \\ 
      \vdots & & \vdots\\
      P_{n1} & \cdots & P_{nn}
    \end{bmatrix}
    \]
    where each row of the matrix sums up to 1, due to the nature of probabilities. A tuple of a set of states and a transition matrix/function \(\llangle S, P \rrangle\) make up a \textbf{Markov Process} (or Markov Chain)
    
    \subsubsection{Reward Function, R}
    This is the scalar feedback signal. It ensures that the agent's learning is based on steps it is taking over time, so $R_t$ is how well the agent is doing at step $t$. It is defined as:

    \[R_s = \mathbb{E} \left[R_{t+1} \mid S_t = s\right]\]
    
    This allows the robot to eventually converge to a solution that maximises the cumulative rewards (the \emph{returns}) for the actions it has taken. 
    
    \subsubsection{Discount Factor, $\gamma$}
    This is mechanism to control the importance of future rewards. Sampled as: \(\gamma \in \left[0, 1\right]\), means that immediate reward is prioritised and while reward form longer sequence of actions decays, avoiding infinite cycles in Markov Chains.

    Combining the reward function and the discount factor we can defined the \emph{return}, $G_t$ as the total dicounted reward from time-step $t$:

    \[ 
      \begin{aligned}
      G_t &= R_{t+1} + \gamma R_{t+2} + \ldots \\ 
      &= \sum_{k=0}^{\inf}\gamma^k R_{t+k+1} 
    \end{aligned}
    \]
    \todo{define value functions and bellman either before or after action}

  \subsubsection{Actions, A}
    This is the set of all possible actions that are available to the robot in each state. Depending on the context of the task, actions can be discrete or continuous.


  So, a \emph{Markov Decision Process} is a tuple \(\langle S, A, P, R, \gamma \rangle\)
