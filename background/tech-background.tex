\section{Technical Background}
\subsection{End-to-End Learning for Robot Control}

  End-to-end refers to a robot learning approach where the robot determines certain \textbf{policies} (actions to perform a certain task) from raw inputs from the action space. The action space can include anything that we think the robot may benefit form knowing. This means the 
  robot learns to map sensory inputs directly to motor commands, bypassing the need for intermediate steps such as feature extraction or state estimation. This approach leverages deep learning techniques \cite{Schmidhuber2015nn}, particularly convolutional neural networks (CNNs) and recurrent neural networks (RNNs), to process high-dimensional sensory data and generate appropriate actions. Which can then be used in techniques like Reinforcement Learning \todo{link section when done} or Imitation Learning. \todo{link section when done} Therefore, recent advancements in machine learning technologies has also reshaped the field of robotics and moved it forward \cite{Pierson18082017,newbury2023graspSynthReview,liu2021DRLminireview}.

  This contrasts with the classical approaches. Classical robotics involves separating the behaviour of a robot into smaller tasks, where each task is managed by a distinct module and the system is functional when the pipeline comes together. Although good at precisely executing repetitive tasks, this approach requires complex and often handcrafted solutions for each module. This can lead to difficulties in scaling and adapting to new tasks or environments. So the cutting-edge research in robotics concerns end-to-end systems in making multi-modal robots. \todo{is multi-modal right?}
  
\subsection{Mathematical Foundations}
  