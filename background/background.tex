% \chapter{Background}
\chapter{Technical Background}
  In this chapter we will cover:
  \begin{enumerate}
    \item \emph{End-to-end learning} in Robotics and what it means.
    \item Common robot learning strategies like: \emph{Reinforcement} and \emph{Imitation Learning}, how to formulate and approach them.
    \item \emph{Computer Vision} and how Robotics as a branch of computing are heavily coupled.
    \item \emph{Active Vision} and why it can move the field of robotics forward.
  \end{enumerate}
\section{End-to-End Learning for Robot Control}

  End-to-end refers to a robot learning approach where the robot determines certain \textbf{policies} (actions to perform a certain task) from raw inputs from the action space. The action space can include anything that we think the robot may benefit form knowing. This means the 
  robot learns to map sensory inputs directly to motor commands, bypassing the need for intermediate steps such as feature extraction or state estimation. This approach leverages deep learning techniques \cite{Schmidhuber2015nn}, particularly convolutional neural networks (CNNs) and recurrent neural networks (RNNs), to process high-dimensional sensory data and generate appropriate actions. Which can then be used in techniques like Reinforcement Learning \todo{link section when done} or Imitation Learning. \todo{link section when done} Therefore, recent advancements in machine learning technologies has also reshaped the field of robotics and moved it forward \cite{Pierson18082017,newbury2023graspSynthReview,liu2021DRLminireview}.

  \missingfigure{add 2 figures as a small outline like a blackbox + a few steps}

  This contrasts with the classical approaches\todo{add refs here}. Classical robotics involves separating the behaviour of a robot into smaller tasks, where each task is managed by a distinct module and the system is functional when the pipeline comes together. Although good at precisely executing repetitive tasks, this approach requires complex and often handcrafted solutions for each module. This can lead to difficulties in scaling and adapting to new tasks or environments. 
  
  Therefore, the cutting-edge research in robotics concerns end-to-end systems in making multi-modal robots \todo{is multi-modal right?} that are capable of complex decision-making given some sort of environment.

\section{Reinforcement Learning (RL)}
  One of the tried and tested methods of end-to-end learning approaches is a branch under machine learning called \emph{Reinforcement Learning (RL)}. \todo{is it a branch under machine learning?}
  \missingfigure{add the classic agent state reward action diagram}

  RL's main focus is training robots, which are called \emph{agents} in making decisions by interacting with the environment. The key objective is to teach a \emph{policy} to the agent that maximises the overall reward -usually defined by the task and involves a \emph{teacher}. The agent will explore, the possibly actions it can take through trail and error, while learning from the feedback given to it by the reward signal and its environment.

  One of the differentiating factors of RL from classical machine learning paradigms is that the feedback is not instantaneous and sequences of decisions influence the subsequent data and signals given to agent \cite{silver2015}.

\subsection{Mathematical Foundations}

  In scenarios involving autonomous acting, the capability of a an robot to reason and navigate complex problems or dynamic environments plays a central role. So in any RL system the agent must make a sequence of decisions that impact later outcomes. Markov Decision Processes (MDPs) provide a mathematical framework for modelling decisions in environments where the probabilistic outcomes are influenced by actions of such an agent. A formally defined MDP will have the following parts \cite{silver2015}:

\subsection{Markov Decision Processes}
  \subsubsection{State, S}
    The system must be able to process all different configurations of the environment. So the state will encapsulate the surroundings through raw sensory inputs (e.g. images, force sensors) in a high-level representation and will capture all relevant information available \cite{Sutton1998} which might be needed to make an informed decision at a particular time step.

  \subsubsection{The Markov Property}
    A state $S_t$ is \emph{Markov} if and only if:
    \[
      \mathbb{P} \left[S_{t+1} \mid S_t\right] = \mathbb{P}\left[ S_{t+1} \mid S_1, \ldots, S_t\right]
    \]

   % \todo{maybe make this a lemma type table??}
    This property ensures that all relevant information from the history is captured within the state. So once the state is knows the past states can be discarded. Making the current state a sufficient statistic for the future \cite{silver2015}.

  \subsubsection{State Transition Matrix, P}
    This matrix defines the transition probabilities from all states $s$ to all successor states $s'$, so:
    
    \[ P_{ss'} = \mathbb{P} \left[S_{t+1} = s'  \mid S_t = s\right]\] 
    and 
    
    \[ P =
    \begin{bmatrix}
      P_{11} & \cdots & P_{1n} \\ 
      \vdots & & \vdots\\
      P_{n1} & \cdots & P_{nn}
    \end{bmatrix}
    \]
    where each row of the matrix sums up to 1, due to the nature of probabilities. A tuple of a set of states and a transition matrix/function \(\llangle S, P \rrangle\) make up a \textbf{Markov Process} (or Markov Chain)

  \subsubsection{Actions, A}
    This is the set of all possible actions that are available to the robot in each state. Depending on the context of the task, actions can be discrete or continuous.

  \subsubsection{Reward Function, R}
    This is the scalar feedback signal. It ensures that the agent's learning is based on steps it is taking over time, so $R_t$ is how well the agent is doing at step $t$. It is defined as:

    \[R_s = \mathbb{E} \left[R_{t+1} \mid S_t = s\right]\]
    
    This allows the robot to eventually converge to a solution that maximises the cumulative rewards (the \emph{returns}) for the actions it has taken. 
  
  \subsubsection{Discount Factor, $\gamma$}
    This is mechanism to control the importance of future rewards. Sampled as: \(\gamma \in \left[0, 1\right]\), means that immediate reward is prioritised and while reward form longer sequence of actions decays, avoiding infinite cycles in Markov Chains.

    Combining the reward function and the discount factor we can defined the \emph{return}, $G_t$ as the total discounted reward from time-step $t$:

    \[ 
    \begin{aligned}
      G_t &= R_{t+1} + \gamma R_{t+2} + \ldots \\ 
      &= \sum_{k=0}^{\inf}\gamma^k R_{t+k+1} 
    \end{aligned}
    \]
    
    Therefore, a \emph{Markov Decision Process} is a tuple \(\langle S, A, P, R, \gamma \rangle\) as the parts are defined above.
    
    
  \subsubsection{Policy, $\pi$}
    The higher-level goal of any RL system is to learn an optimal policy \(\pi \left( a \mid s\right) = \mathbb{P} \left[A_t = a \mid S_t = s\right]\) which aims to maximise the return. Policies fully define the behaviour of the agent. As seen by the function's type \(\pi: S \rightarrow A \) they only depend on the current state and are time independent \( A_t \sim \pi\left( \cdot \mid S_t\right), \forall t > 0 \)
    
  \subsubsection{Value Functions}
    On top of these we define two calue function, \emph{state-value}: the expeted return starting from state $s$, and then following policy $\pi$:

    \[ v_\pi \left(s\right) = \mathbb{E} = \left[G_t \mid S_t = s\right]\]

    and the \emph{action-value} function, which is the expected return starting from state $s$, taking action $a$, and then continuing a policy $\pi$:

    \[ q_\pi \left(s, a\right) = \mathbb{E} \left[ G_t \mid S_t = s, A_t = a\right]\]


    while the optimal versions can be defined as:
    \[v_* \left(s\right) = \underset{\pi}{\max} \ v_\pi \left(s\right)\]

    \[q_* \left(s, a\right) = \underset{\pi}{\max} \ q_\pi \left(s, a\right)\]
    
    The optimal means the best possible performance can be achieved in the MDP and it can be considered ``solved'' once we find these optimal functions.

    \todo{should I define the bellman function?}

\subsection{Different RL Models}
  The flavours of different Reinforcement Learning models can be broadly categorised into 2 types: \textbf{Model-Based} and \textbf{Model-Free}


\subsection{Model-Based RL}
  Model based approaches involve methods of creating an internal model and representation of the state the agent is interacting with. It usually involves two main steps: dynamics and model learning, then planning-learning in that model \cite{MAL-086}. This allows the agent to plan ahead in its environment and improve sample efficiency \cite{wu23robotLearn,liu2021DRLminireview}.





\subsection{Model-Free RL}
  This collection of models attempt to learn a policy directly by trial and error, without explicitly modelling the environments dynamics. Usually preffered when the environments is too complex or too costly to model.
  \subsubsection{Value-Based}
  \subsubsection{Policy-Based}
  \subsubsection{Actor-Teacher}

  Hydrid methods are also a viability, where the characteristics from each can be conbined [ref?]


\chapter{Related Work}
  In this chapter we will discuss:
  \begin{enumerate}
    \item Current research in robotics with active vision and derivatives.
    \item Some limitations of active vision it the current landscape.
    \item How different techniques can be leveraged to fit full vision policies together
    \item Some comparison of test-time and action-time vision\todo{wanted to do this, but might integrate this into plan or a segue into plan later}
    % \item Grasp Synthesis with AV Methods paper. Heuristic and ML (self-supervised and Q learning) approaches compares to other algorithms as well as real life most optimal solutions for evaluation.Q-learning performs worse in real life tests, mostly due to differences in the depth sensor acting differently. shows 3d heuristic performs better compared to ML and naive algorithmic approaches. ML is more sensitive to sensor noise so the heuristic approaches generalise better to the real world from simulations.
    % \item Closed-Loop NBV Planning for target-driven grasping. This paper talks about active exploration of the space, during test time. policies are already trained (system called MoveIt is used) at every it generates the best grasp candidate and the next best view along with the information gain. Then depending on the stopping criteria it will execute this task, or check the next-best-view found. Stopping criteria to balance exploration and exploitation. Example of exploration and reasoning to find the optimal pose, number 1 from 1.2 Objectives. Downsides is that it requires prior information about the target object's bounding box. Mostly heuristic I think
  \end{enumerate}
\section{Active Vision Resaerch}

As active vision within robotics is an emerging area of research, there isn't a consensus on a single approach to tackle this issue. On top of this aspect, by the inherent nature of differing tasks requiring differing approaches, a general solution may not be possible for every task. However, using techniques highlighted earlier and building on top of these we can come up with frameworks that are competent and most importantly highly competitive compared to current static camera systems.

Here we will highlight some significant contributions that are similar in scope to what we are aiming to achieve in this project.

\section{Constraints in Active Vision}
The main constraint is that for effective and optimal learning large number of demonstrations must be fed into an agent. However, the manual aspect of generating ``expert'' demonstrations is not always possible. This naturally leads the literature to be mainly concerned with approaches that deal with few-shot learning (\todo[color=green]{define earlier in av and link}) and self-supervision,

A direct by-product of this constraint is incorporating prior knowledge into the learning. Such as the information about an object in an \emph{object-centred} problem, meaning the task revolves around manipulation of specific object or objects. This allows to compensate for lack of large set of demonstrations. These could include object poses \cite{huang2018generalisedTPlearning, hu2023modelpredictiveoptimisation} or meta-learning policies from pre-trained models on desirable datasets \cite{finn2017oneshotvisualimitationlearning, mandi2022}.


\section{Object Priors}
Following the constraints, an active perception learning robot is usually given information about the task and the objects that are important to that task, that way extracting the relevant policies gives us a baseline on the prominence of such a policy.

\subsubsection{\emph{Robot See Robot Do}}
This is explored in \emph{Kerr et al.}, where the work revolves around teaching a robot to interact with manipulable objects, such as chests, drawers, glasses and toys \cite{kerr2024robotrobotdoimitating}. At its core this is a grasping task integrated with one-shot learning, however, with the important prior that the object's 4D model -recovered by 4D Differentiable Part Models (4D-DPM). This way the we can observe what the robot can synthesise the correct manipulation points from its viewpoint and generalise this to all given viewpoints. Although, this isn't immediately an active-vision robot, this is a good starting point in understanding to design systems that are using their perception to make meta-decisions that influence their movement in their main policies. 

\section{Semi-Active Vision}
The simplest idea in teaching a robot to see in a human-like manner is to train it on data directly generated by human interactions. This also means the subject specific information like priors can be inferred by the policy, instead of being explicitly provided by the researchers.

\emph{Chuang et al.} explore the idea of teaching a robot the task policy joint with the policy of moving a camera fitted arm \cite{chuang2024activevisionneedexploring}. Their setup includes a Virtual Reality (VR) Headset, which allows the demonstrator to move the AV camera on the robotic arm. This allows them to teach policies for tasks where the subject may be blocked by small static cameras around the scene, or the camera fitted in a eye-in-hand configuration being occluded or out-of-bounds due to orientation of the gripper with respect to the task. 

Although, their research is promising they acknowledge that active vision also brings some issues that need solving. Some important mentions are: operational and architectural complexity, though, they used similar architectures for all their different camera configurations, they note that a bespoke system for AV might benefit such a system. 

The expanded action space also poses an issue as the state space complexity explodes, and hint that decoupling the vision and the control system might help with the issue. Finally, they touch on distribution shifts being a big issue, this mainly stems from the earlier implicit subject information. As the model is not aware of targets it will learn what it can infer, so if the demonstrations do not contain enough variations in poses, generalising the  locations and other characteristics for an object become tougher. A possible strategy in solving that particular issue is generating augmented samples, to fight this covariate shift.

\section{Self-Supervision and Data Augmentation}
Another widely used strategy to mitigate providing vast amounts of manually crafted and hand labelled data is to subscribe to the idea of self-supervision \todo{link to earlier definition from cv/av}. These are used to counteract the instability of reinforcement learning and the inefficiency of random exploration.

\subsubsection{\emph{Making Imitation Learning Easy with Self-Supervision: MILES}}
Incorporating this into robot learning tasks usually takes the form of generating augmented movement trajectories. Which are simulated and sampled from the limited number of human trajectories (usually accepted to be expert behaviour) given to the agent. \emph{Papagiannis et al.} using the \emph{MILES} framework, simplifies the process by removing the human intervention aspects from highly repeatable supervision parts of the learning \cite{papagiannis2024milesmakingimitationlearning}. 

This mimics exploration-based RL learning systems where given a list of trajectories from the expert behaviour,the agent will move to a pose near the demonstration and attempt to move itself back to original trajectory. In the process creating an augmented path that eventually joins with the ``correct'' one. As it collects sensor data along the path (per waypoint) it therefore, creates augmented data that can now be used to aid its training. So, it is self-supervised in the way it collects data. As no interaction is needed to correct the agent back to starting position or other environmental resets (assuming the learning tasks doesn't manipulate the scene in a non-recoverable way),

This is achieved by training a separate policy for each task as a \emph{LSTM} (long-short term memory) network, based on Behavioural Cloning, which is a type of Recurrent Neural Network (RNN) which handles sequential data \cite{medsker2001recurrent}.This allows the policy to learn the gradually changing trajectory while remembering the steps taken in the past. On top of this no object pose priors are given to the network, meaning the networks learnt policies should be applicable to different object poses.

However, another important part of \emph{MILES} is that force sensors are also included in the decision making policies. Which is not a make-or-break addition, as they conclude, the force modality sometimes helps the system achieve better accuracy when coupled with vision, and sometimes not. While just force -without vision, in a somewhat expected manner- performs quite badly in any of the evaluation tasks they have chosen. 



% \cite{natarajan2021graspsynthesisnovelobjects}
\todo[color=green]{this (see comment in file) also kinf of goes under learing using heuristics but not really trajectory maybe move to next section and so some explanation if needed}

\section{Attention and Information Gain (IG)}
This is pivotal part of object-centric tasks. Because, if an agent knows what to focus on, then we can teach it a policy to learn to focus on specific subjects.

\subsubsection{\emph{Observe Then Act}}
Taking a third-person-view look to the classical grasping task, \emph{Wang et al.} aim to optimise this viewpoint based on the task goal \cite{wang2024observeactasynchronousactive}. They take a asynchronous viewpoint control approach (see \ref{sec:asynch-synch}); this separation of the camera systems and the motor actions over time. Leading to less need for coordination between the two section and instead, the model focuses more on task-specific movements and distribute this coordination throughout the task. 
They, again, follow a few-shot learning approach. The model comprises of two separate agents, a next-best-view (NBV) agent for optimal viewpoints and next-best-pose (NBP) for determining the gripper's action based on the previous agent's output. Active perceptions is achieved by alternating between sensor and motor action interfaces in each episode; which then leads to the learning of the tasks. These tasks are usually of sequential nature as the asynchronous approach works best with such systems, as they discuss.
\missingfigure{the neuron figure form the paper}

They follow a POMDP (\ref{sec:pomdp}) formulation to model the problem:
\[
  \langle \mathcal{O}, \mathcal{A}^c, P, \mathcal{A}^g, \mathcal{O}', P', R, \gamma \rangle
\]

where $\mathcal{O}$ and $\mathcal{O}'$ represent the observation spaces at times $t$ and $t'$ the NBV policy $\pi_v$, determines the camera viewpoint action $a_t^c \in \mathcal{A}^c$, given an observation $o_t \in \mathcal{O}$, and obtains a new observation $o_{t'} \in \mathcal{O}'$ through the transition probability $p\left(o_{t'} \mid o_t, a_t^c\right) \in P$. Finally, the NBP policy, $\pi_g$, then determines the gripper action $a^g_{t'} \in \mathcal{A}^g$ based on the new observation $o_{t'}$. then using the transition probability $p'\left(o_{t+1} \mid o_t, a^c_t\right) \in P'$ the next scene observation can be obtained and reward $r_{t+1} \in R$ will be provided. So, the model will be learning these policies: $\pi^*_v$ and $\pi^*_g$ then try to jointly maximise the return for the joint reward for this collective task: 
\[
  \pi_v^*, \pi_g^* = 
  arg~\underset{\pi_v, \pi_g}{max} 
  ~\mathbb{E}
  \left[
    \sum_{t=0}^{\inf}{\gamma^t R(o_t, a^c_t, a^g_t)}
  \right]
\]
\todo[color=green]{more information about the 3D voxels for environment normalisation?? this is more about fighting the caveats of few-shot learning not really fitting here, but might include if it comes in useful later}

Another important contribution here is the use of augmented trajectories again. Similar to \cite{papagiannis2024milesmakingimitationlearning}, demonstration trajectories are augmented in a \emph{viewpoint-aware} manner to expand the learning set by sampling the observations and discovering key frames. It does this while the viewpoint is allowed to shift, meaning the samples getting generated between time frame $T_t$ and $T-{t+1}$ (where $T$ is a trajectory) can have different views, which aids the camera policy in progressive movements. This is done for keyframes for both camera movement and gripper pose, for the two policies.

\subsubsection{Closed-Loop Next-Best-View Planning for Target-Driven Grasping}
Similar in idea, \emph{Breyer et al.} explore yet another grasping task (although first-person-view this time) this time. They augment classic grasp synthesis tasks that are mostly reliant on deep learning approaches, which suffer depending on the visual information it has available. Similar to \emph{Wang et al.}, they interact with the environment in an asynchronous manner within a fixed rate. They determine the best candidates for grasping then compute the next-best-view with its associated information gain. 
The information gain metric, allows the system to explore viewpoints in the neighbourhood of the current view and estimate what might be ideal for the grasping, bridging the rewards of the two policies. This again is made possible through the assumption that a task is object dependent hence the model receives bounding boxes and the geometry of the subject of the task.

However, the interesting takeaway from their system is an early stopping mechanism. Due to the nature of exploration multiple views can be scanned and many relevant ones can be identified by the system. Without stops, this algorithms may over explore and end up quite inefficient. These stopping conditions include: timeouts (and due to time-framing, limited number of policy updates), minimum thresholding on the information gain so that no unnecessary exploration is done where the estimated IG might be lower; and finally, convergence, when the VGN (\todo{explain grasp network}) outputs converge the exploration will stop. 

Therefore, the importance of their research is highlighted in the robustness of their system and the balance they managed to strike between exploration and exploitation. 

\todo[color=green]{talk about attention and active perception}

\subsection{Outline}
In summary, work in the field of active vision usually follows an attention metric which is guided by a reward system, usually depending on some priors. The need for priors, although, can be mitigated by providing more demonstrations, this complicates the test setups and operations. Therefore, a common approach is to create augmented data and broaden the horizon of the agent through the exploration of that synthetic space.

Combining these ideas shows us that active vision policy adjustments are not only possible during policy execution but also very promising way to advance the field of robot learning.



%//CHECK interesting paragraph not sure about it tho
% Another interesting point is the relevance of active vision for a task which isn't object-centred. 
% Does the use of active vision make sense to a robot where there isn't a defined subject? Or should the agent explore and find objects of importance?
% \todo{not sure about this paragraph, does this make sense maybe for the final report}

