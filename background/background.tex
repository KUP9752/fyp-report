% \chapter{Background}
\chapter{Technical Background}
  In this chapter we will cover:
  \begin{enumerate}
    \item \emph{End-to-end learning} in Robotics and what it means.
    \item Common robot learning strategies like: \emph{Reinforcement} and \emph{Imitation Learning}, how to formulate and approach them.
    \item \emph{Computer Vision} and how Robotics as a branch of computing are heavily coupled.
    \item \emph{Active Vision} and why it can move the field of robotics forward.
  \end{enumerate}
  \todo{do this section}
\section{End-to-End Learning for Robot Control}

  End-to-end refers to a robot learning approach where the robot determines certain \textbf{policies} (actions to perform a certain task) from raw inputs from the action space. The action space can include anything that we think the robot may benefit form knowing. This means the 
  robot learns to map sensory inputs directly to motor commands, bypassing the need for intermediate steps such as feature extraction or state estimation. This approach leverages deep learning techniques \cite{Schmidhuber2015nn}, particularly convolutional neural networks (CNNs) and recurrent neural networks (RNNs), to process high-dimensional sensory data and generate appropriate actions. Which can then be used in techniques like Reinforcement Learning \ref{sec:rl} or Imitation Learning \ref{sec:il}. Therefore, recent advancements in machine learning technologies has also reshaped the field of robotics and helps move it forward \cite{Pierson18082017,newbury2023graspSynthReview,liu2021DRLminireview}.

  \missingfigure{add 2 figures as a small outline like a blackbox + a few steps}

  This contrasts with the classical approaches\todo{citations here from the older theory papers maybe}. Classical robotics involves separating the behaviour of a robot into smaller tasks, where each task is managed by a distinct module and the system is functional when the pipeline comes together. Although good at precisely executing repetitive tasks, this approach requires complex and often handcrafted solutions for each module. This can lead to difficulties in scaling and adapting to new tasks or environments. 
  
  Therefore, the cutting-edge research in robotics concerns end-to-end systems in making multi-modal robots, which can be tuned for specific tasks if needed, using their capabilities of complex decision-making withing the given environments.

% Math Foundations
\section{Mathematical Foundations}

  In scenarios involving autonomous acting, the capability of a an robot to reason and navigate complex problems or dynamic environments plays a central role. So in any RL system the agent must make a sequence of decisions that impact later outcomes. Markov Decision Processes (MDPs) provide a mathematical framework for modelling decisions in environments where the probabilistic outcomes are influenced by actions of such an agent. A formally defined MDP will have the following parts \cite{silver2015}:

\subsection{Markov Decision Processes}
  \subsubsection{State, S}
    The system must be able to process all different configurations of the environment. So the state will encapsulate the surroundings through raw sensory inputs (e.g. images, force sensors) in a high-level representation and will capture all relevant information available \cite{Sutton1998} which might be needed to make an informed decision at a particular time step.

  \subsubsection{The Markov Property}
    A state $S_t$ is \emph{Markov} if and only if:
    \[
      \mathbb{P} \left[S_{t+1} \mid S_t\right] = \mathbb{P}\left[ S_{t+1} \mid S_1, \ldots, S_t\right]
    \]

   % \todo{maybe make this a lemma type table??}
    This property ensures that all relevant information from the history is captured within the state. So once the state is knows the past states can be discarded. Making the current state a sufficient statistic for the future \cite{silver2015}.

  \subsubsection{State Transition Matrix, P}
    This matrix defines the transition probabilities from all states $s$ to all successor states $s'$, so:
    
    \[ P_{ss'} = \mathbb{P} \left[S_{t+1} = s'  \mid S_t = s\right]\] 
    and 
    
    \[ P =
    \begin{bmatrix}
      P_{11} & \cdots & P_{1n} \\ 
      \vdots & & \vdots\\
      P_{n1} & \cdots & P_{nn}
    \end{bmatrix}
    \]
    where each row of the matrix sums up to 1, due to the nature of probabilities. A tuple of a set of states and a transition matrix/function \(\llangle S, P \rrangle\) make up a \textbf{Markov Process} (or Markov Chain)

  \subsubsection{Actions, A}
    This is the set of all possible actions that are available to the robot in each state. Depending on the context of the task, actions can be discrete or continuous.

  \subsubsection{Reward Function, R}
    This is the scalar feedback signal. It ensures that the agent's learning is based on steps it is taking over time, so $R_t$ is how well the agent is doing at step $t$. It is defined as:

    \[R_s = \mathbb{E} \left[R_{t+1} \mid S_t = s\right]\]
    
    This allows the robot to eventually converge to a solution that maximises the cumulative rewards (the \emph{returns}) for the actions it has taken. 
  
  \subsubsection{Discount Factor, $\gamma$}
    This is mechanism to control the importance of future rewards. Sampled as: \(\gamma \in \left[0, 1\right]\), means that immediate reward is prioritised and while reward form longer sequence of actions decays, avoiding infinite cycles in Markov Chains.

    Combining the reward function and the discount factor we can defined the \emph{return}, $G_t$ as the total discounted reward from time-step $t$:

    \[ 
    \begin{aligned}
      G_t &= R_{t+1} + \gamma R_{t+2} + \ldots \\ 
      &= \sum_{k=0}^{\inf}\gamma^k R_{t+k+1} 
    \end{aligned}
    \]
    
    Therefore, a \emph{Markov Decision Process} is a tuple \(\langle S, A, P, R, \gamma \rangle\) as the parts are defined above.
    
    
  \subsubsection{Policy, $\pi$}
    The higher-level goal of any RL system is to learn an optimal policy \(\pi \left( a \mid s\right) = \mathbb{P} \left[A_t = a \mid S_t = s\right]\) which aims to maximise the return. Policies fully define the behaviour of the agent. As seen by the function's type \(\pi: S \rightarrow A \) they only depend on the current state and are time independent \( A_t \sim \pi\left( \cdot \mid S_t\right), \forall t > 0 \)
    
  \subsubsection{Value Functions}
    On top of these we define two value function, \emph{state-value}: the expected return starting from state $s$, and then following policy $\pi$:

    \[ v_\pi \left(s\right) = \mathbb{E} = \left[G_t \mid S_t = s\right]\]

    and the \emph{action-value} function, which is the expected return starting from state $s$, taking action $a$, and then continuing a policy $\pi$:

    \[ q_\pi \left(s, a\right) = \mathbb{E} \left[ G_t \mid S_t = s, A_t = a\right]\]


    while the optimal versions can be defined as:
    \[v_* \left(s\right) = \underset{\pi}{\max} \ v_\pi \left(s\right)\]

    \[q_* \left(s, a\right) = \underset{\pi}{\max} \ q_\pi \left(s, a\right)\]
    
    The optimal means the best possible performance can be achieved in the MDP and it can be considered ``solved'' once we find these optimal functions.

    \todo{should I define the bellman function?}

    

% Reinforcement Learning
\section{Reinforcement Learning (RL)}\label{sec:rl}
  One of the tried and tested methods of end-to-end learning approaches is a branch under machine learning called \emph{Reinforcement Learning (RL)}. 

  \begin{figure}[h]
    \centering
    \includegraphics[width=0.6\textwidth]{assets/background/silver-rl.png}
    \caption{Simple RL diagram to reinforce intuition \cite{silver2015}}\label{fig:rl-diag}
  \end{figure}

  RL's main focus is training algorithms, which are called \emph{agents}, in making optimal decisions by interacting with the environment. The key objective is to teach a \emph{policy} to the agent that maximises the overall reward -usually defined by the task and involves a reward or \emph{teacher}. The agent will explore, the possible actions it can take through trial and error, while learning from the feedback given to it by the reward signal and its environment.

  One of the differentiating factors of RL from classical machine learning paradigms is that the feedback is not instantaneous and sequences of decisions influence the subsequent data and signals given to agent \cite{silver2015}.

\subsection{RL in Practice}
  Reinforcement learning, can be used to train a variety of agents that is not limited by physical robots. It can learn to play video-games \cite{comi2018}, automation tasks \cite{} , in natural language processing \cite{paulus2017deepreinforcedmodelabstractive}, applications in healthcare (where RL is categorised as dynamic treatment regimes) for use in chronic diseases or critical care \cite{yu2020reinforcementlearninghealthcaresurvey} and lastly -most importantly for us- learning movement behaviours for robots.

  \subsection{Exploration and Exploitation}
  
  A fundamental challenge in utilising RL is the constant act of balancing \textbf{exploration} and \textbf{exploration}. A trade-off must be made in the created system.
  \begin{enumerate}
    \item \textbf{Exploration:}
    This is when the agent decides to \emph{explore} new actions that might potentially lead to better long-term outcomes. An issue this can cause is that the time it takes to explore all possibilities might not be feasible. But crucial to utilise in in problems with sparse reward models.\todo[color=green]{citations}.
    \item \textbf{Exploitation:}
    When the agent prioritises short-term, immediate rewards by exploiting its current knowledge. For example, taking an agent playing a video game; if a high score was found the agent might be unaware that a higher score can be achieved with a different set of moves. 
  \end{enumerate}

  The trade-offs must be balanced between the two in any RL algorithm or model while considering sampling efficiencies and ease of training \cite{liu2019simpleexplorationsampleefficient}. To relate it back to a robotics example, too much exploration might lead to inefficient training and instability; while too much reliance on exploitation might lead to suboptimal behaviours in the movement of a robot executing a task. 

  % Some common strategies are: $\epsilon$-Greedy, Decay $\epsilon$-Greedy, Upper Confidence Bound (UCB), Thompson Sampling, Intrinsic Motivation (Curiosity-Driven Exploration).
  % \todo[color=green]{citations for these or just remove?}
  
  Along with this widespread use and elemental challenges, comes differing methods of utilising the RL framework. The likes of which can be broadly classified into two types: \textbf{Model-Based} and \textbf{Model-Free}.
  
  \subsection{Model-Based RL}
  Model based approaches involve methods of creating an internal model and representation of the state the agent is interacting with. It usually involves two main steps: learning the dynamics of the model, then planning and learning within it \cite{MAL-086}. These models have underlying principals of the Markov Decision Process (\ref{sec:mdp}).

  Main power of this approach comes from its simulated core. As fewer real interactions can allow it to have a higher sample efficiency \cite{liu2021DRLminireview,wu23robotLearn}, meaning the amount of experience needed (mostly relating to time spent) to learn optimal policies is quite low, and good policies can be quickly learnt.

  On top of that having an underlying model essentially allows the agent to understand its surroundings better, without having to guess or learn them. This means the model can focus on learn the model and generalise better for unseen states \cite{MAL-086}.
  
  However, it also has some drawbacks. Mainly the model introduces a bias into the system, which means inadequate representations or faulty models can create policies that exploit deficiencies in these models \cite{Deisenroth2011PILCO,wang2019benchmarkingmodelbasedreinforcementlearning}, although, some recent works have helped alleviate that bias by characterising the uncertainty of these learnt models \cite{kurutach2018modelensembletrustregionpolicyoptimization,chua2018deepreinforcementlearninghandful,clavera2018modelbasedreinforcementlearningmetapolicy}. Therefore, modelling is a manual, human task which can lead to human-error or insufficient coverage of the action space, leading to suboptimal results.

  One of the most important issues being the learning of the dynamics being coupled with the policy. This makes the agent more prone to performance local-minima, which stem from exploitation and off-learning not being fully investigated under model-based approaches.
  \todo[color=green]{verify this is true and check some sources}

  \subsubsection{Applications in Robotics}
  These methods can be used for motion planning, trajectory optimisation and learning from limited interactions, some common methods are: Monte Carlo Tree Search (MCTS),Probabilistic Inference for Learning Control (PILCO) \cite{Deisenroth2011PILCO}, MuZero: Combines model-based approaches with deep learning.\todo[color=green]{cite here}
  
  \subsection{Model-Free RL}
  This collection of schemes attempt to learn a policy directly by trial and error, without explicitly modelling the environments dynamics. Usually preferred when the environment is too complex or too costly to model.

  There are a few different variants of model-free approaches, all of them similar in the way they omit a model, but have different methods in extracting the optimal policies.
  

  \subsubsection{Value-Based}
    These techniques aim to learn the value of states (or an estimate for the value of states) and actions. So they learn the $v_\pi$ or $q_\pi$ function. which can then be used to extract the optimal policy $\pi_*$ for deciding the actions.

    Such systems are mainly used for simple navigation tasks, basic motor control and arm reaching scenarios. Some example methods are: Q-Learning [Off-Policy], Deep Q-Networks (DQN) [Off-Policy], SARSA [On-Policy]. \todo[color=green]{citations}

  \subsubsection{Policy-Based}
  This on the other hand learn a policy directly. Bypassing the need to learn the values of states or actions completely. This can be helpful if the state space and/or the action space are quite large. For example, if the action space was infinite, then value-based approaches would not be feasible as all actions must be tried to find the best, which makes directly learning the policy is the only possible approach.

  These systems are for more complex control tasks, such as dexterous limb manipulation, robot hand grasping. Some examples are: REINFORCE [On-Policy], Proximal Policy Optimisation (PRO) [On-Policy], Trust Region Policy Optimisation (TRPO) [On-Policy]. \todo[color=green]{citations}

  \subsubsection{Actor-Teacher} 
  The main idea in this approach is the agent improves its policy from direct feedback from a \emph{teacher}, this teacher can be an external system that provides rewards or a complete manual system where the teacher is an ``expert human'' providing demonstrations (which goes into Imitation Learning, see \ref{sec:il}). Some common approaches are
  \begin{itemize}
    \item \textbf{Direct Action Supervision}: where the teacher suggest actions directly to the agent.
    \item \textbf{Criticising Actions}: where the actions are suggested by the teacher.
    \item \textbf{Shaped Rewards}: the teacher modifies the reward signal dynamically to encourage desirable behaviours.
    \item \textbf{Curriculum Learning}: the teacher controls the difficulty of tasks over time to ease the learning of the agent.
  \end{itemize}
  
  \subsubsection{Hybrid Approaches}
  Mixing the two together is also a viability, where the characteristics from each can be combined benefit from different guarantees each provides \cite{qu2020combiningmodelbasedmodelfreemethods}
  On top of the model dependence, the RL approach can be of two flavours:

  \subsection{On-Policy}
  This approach means that the agent updates its policy from data generated by the current policy. So an agent can only learn from the actions it purposefully took under its latest policy. Which can restrict improvements, but means that no outdated or stale data can influence the present time actions or learning.
  This leads to more stable learning in stochastic environments \todo[color=green]{citation?} while ensuring policy consistency.

  However, it also means that the agent becomes sample inefficient, as fresh iterations are needed to learn and sharpen the current policy. On top of that, the learning rate slows down because the agent can't use older experience (only the latest policy), slowing training \cite{andrychowicz2020onpolicyRL}.

  \subsection{Off-Policy}
  In this case, the agent can utilise past experiences, regardless of the policy generated by the training. This recycling of information previously known makes the learning more efficient \cite{uehara2022reviewoffpolicyevaluationreinforcement}, as sample efficiency is increased.
  Having past experiences also means that the agent will need fewer exploration steps and less iterations because of these priors. Making this system efficient in deterministic environments. \todo[color=green]{not quite sure why, add here}
  
  This however, brings instability in stochastic environments, and the the policy diverging due to incorporating outdated and possibly uninformative (at least at the present) data. Therefore, some sort of importance of older steps should be kept track of, and maybe even decayed like the discount factor from MDPs (\ref{sec:mdp}). \cite{maroti2019rbed} \todo{check this evidence not sure, and link to earlier}
  \\\\
  Both of these approaches can be combined with either \textbf{Model-Based} or \textbf{Model-Free} approaches. Because at the end of the day, making a good model is about picking the correct trade-offs that is relevant to the scenario of the problem we are attempting to solve.

% Imitation Learning
\section{Imitation Learning (IL)}\label{sec:il}
  While traditional Reinforcement learning models focus on interactions with the environment to optimise a reward signal to find an optimal policy, these can be sample inefficient, or very challenging in high-dimensional tasks. So, another approach we can take in teaching robots is Imitation Learning (IL) where an agent learns directly from expert demonstrations \cite{attia2018globaloverviewimitationlearning}, potentially bypassing the need of extensive exploration such an adjacent RL system would need. Imitation Learning, or as some literature refers: \emph{Learning from Demonstrations (LfD)} \cite{ARGALL2009469}; is a form of \emph{Supervised Learning} \cite{hastie2009overview,cunningham2008supervised}, where the model is presented with labelled expert demonstrations and a model is learnt from those.
  The roots of this idea comes from humans and animals \cite{bakker1996robot} learning from observations to copy movements from carers \emph{(supervisors)} for survival This approach is beneficial for systems where the exploration of the action space is dangerous, expensive or inefficient. We can broadly classify IL into two main categories, \textbf{Behavioural Cloning} and \textbf{Inverse Reinforcement Learning} each addressing the challenge of learning from demonstration slightly differently.


\subsection{Behavioural Cloning (BC)}\label{sec:bc}
 Behavioural Cloning (BC) approaches achieve this goal by  mimicking the action given to it by training a model that predicts the function between the state and the actions taken  \cite{pomerlau1991neco.1991.3.1.88, ross2011reductionimitationlearningstructured}. The expert behaviour will be recorded as a set of demonstrations (or \emph{trajectories} for a moving robot) $\tau^* = \lbrace(s_i^*, a_i^*)\rbrace_{i = 0}^N$ (for $N$ total demonstrations). Where the demonstrations are state-action pairs and \emph{$*$} meaning \emph{expert or optimal}.
 Then using supervised learning methods and treating the demonstrations as the training data we can predict a policy $\pi_\theta\left(a | s\right)$ along with a loss function $L \left( a_i^*, \pi_\theta\left(s_i^*\right) \right)$. The loss function is typically Mean-Squared Loss (MSE) for continuous and Cross-Entropy for discrete actions.

 However, one big downside of this method is that the demonstrations are heavily coupled to the model, meaning slight deviations from the given behaviour in the learnt optimal policy might lead the agent into unfamiliar states. Which then can lead to compounding errors known as covariate shift, and should be controlled for stability of learning \cite{mehta2024stablebccontrollingcovariateshift}. 
 \label{para:covariate-shift}. Another issue a method like this faces is the adaptation problem. The model does not understand but copies. So, without shaped-rewards (like RL) and dubious quality demonstrations, the agent might not be very capable.
 
\subsection{Inverse Reinforcement Learning (IRL)}
As before in \ref{sec:bc}, this method will be provided the expert demonstrations $\tau^*$. Although, this time the model will assume the expert is acting according to some unknown reward function $R\left(s, a\right)$. And differently to BC, IRL methods aim to estimate this reward function which can later be used to derive the optimal policy. Allowing RL approaches to apply to a broader set of problems.

This also allows policies to be agent agnostic (to the extend of simulation training applying to the real world, though with caveats) as the reward function is more transferrable compared to the optimal policy \cite{russell1998learning}.

And an interesting side product is the reward function that the model learns. It can be extracted for downstream applications. While being more tolerant and robust to faulty demonstrations due to errors being -ideally- outliers in the data and the predicted reward can correctly discourage such actions. 

However, demonstration quality sensitivity is still present. As well as the solution complexity disproportionately grows compared to the problem size \cite{ARORA2021103500}. Each iteration is dominated by the complexity of solving the underlying MDP with the currently learnt reward function, which is polynomial in size of its parameters. Which are exponential in the number of dimensions of the state vector. On top of this as the problem size increases the sample complexity must also increase, meaning the expert should cover more trajectories in the training set for sufficient coverage of the state space and the optimal prediction of the underlying reward. And finally, verifying that correct reward is predicted is hard as it is inherently variable what is learnt.


\section{Demonstration Quality and Abundance}
A unique challenge of learning from demonstrations is that the quality of the provided information should be good, in terms of achieving optimal solutions, so then an agent can also infer the optimal policy. But also, there should be enough demonstrations so that the agent can generalise to more scenarios.

\subsubsection{Few-Shot Learning}\label{sec:few-shot}
This paradigm allows models to generalise behaviours from a limited number of demonstrations. This is done to overcome the problem of having to provide large collections of manually curated or at least verified data \cite{fewshotsurvey}. A usual approach is to use generative models (see \ref{sec:gail}) to create more data from the distributions of the data available. Main reason is to mitigate issues such as covariate shifts in the sampled data. An extreme end of this paradigm is \emph{one-shot} learning (such as \cite{vitiello2023one}) where the learning must be done on a single demonstration.

\section{Imitation or Reinforcement? - Hybrid Approaches}

\subsection{Offline Reinforcement Learning}
This ideas extends the \emph{data-driven} paradigm of IL and composes it with traditional RL approaches. As opposed to Online Reinforcement Learning, which iteratively collects experience and interacts with a given environment which is then used to improve the policy for the next episodes of iteration. This can be impractical due to data collection being costly (e.g. in robotics and healthcare domains) and sometimes dangerous (like autonomous driving). 

So offline RL, relies on previously collected data instead of fresh environmental interaction \cite{levine2020offlinereinforcementlearningtutorial}. With help from advancements in deep learning, it has been made possible to create generalisable \emph{decision making engines} \cite{levine2020offlinereinforcementlearningtutorial} if sufficient prior data can be obtained.

\subsection{Fighting Distributional Shifts: DAgger}
Another interesting solution to the problem mentioned above \ref{para:covariate-shift} is the Dataset Aggregation (DAgger) \cite{ross2011reductionimitationlearningstructured}. This is used to fight distribution shifts in standard BC \ref{sec:bc}. A policy is trained on expert demonstrations by because of the shift, there may be non-encountered states during deployment. So, DAgger iteratively collects new data; this new data, in the form or state-action pairs gets added to the training and gradually improves robustness of a policy against novel situations. Although, the constant expert supervision, while generating new data makes it costly for real-world applications; or where a algorithmic reward system is not in place.

\subsection{Generative Adversarial Imitation Learning (GAIL)}\label{sec:gail}

GAIL \cite{ho2016generativeadversarialimitationlearning}  takes inspiration form Generative Adversarial Networks (GANs) \cite{goodfellow2014generativeadversarialnetworks} and can create model-free algorithms for IL in robots. GANs are deep learning models for generative tasks, they are mainly given a distribution of data and can generate synthetic data from following that distribution. For IL, they are useful for creating useful samples for self-supervision.
% Computer Vision 
\section{Computer Vision in Robotics}
On top of the various training and learning algorithms, another cornerstone of fully autonomous robot movement is computer vision. Through the integration of visual sensors (and possibly others to support and reinforce this perception) and advanced image processing techniques, robots can be taught to identify surroundings and make informed decisions. 

Although autonomisation is possible without vision, having a generalisable view of an environment or task allows the robots actions to also be generic executors. Take for example a factory robot assembling cars. It can be efficiently made automatic without any complicated models, and with just a simple algorithm. However this means that the environment (i.e the car parts, and maybe the work-area layout) must be presented in identical configurations for each episodic repetition of the task.
As the focus in robotics is shifting towards generic dynamically moving robots, adapting to their environments, vision becomes an inevitable sending medium.

\subsection{The Camera Model}
Camera models are essential for vision. As us humans perceive the world in an analogue manner through light, the robot must also be able to interpret its surroundings the same way. A \emph{camera} is a device that captures light in a scene and a \emph{camera model} is therefore defined to be the how that analogue information is mapped onto a 2D coordinates in a mathematical manner \cite{zhang2021cameramodels}. 

An essential process when using a camera is calibration. This is required so that we can normalise what the robot ``sees'' and using some pre-defined criteria (such as a known object or pattern) so that we can be assured the information the camera provides is within a specified degree of confidence. This uncertainty range should be as low as possible as tasks like localisation, mapping and object interactions in robotics usually require precise camera measurements.

While calibrating we need to have some idea of physical properties of the camera (\emph{intrinsic}) and information about the mappings of the scene (\emph{extrinsic}).

\subsubsection{Intrinsic Parameters}
\label{subsubsec:intrinsic}

These cover the internal characteristics of the camera, and how the captured three dimensional (3D) world data will be projected down onto the two dimensional (2D) image plane.
Some important parameters are:

\begin{itemize}
  \item \textbf{Focal Length ($f$):} The distance netween the camera lens and the image sensor. Determines field of view (FOV) and required for scaling the scene.
  \item \textbf{Principal Point (c):} Point of intersection for the optical axis and image plane. Usually at the centre of the image.
  \item \textbf{Skew (s):} Non-orthogonality factor of the sensor axes of the camera. (Often assumed to be zero.)\todo{cite or remove?}
  \item \textbf{Distortion Coefficients:} Some parameters for distortion correction. Important for camera model systems like cameras with fish-eye lenses \cite{king1989history}
\end{itemize}

\missingfigure{maybe a photo with all of these on it if that is possible}

\subsubsection{Extrinsic Parameters}
Extrinsic parameters represent the physical placement of the camera in the scenes. Such as the position and orientation of the camera in the scene. Using these values we can map  3D representation of the world the camera sees (which is in world coordinates) into the camera coordinate system.
\\

\subsection {The Pin-Hole Camera}
One of the most foundational and widely used models to describe this calibration is the pin-hole (or the  \emph{projective}) camera model. 

\subsubsection{Mathematics Behind Pin-Hole}
The light passes through a single point, called the camera centre, $C$, before it is projected onto the 2D image plane (giving the name pin-hole). 
% // TODO\missingfigure{include classic image, steal from https://www.oreilly.com/library/view/programming-computer-vision/9781449341916/ch04.html#:~:text=The%20pin-hole%20camera%20model%20(or%20sometimes%20projective%20camera%20model,a%20dark%20box%20or%20room.
% }
A 3D point $\textbf{X}$ is projected onto image point $\textbf{x}$ using the equation:
\[\lambda \textbf{x} = P\textbf{X}\]
where $P$ is the a 3x4 matrix called the camera (or \emph{projection}) matrix and $\textbf{X}$ is 1x4 and has four elements in homogenous coordinates, \(\textbf{X} = [x, y, z, w]\) and $\lambda$ is the inverse depth of the 3D point. Which can be needed if we want all coordinates to be homogenous with the last value ($w$) normalised to $1$.

The Camera Matrix, P can will all calibration values for the camera. so:
\[P = K \left[R \mid t\right] \]

R is the (3x3) rotational matrix describing the orientation of the camera, and t a (3x1) translational vector describing the position of C.
Also the intrinsic calibration matrix, K, will encode camera attributes discussed above.
\[
  K = 
  \begin{bmatrix}
    \alpha f & s & c_x \\
    0 & f & c_y \\
    0 & 0 & 1
  \end{bmatrix}
\]

Where the values are from \ref{subsubsec:intrinsic} and $\alpha$ is the aspect ratio used for non-square pixel elements (usually safe to assume $a=1$).
% Active Vision 
\section{Active Vision (AV)}
    Physical robots, by actively controlling their cameras \cite{Aloimonos1988}, as well as using attention and focus mechanisms; can identify the relevant parts of an environment. This, by introducing another level of decision making in the form of: \textbf{where} and \textbf{what} to perceive, enables the robot to move and act more intelligently within the space provided. This approach is mostly beneficial in applications such as: autonomous navigation, multi-robot control and human-robot interactions. \cite{breazeal2001hri}.
    \todo[color=green]{more citations for the examples?}
    
  \subsection{Principles of Active Vision}
  Traditional static vision can be enhanced by incorporating camera and viewpoint optimisations into the perception systems. Some principles include:
  \begin{itemize}
    \item \textbf{Gaze Control:} This is the fundamental idea of active vision. Move the camera and the viewpoint, (ideally) to regions of interest.
    \item \textbf{Next-Best-View Planning:} Strategically adjusting the camera to maximise the visual information the robot acquires. They often use \emph{attention mechanisms} which ensure they focus on the specific items or areas \cite{Burusa_2024} while ignoring the other cluttering information in the scene. These attention mechanisms can be \emph{spatial}, focusing on regions of interest; or \emph{temporal}: used with tracking moving objects.
    \item \textbf{Task-Driven Perception:} Optimising the visual input to the task at hand by using methods like \emph{next-best-pose} to ensure the robot stays focused on task-specific manipulations, such as keeping its gripped in the correct pose before reaching an object.
  \end{itemize}

  Some combinations of these fundamental ideas, fitting them to the use-case means a robot can acquire better observations of the space instead of relying on prior human knowledge to strategically pre-place the camera where it would most benefit the task.

  \subsection{Back To Markov}
  \subsubsection{Partially Observable Markov Decision
  Processes (POMDPs)}\label{sec:pomdp}
  Adding on to the definitions above in \ref{sec:mdp}, we now require \emph{Partially Observable Markov Decision Processes (POMDPs)} 
  which is an extension of MDPs where it is defined as a 7-tuple \cite{thrun2002probabilistic,placed2023surveyactivesimultaneouslocalization}: 
  \[\langle S, A, \mathcal{Z}, \xi_S, \xi_{\mathcal{Z}}, E, \gamma \rangle \]
  
  in addition to earlier, the state transition function \( \xi_S ~\colon~ S \times A \rightarrow \Pi\left(S\right)\) and $\Pi\left(S\right)$ is the space of probability density functions (pdf) over $S$. The observation space $\mathcal{Z}$, and the conditional likelihood of making any of those observations \(\xi_{\mathcal{Z}} ~\colon~ S \rightarrow \Pi\left(Z\right)\) where $\Pi\left(\mathcal{Z}\right)$ is the space of pdfs over $\mathcal{Z}$.
  Differently to the fully observable case, agents here cannot reliably know their own true state. So, they maintain an internal \emph{belief} system (historic), $b_t\left(s_t\right)$ which represents the posterior probability over states at time $t$, given the available data collected up until that time. \(b_t\left(s_t\right) \triangleeq  \mathbb{P}\left(s_t \mid z_{1 \colon t}, a_{1 \colon t - 1}\right)\) where the given is the history, $h$.
  \\\\
  Introducing this belief system we can now model active tasks which rely on exploring the space for the optimal states and selecting an action that decreases the uncertainty in the belief state, leading to actively informed decision-making.

  
  \subsection{Active Vision in Practice}
  There are a number of ways to implement active vision in practice. The two main temporal aspects are \emph{synchronous} and \emph{asynchronous} vision-action models \cite{divyaHandEyeCoordsination}. Synchronous models need to make real-time decisions on the next action based on the current observation information. This can be challenging in practice as the sensing mediums must be well coordinated and the observation may not contain informative information yet, causing issues with capabilities of a model\label{sec:asynch-synch}

  A more realistic model is the asynchronous one. It is more human-like in the sense that decisions in movement and viewpoint usually do not occur simultaneously, although quickly we usually adjust our views then act. So this creates a nice vision to action pipeline that can be more effective for agents learning active vision policies.
  

  \subsubsection{RL for Active Perception}
    Framing this problem as a decision-making one, so: ``Where should the agent face to maximise reward?'' allows us to come up with a policy which can learn where to look next, optimise the pose of the camera to minimise uncertainty in decisions and adapt to different environments \cite{rothbucher2011,zhangembodied}. Which can then be modelled as a POMDP system to then be solved as classical RL, giving us a way to control the gaze during a tasks execution.

  \subsubsection{Predictive Control}
  Instead of allowing random exploration of the viewpoint space we can also train models that predict the next best view using techniques such as Entropy-based viewpoint selection and Bayesian optimisations with the use of prior information such as the knowledge of the scene, object models' geometry and symmetries \cite{dhami2023prednbvpredictionguidednextbestview3d,breyer2022closedloopnextbestviewplanningtargetdriven}
  \\\\
  With these implementation strategies we can use active learning for tasks such as:
  \begin{itemize}
    \item \textbf{Dexterous manipulation robots} (i.e. grasping robots), where a eye-in-hand camera (maybe) coupled with depth sensing can refine its understanding of its surroundings and actively adjust views to avoid occlusions. 
    \item \textbf{Autonomous navigation,} mobile robots can use approaches like Active SLAM \cite{s23198097}, to map their environments and refine their understanding of it constantly. Some examples may be aerial drones continuously exploring and learning more about different regions \cite{drones6040085}.
    \item \textbf{Human-robot interactions,} social robots can be made to keep eye contact or learn to track human mannerisms from gestures and expressions to make them more realistic in interactions.
  \end{itemize}
    
  \subsection{Challenges}
  Despite all the discussed possibilities, Active Vision still remains fairly theoretical and unexplored compared to other areas of robotics. Although, the ideas have been floating around in computing circles for almost three decades, there are several challenges that need to be overcome for active vision to be viable for all its possible applications. These issues include:
  \begin{enumerate}
    \item \textbf{Computational Complexity}
    \item \textbf{Sensor, Hardware and Environmental Limitations}
    \item \textbf{Integrations with other Sensing Mediums
    }
  \end{enumerate}

  Although, these cannot be classified as \emph{solved} there are some approaches in tackling this active vision perspective of robotics which we will review in the next chapter.


\chapter{Related Work}
  In this chapter we will discuss:
  \begin{enumerate}
    \item papers
    \item Vision-action robot manipulation:
    \item Grasp Synthesis with AV Methods paper. Heuristic and ML (self-supervised and Q learning) approaches compares to other algorithms as well as real life most optimal solutions for evaluation.Q-learning performs worse in real life tests, mostly due to differences in the depth sensor acting differently. shows 3d heuristic performs better compared to ML and naive algorithmic approaches. ML is more sensitive to sensor noise so the heuristic approaches generalise better to the real world from simulations.
    \item Closed-Loop NBV Planning for target-driven grasping. This paper talks about active exploration of the space, during test time. policies are already trained (system called MoveIt is used) at every it generates the best grasp candidate and the next best view along with the information gain. Then depending on the stopping criteria it will execute this task, or check the next-best-view found. Stopping criteria to balance exploration and exploitation. Example of exploration and reasoning to find the optimal pose, number 1 from 1.2 Objectives. Downsides is that it requires prior information about the target object's bounding box. Mostly heuristic I think
  \end{enumerate}
\section{Active Vision Resaerch}

As active vision within robotics is an emerging area of research, there isn't a consensus on a single approach to tackle this issue. On top of this aspect, by the inherent nature of differing tasks requiring differing approaches, a general solution may not be possible for every task. However, using techniques highlighted earlier and building on top of these we can come up with frameworks that are competent and most importantly highly competitive compared to current static camera systems.

Here we will highlight some significant contributions that are similar in scope to what we are aiming to achieve in this project.

\section{Constraints in Active Vision}
The main constraint is that for effective and optimal learning large number of demonstrations must be fed into an agent. However, the manual aspect of generating ``expert'' demonstrations is not always possible. This naturally leads the literature to be mainly concerned with approaches that deal with few-shot learning (see \ref{sec:few-shot}) and self-supervision.

A direct by-product of this constraint is incorporating prior knowledge into the learning. Such as the information about an object in an \emph{object-centred} problem, meaning the task revolves around manipulation of specific object or objects. This allows to compensate for lack of large set of demonstrations. These could include object poses \cite{huang2018generalisedTPlearning, hu2023modelpredictiveoptimisation} or meta-learning policies from pre-trained models on desirable datasets \cite{finn2017oneshotvisualimitationlearning, mandi2022}.


\section{Object Priors}
Following the constraints, an active perception learning robot is usually given information about the task and the objects that are important to that task, that way extracting the relevant policies gives us a baseline on the prominence of such a policy.

\subsubsection{\emph{Robot See Robot Do}}
This is explored in \emph{Kerr et al.}, where the work revolves around teaching a robot to interact with manipulable objects, such as chests, drawers, glasses and toys \cite{kerr2024robotrobotdoimitating}. At its core this is a grasping task integrated with one-shot learning, however, with the important prior that the object's 4D model -recovered by 4D Differentiable Part Models (4D-DPM). This way the we can observe what the robot can synthesise the correct manipulation points from its viewpoint and generalise this to all given viewpoints. Although, this isn't immediately an active-vision robot, this is a good starting point in understanding to design systems that are using their perception to make meta-decisions that influence their movement in their main policies. 

\section{Semi-Active Vision}
The simplest idea in teaching a robot to see in a human-like manner is to train it on data directly generated by human interactions. This also means the subject specific information like priors can be inferred by the policy, instead of being explicitly provided by the researchers.

\emph{Chuang et al.} explore the idea of teaching a robot the task policy joint with the policy of moving a camera fitted arm \cite{chuang2024activevisionneedexploring}. Their setup includes a Virtual Reality (VR) Headset, which allows the demonstrator to move the AV camera on the robotic arm. This allows them to teach policies for tasks where the subject may be blocked by small static cameras around the scene, or the camera fitted in a eye-in-hand configuration being occluded or out-of-bounds due to orientation of the gripper with respect to the task. 

Although, their research is promising they acknowledge that active vision also brings some issues that need solving. Some important mentions are: operational and architectural complexity, though, they used similar architectures for all their different camera configurations, they note that a bespoke system for AV might benefit such a system. 

The expanded action space also poses an issue as the state space complexity explodes, and hint that decoupling the vision and the control system might help with the issue. Finally, they touch on distribution shifts being a big issue, this mainly stems from the earlier implicit subject information. As the model is not aware of targets it will learn what it can infer, so if the demonstrations do not contain enough variations in poses, generalising the  locations and other characteristics for an object become tougher. A possible strategy in solving that particular issue is generating augmented samples, to fight this covariate shift.

\section{Self-Supervision and Data Augmentation}
Another widely used strategy to mitigate providing vast amounts of manually crafted and hand labelled data is to subscribe to the idea of self-supervision. These are used to counteract the instability of reinforcement learning and the inefficiency of random exploration.\todo{link to earlier definition from cv/av}

\subsubsection{\emph{Making Imitation Learning Easy with Self-Supervision: MILES}}
Incorporating this into robot learning tasks usually takes the form of generating augmented movement trajectories. Which are simulated and sampled from the limited number of human trajectories (usually accepted to be expert behaviour) given to the agent. \emph{Papagiannis et al.} using the \emph{MILES} framework, simplifies the process by removing the human intervention aspects from highly repeatable supervision parts of the learning \cite{papagiannis2024milesmakingimitationlearning}. 

This mimics exploration-based RL learning systems where given a list of trajectories from the expert behaviour,the agent will move to a pose near the demonstration and attempt to move itself back to original trajectory. In the process creating an augmented path that eventually joins with the ``correct'' one. As it collects sensor data along the path (per waypoint) it therefore, creates augmented data that can now be used to aid its training. So, it is self-supervised in the way it collects data. As no interaction is needed to correct the agent back to starting position or other environmental resets (assuming the learning tasks doesn't manipulate the scene in a non-recoverable way),

This is achieved by training a separate policy for each task as a \emph{LSTM} (long-short term memory) network, based on Behavioural Cloning, which is a type of Recurrent Neural Network (RNN) which handles sequential data \cite{medsker2001recurrent}.This allows the policy to learn the gradually changing trajectory while remembering the steps taken in the past. On top of this no object pose priors are given to the network, meaning the networks learnt policies should be applicable to different object poses.

However, another important part of \emph{MILES} is that force sensors are also included in the decision making policies. Which is not a make-or-break addition, as they conclude, the force modality sometimes helps the system achieve better accuracy when coupled with vision, and sometimes not. While just force -without vision, in a somewhat expected manner- performs quite badly in any of the evaluation tasks they have chosen. 



% \cite{natarajan2021graspsynthesisnovelobjects}
\todo[color=green]{this (see comment in file) also kinf of goes under learing using heuristics but not really trajectory maybe move to next section and so some explanation if needed}

\section{Attention and Information Gain (IG)}
This is pivotal part of object-centric tasks. Because, if an agent knows what to focus on, then we can teach it a policy to learn to focus on specific subjects.

\subsubsection{\emph{Observe Then Act}}
Taking a third-person-view look to the classical grasping task, \emph{Wang et al.} aim to optimise this viewpoint based on the task goal \cite{wang2024observeactasynchronousactive}. They take a asynchronous viewpoint control approach (see \ref{sec:asynch-synch}); this separation of the camera systems and the motor actions over time. Leading to less need for coordination between the two section and instead, the model focuses more on task-specific movements and distribute this coordination throughout the task. 
They, again, follow a few-shot learning approach. The model comprises of two separate agents, a next-best-view (NBV) agent for optimal viewpoints and next-best-pose (NBP) for determining the gripper's action based on the previous agent's output. Active perceptions is achieved by alternating between sensor and motor action interfaces in each episode; which then leads to the learning of the tasks. These tasks are usually of sequential nature as the asynchronous approach works best with such systems, as they discuss.
\missingfigure{the neuron figure form the paper}

They follow a POMDP (\ref{sec:pomdp}) formulation to model the problem:
\[
  \langle \mathcal{O}, \mathcal{A}^c, P, \mathcal{A}^g, \mathcal{O}', P', R, \gamma \rangle
\]

where $\mathcal{O}$ and $\mathcal{O}'$ represent the observation spaces at times $t$ and $t'$ the NBV policy $\pi_v$, determines the camera viewpoint action $a_t^c \in \mathcal{A}^c$, given an observation $o_t \in \mathcal{O}$, and obtains a new observation $o_{t'} \in \mathcal{O}'$ through the transition probability $p\left(o_{t'} \mid o_t, a_t^c\right) \in P$. Finally, the NBP policy, $\pi_g$, then determines the gripper action $a^g_{t'} \in \mathcal{A}^g$ based on the new observation $o_{t'}$. then using the transition probability $p'\left(o_{t+1} \mid o_t, a^c_t\right) \in P'$ the next scene observation can be obtained and reward $r_{t+1} \in R$ will be provided. So, the model will be learning these policies: $\pi^*_v$ and $\pi^*_g$ then try to jointly maximise the return for the joint reward for this collective task: 
\[
  \pi_v^*, \pi_g^* = 
  arg~\underset{\pi_v, \pi_g}{max} 
  ~\mathbb{E}
  \left[
    \sum_{t=0}^{\inf}{\gamma^t R(o_t, a^c_t, a^g_t)}
  \right]
\]
\todo[color=green]{more information about the 3D voxels for environment normalisation?? this is more about fighting the caveats of few-shot learning not really fitting here, but might include if it comes in useful later}

Another important contribution here is the use of augmented trajectories again. Similar to \cite{papagiannis2024milesmakingimitationlearning}, demonstration trajectories are augmented in a \emph{viewpoint-aware} manner to expand the learning set by sampling the observations and discovering key frames. It does this while the viewpoint is allowed to shift, meaning the samples getting generated between time frame $T_t$ and $T-{t+1}$ (where $T$ is a trajectory) can have different views, which aids the camera policy in progressive movements. This is done for keyframes for both camera movement and gripper pose, for the two policies.

\subsubsection{Closed-Loop Next-Best-View Planning for Target-Driven Grasping}
Similar in idea, \emph{Breyer et al.} explore yet another grasping task (although first-person-view this time) this time. They augment classic grasp synthesis tasks that are mostly reliant on deep learning approaches, which suffer depending on the visual information it has available. Similar to \emph{Wang et al.}, they interact with the environment in an asynchronous manner within a fixed rate. They determine the best candidates for grasping then compute the next-best-view with its associated information gain. 
The information gain metric, allows the system to explore viewpoints in the neighbourhood of the current view and estimate what might be ideal for the grasping, bridging the rewards of the two policies. This again is made possible through the assumption that a task is object dependent hence the model receives bounding boxes and the geometry of the subject of the task.

However, the interesting takeaway from their system is an early stopping mechanism. Due to the nature of exploration multiple views can be scanned and many relevant ones can be identified by the system. Without stops, this algorithms may over explore and end up quite inefficient. These stopping conditions include: timeouts (and due to time-framing, limited number of policy updates), minimum thresholding on the information gain so that no unnecessary exploration is done where the estimated IG might be lower; and finally, convergence, when the Volumetric Grasp Network (VGN) \cite{breyer2021volumetricgraspingnetworkrealtime} outputs converge the exploration will stop. 

Therefore, the importance of their research is highlighted in the robustness of their system and the balance they managed to strike between exploration and exploitation. 

\todo[color=green]{talk about attention and active perception}

\subsection{Outline}
In summary, work in the field of active vision usually follows an attention metric which is guided by a reward system, usually depending on some priors. The need for priors, although, can be mitigated by providing more demonstrations, this complicates the test setups and operations. Therefore, a common approach is to create augmented data and broaden the horizon of the agent through the exploration of that synthetic space.

Combining these ideas shows us that active vision policy adjustments are not only possible during policy execution but also very promising way to advance the field of robot learning.



%//CHECK interesting paragraph not sure about it tho
% Another interesting point is the relevance of active vision for a task which isn't object-centred. 
% Does the use of active vision make sense to a robot where there isn't a defined subject? Or should the agent explore and find objects of importance?
% \todo{not sure about this paragraph, does this make sense maybe for the final report}

\todo{note for evaluation, section on real-life testing and simulation tsting separately, to observe potential differences}