\documentclass[a4paper, twoside]{report}

% \includeonly{
%   % introduction/introduction,
%   % background/background,
%   % project/early-work/ew,
%   % project/cam-comb/cam-comb,
%   % project/appl/appl,
%   evaluation/evaluation,
%   conclusion/conclusion,
%   % appendix/appendix
% }

%% Language and font encodings
\usepackage[english]{babel}
\usepackage[utf8x]{inputenc}
\usepackage[T1]{fontenc}
\usepackage{tabularx}
%% Sets page size and margins
\usepackage[a4paper,top=3cm,bottom=2cm,left=3cm,right=3cm,marginparwidth=1.75cm]{geometry}
\usepackage{tikz}
\usetikzlibrary{shapes.geometric, arrows, positioning}
\usepackage{subcaption}

% Datetime
\usepackage[en-GB]{datetime2}
\DTMlangsetup[en-GB]{ord=raise,showyear=false}

%% Useful packages
\usepackage{amsmath, amsfonts}
\usepackage{graphicx}
\usepackage{MnSymbol}
\usepackage[colorinlistoftodos, disable]{todonotes}
% \setuptodonotes{color=green}
\usepackage[colorlinks=true, allcolors=blue]{hyperref}
\usepackage{minted}
\usepackage{float}
\usepackage{multirow}

\title{APPLe: Active Perception Policy Learning and Multi-Modal Feature Combinations}
\author{Kağan Uğur Pekgöz}
% Update supervisor and other title stuff in title/title.tex

\begin{document}

\include{title/title}

\begin{abstract}
  
  A significant obstacle in robot and computer vision is incomplete data. Information is critical, without knowing; it is impossible to act, react or plan next steps. Robots who plan on external stimuli are no different. To localise and understand where they are and what they should be doing next; they need an accurate representation of the environment around them. And when they are not given accurate data, they need to do something about it.

  Vision, is a critical system. Us humans rely on being able to see, to go bout our daily lives. However, the most important skill we have is reasoning. Sight only attaches onto that as an external sensor. When we cannot fully see something we can reposition, or look at it differently. We can understand what we see, or at least try our best to understand what we are seeing. Robots can benefit from this, but how do we make them \emph{understand} what they see.

  In this project I explored the idea of what a robot can understand from a scene. I created toy scenarios in reaching and grasping tasks. Using Imitation Learning, I demonstrated to an agent how each task can be solved and exposed it to different modalities of inputs. Then tested its ability to extract information by a systematic modular multi-modal fusion framework to assess and evaluate its performance.
  
  Once I quantified what the agent understood I then formulated \emph{active vision} policies that are not fully end-to-end due to the intricate nature of tuning specific modules. And evaluated the performance of a 3D reasoning agent in the toy tasks.
  
  I find that not enough data and too much data can cause problems of similar nature and limit the true potential of an agent learning to execute a task. At the same time, active intricate systems with limiting prior knowledge requirements that can actively seek within an environment; will show double the success of naive agents. However, at the cost of being calculated yet slow.
\end{abstract}

\renewcommand{\abstractname}{Acknowledgements}
\begin{abstract}
I would like to thanks all my friends for all their support during my time in university and life in general.
Would also like to thank my supervisor for proposing such a fascinating project, although I though I was done with ML after doing all ML modules. I really enjoyed working with, and being frustrated at robots.
And lastly would like to thank my family for their unending support and love.
\end{abstract}

\tableofcontents
% \listoffigures
% \listoftables

\chapter{Introduction}
\section{Motivation}
    
  Sight is one of the most fundamental senses we - humans- posses. \todo{stats about vision, and the interesting paper I found on visual ensemvles and summary statistic} Our brains are evolved to process information quickly to allow us to reason with this information and react external stimuli. We also rely heavily on this skill to navigate the world; make sense of the unknown by observing it. Robots are no different. As the robotics industry evolved over the many decades, Computer Vision \todo{ref somethinhg} Based solutions became a staple in allowing machines to \emph{see}.

  Reinforcement \todo{ref} and Imitation Learning \todo{ref} are two prominent methods in tuning such visual robots. Where a robot will be fitted with an assortment of cameras and sensors to perceive its environment. These cameras can be mounted in many configurations: Mounted on their wrists \cite{chi2024UMIinthewild,openXEmbodimentRoboticLearning2024}, over the shoulder \cite{wang2024observeactasynchronousactive}, or in some cases placed around the environment \cite{exploringActiveVision2024chuang} that the robot is interacting to allow entire scenes to be observed. These configurations are not mutually exclusive and multiple can be used to maximise information gain. A major problem this introduces is cost and mobility:
  \begin{itemize}
    \item Third-person cameras are challenging to incorporate into non-static tasks. And on-body solutions, like wrist-mounting, provide limited visual understanding with occlusions and increased environmental entropy becoming a challenge. 
    \item Multi-view setups are usually big and clunky \todo{ref here}, losing on dexterity and more importantly stop being general-use robots
  \end{itemize} 

  To take inspiration from humans: we use our eyes which can be operated independently from our other extremities. This optimising aspect of human visual feedback \cite{findlay2003active,maiello2021humans,goodman2018using} is one of the cornerstones of human learning and environment manipulation. Take, for example, a USB stick and its port. It is unlikely (impossible even) to ensure success in the first insertion attempt. However, shifting our view to make sure the port, the USB stick, and our hand are \emph{in-frame}, meaning we can see them clearly. The success rate of the next attempt sky-rockets.
  
  Therefore, a possible solution to robot observation uncertainty is to actively explore the environment and understand the environment before acting. In this project I will explore what it means for a robot to view and understand what it is seeing. And explore geometric reasoning and uncertainty ensemble methods to overcome problems arising from static observations.


\section{Contributions}
  This is an experimentation-driven project. With end goal being of understanding 3D tasks and robots' reasoning of them to create competent active vision policies. The work is laid out as follows:

  \begin{itemize}
    \item Explore CoppeliaSim and propose some reaching and grasping tasks for training my policies on (Sections \ref{sec:3d-reaching-tasks}, \ref{sec:grasping-tasks}, \ref{sec:reach-obs})
    \item Propose and explore ideas of feature fusion to increase the learning potential and the competency of robot policies (Sections \ref{sec:depth-interfacing}, \ref{sec:multi-modal-policies}, \ref{sec:proposed-fusion})
    \item Design and reason about the foundations of two active vision policies and their implementations (Chapter \ref{ch:appl})
    \item Finally, evaluate and compare the competencies and potential drawbacks of the various policies and feature fusion methods (Chapter \ref{ch:eval})
  \end{itemize}


% \chapter{Background}
\chapter{Technical Background}
\section{End-to-End Learning for Robot Control}

  End-to-end refers to a robot learning approach where the robot determines certain \textbf{policies} (actions to perform a certain task) from raw inputs from the action space. The action space can include anything that we think the robot may benefit form knowing. This means the 
  robot learns to map sensory inputs directly to motor commands, bypassing the need for intermediate steps such as feature extraction or state estimation. This approach leverages deep learning techniques \cite{Schmidhuber2015nn}, particularly convolutional neural networks (CNNs) and recurrent neural networks (RNNs), to process high-dimensional sensory data and generate appropriate actions. Which can then be used in techniques like Reinforcement Learning \ref{sec:rl} or Imitation Learning \ref{sec:il}. Therefore, recent advancements in machine learning technologies has also reshaped the field of robotics and helps move it forward \cite{Pierson18082017,newbury2023graspSynthReview,liu2021DRLminireview}.

  \missingfigure{add 2 figures as a small outline like a blackbox + a few steps}

  This contrasts with the classical approaches\todo{citations here from the older theory papers maybe}. Classical robotics involves separating the behaviour of a robot into smaller tasks, where each task is managed by a distinct module and the system is functional when the pipeline comes together. Although good at precisely executing repetitive tasks, this approach requires complex and often handcrafted solutions for each module. This can lead to difficulties in scaling and adapting to new tasks or environments. 
  
  Therefore, the cutting-edge research in robotics concerns end-to-end systems in making multi-modal robots, which can be tuned for specific tasks if needed, using their capabilities of complex decision-making withing the given environments.

% Math Foundations
\section{Mathematical Foundations}

  In scenarios involving autonomous acting, the capability of a an robot to reason and navigate complex problems or dynamic environments plays a central role. So in any RL system the agent must make a sequence of decisions that impact later outcomes. Markov Decision Processes (MDPs) provide a mathematical framework for modelling decisions in environments where the probabilistic outcomes are influenced by actions of such an agent. A formally defined MDP will have the following parts \cite{silver2015}:

\subsection{Markov Decision Processes}
  \subsubsection{State, S}
    The system must be able to process all different configurations of the environment. So the state will encapsulate the surroundings through raw sensory inputs (e.g. images, force sensors) in a high-level representation and will capture all relevant information available \cite{Sutton1998} which might be needed to make an informed decision at a particular time step.

  \subsubsection{The Markov Property}
    A state $S_t$ is \emph{Markov} if and only if:
    \[
      \mathbb{P} \left[S_{t+1} \mid S_t\right] = \mathbb{P}\left[ S_{t+1} \mid S_1, \ldots, S_t\right]
    \]

   % \todo{maybe make this a lemma type table??}
    This property ensures that all relevant information from the history is captured within the state. So once the state is knows the past states can be discarded. Making the current state a sufficient statistic for the future \cite{silver2015}.

  \subsubsection{State Transition Matrix, P}
    This matrix defines the transition probabilities from all states $s$ to all successor states $s'$, so:
    
    \[ P_{ss'} = \mathbb{P} \left[S_{t+1} = s'  \mid S_t = s\right]\] 
    and 
    
    \[ P =
    \begin{bmatrix}
      P_{11} & \cdots & P_{1n} \\ 
      \vdots & & \vdots\\
      P_{n1} & \cdots & P_{nn}
    \end{bmatrix}
    \]
    where each row of the matrix sums up to 1, due to the nature of probabilities. A tuple of a set of states and a transition matrix/function \(\llangle S, P \rrangle\) make up a \textbf{Markov Process} (or Markov Chain)

  \subsubsection{Actions, A}
    This is the set of all possible actions that are available to the robot in each state. Depending on the context of the task, actions can be discrete or continuous.

  \subsubsection{Reward Function, R}
    This is the scalar feedback signal. It ensures that the agent's learning is based on steps it is taking over time, so $R_t$ is how well the agent is doing at step $t$. It is defined as:

    \[R_s = \mathbb{E} \left[R_{t+1} \mid S_t = s\right]\]
    
    This allows the robot to eventually converge to a solution that maximises the cumulative rewards (the \emph{returns}) for the actions it has taken. 
  
  \subsubsection{Discount Factor, $\gamma$}
    This is mechanism to control the importance of future rewards. Sampled as: \(\gamma \in \left[0, 1\right]\), means that immediate reward is prioritised and while reward form longer sequence of actions decays, avoiding infinite cycles in Markov Chains.

    Combining the reward function and the discount factor we can defined the \emph{return}, $G_t$ as the total discounted reward from time-step $t$:

    \[ 
    \begin{aligned}
      G_t &= R_{t+1} + \gamma R_{t+2} + \ldots \\ 
      &= \sum_{k=0}^{\inf}\gamma^k R_{t+k+1} 
    \end{aligned}
    \]
    
    Therefore, a \emph{Markov Decision Process} is a tuple \(\langle S, A, P, R, \gamma \rangle\) as the parts are defined above.
    
    
  \subsubsection{Policy, $\pi$}
    The higher-level goal of any RL system is to learn an optimal policy \(\pi \left( a \mid s\right) = \mathbb{P} \left[A_t = a \mid S_t = s\right]\) which aims to maximise the return. Policies fully define the behaviour of the agent. As seen by the function's type \(\pi: S \rightarrow A \) they only depend on the current state and are time independent \( A_t \sim \pi\left( \cdot \mid S_t\right), \forall t > 0 \)
    
  \subsubsection{Value Functions}
    On top of these we define two value function, \emph{state-value}: the expected return starting from state $s$, and then following policy $\pi$:

    \[ v_\pi \left(s\right) = \mathbb{E} = \left[G_t \mid S_t = s\right]\]

    and the \emph{action-value} function, which is the expected return starting from state $s$, taking action $a$, and then continuing a policy $\pi$:

    \[ q_\pi \left(s, a\right) = \mathbb{E} \left[ G_t \mid S_t = s, A_t = a\right]\]


    while the optimal versions can be defined as:
    \[v_* \left(s\right) = \underset{\pi}{\max} \ v_\pi \left(s\right)\]

    \[q_* \left(s, a\right) = \underset{\pi}{\max} \ q_\pi \left(s, a\right)\]
    
    The optimal means the best possible performance can be achieved in the MDP and it can be considered ``solved'' once we find these optimal functions.

    \todo{should I define the bellman function?}

    

% Reinforcement Learning
\section{Reinforcement Learning (RL)}\label{sec:rl}
  One of the tried and tested methods of end-to-end learning approaches is a branch under machine learning called \emph{Reinforcement Learning (RL)}. 

  \begin{figure}[h]
    \centering
    \includegraphics[width=0.6\textwidth]{assets/background/silver-rl.png}
    \caption{Simple RL diagram to reinforce intuition \cite{silver2015}}\label{fig:rl-diag}
  \end{figure}

  RL's main focus is training algorithms, which are called \emph{agents}, in making optimal decisions by interacting with the environment. The key objective is to teach a \emph{policy} to the agent that maximises the overall reward -usually defined by the task and involves a reward or \emph{teacher}. The agent will explore, the possible actions it can take through trial and error, while learning from the feedback given to it by the reward signal and its environment.

  One of the differentiating factors of RL from classical machine learning paradigms is that the feedback is not instantaneous and sequences of decisions influence the subsequent data and signals given to agent \cite{silver2015}.

\subsection{RL in Practice}
  Reinforcement learning, can be used to train a variety of agents that is not limited by physical robots. It can learn to play video-games \cite{comi2018}, automation tasks \cite{} , in natural language processing \cite{paulus2017deepreinforcedmodelabstractive}, applications in healthcare (where RL is categorised as dynamic treatment regimes) for use in chronic diseases or critical care \cite{yu2020reinforcementlearninghealthcaresurvey} and lastly -most importantly for us- learning movement behaviours for robots.

  \subsection{Exploration and Exploitation}
  
  A fundamental challenge in utilising RL is the constant act of balancing \textbf{exploration} and \textbf{exploration}. A trade-off must be made in the created system.
  \begin{enumerate}
    \item \textbf{Exploration:}
    This is when the agent decides to \emph{explore} new actions that might potentially lead to better long-term outcomes. An issue this can cause is that the time it takes to explore all possibilities might not be feasible. But crucial to utilise in in problems with sparse reward models.\todo[color=green]{citations}.
    \item \textbf{Exploitation:}
    When the agent prioritises short-term, immediate rewards by exploiting its current knowledge. For example, taking an agent playing a video game; if a high score was found the agent might be unaware that a higher score can be achieved with a different set of moves. 
  \end{enumerate}

  The trade-offs must be balanced between the two in any RL algorithm or model while considering sampling efficiencies and ease of training \cite{liu2019simpleexplorationsampleefficient}. To relate it back to a robotics example, too much exploration might lead to inefficient training and instability; while too much reliance on exploitation might lead to suboptimal behaviours in the movement of a robot executing a task. 

  % Some common strategies are: $\epsilon$-Greedy, Decay $\epsilon$-Greedy, Upper Confidence Bound (UCB), Thompson Sampling, Intrinsic Motivation (Curiosity-Driven Exploration).
  % \todo[color=green]{citations for these or just remove?}
  
  Along with this widespread use and elemental challenges, comes differing methods of utilising the RL framework. The likes of which can be broadly classified into two types: \textbf{Model-Based} and \textbf{Model-Free}.
  
  \subsection{Model-Based RL}
  Model based approaches involve methods of creating an internal model and representation of the state the agent is interacting with. It usually involves two main steps: learning the dynamics of the model, then planning and learning within it \cite{MAL-086}. These models have underlying principals of the Markov Decision Process (\ref{sec:mdp}).

  Main power of this approach comes from its simulated core. As fewer real interactions can allow it to have a higher sample efficiency \cite{liu2021DRLminireview,wu23robotLearn}, meaning the amount of experience needed (mostly relating to time spent) to learn optimal policies is quite low, and good policies can be quickly learnt.

  On top of that having an underlying model essentially allows the agent to understand its surroundings better, without having to guess or learn them. This means the model can focus on learn the model and generalise better for unseen states \cite{MAL-086}.
  
  However, it also has some drawbacks. Mainly the model introduces a bias into the system, which means inadequate representations or faulty models can create policies that exploit deficiencies in these models \cite{Deisenroth2011PILCO,wang2019benchmarkingmodelbasedreinforcementlearning}, although, some recent works have helped alleviate that bias by characterising the uncertainty of these learnt models \cite{kurutach2018modelensembletrustregionpolicyoptimization,chua2018deepreinforcementlearninghandful,clavera2018modelbasedreinforcementlearningmetapolicy}. Therefore, modelling is a manual, human task which can lead to human-error or insufficient coverage of the action space, leading to suboptimal results.

  One of the most important issues being the learning of the dynamics being coupled with the policy. This makes the agent more prone to performance local-minima, which stem from exploitation and off-learning not being fully investigated under model-based approaches.
  \todo[color=green]{verify this is true and check some sources}

  \subsubsection{Applications in Robotics}
  These methods can be used for motion planning, trajectory optimisation and learning from limited interactions, some common methods are: Monte Carlo Tree Search (MCTS),Probabilistic Inference for Learning Control (PILCO) \cite{Deisenroth2011PILCO}, MuZero: Combines model-based approaches with deep learning.\todo[color=green]{cite here}
  
  \subsection{Model-Free RL}
  This collection of schemes attempt to learn a policy directly by trial and error, without explicitly modelling the environments dynamics. Usually preferred when the environment is too complex or too costly to model.

  There are a few different variants of model-free approaches, all of them similar in the way they omit a model, but have different methods in extracting the optimal policies.
  

  \subsubsection{Value-Based}
    These techniques aim to learn the value of states (or an estimate for the value of states) and actions. So they learn the $v_\pi$ or $q_\pi$ function. which can then be used to extract the optimal policy $\pi_*$ for deciding the actions.

    Such systems are mainly used for simple navigation tasks, basic motor control and arm reaching scenarios. Some example methods are: Q-Learning [Off-Policy], Deep Q-Networks (DQN) [Off-Policy], SARSA [On-Policy]. \todo[color=green]{citations}

  \subsubsection{Policy-Based}
  This on the other hand learn a policy directly. Bypassing the need to learn the values of states or actions completely. This can be helpful if the state space and/or the action space are quite large. For example, if the action space was infinite, then value-based approaches would not be feasible as all actions must be tried to find the best, which makes directly learning the policy is the only possible approach.

  These systems are for more complex control tasks, such as dexterous limb manipulation, robot hand grasping. Some examples are: REINFORCE [On-Policy], Proximal Policy Optimisation (PRO) [On-Policy], Trust Region Policy Optimisation (TRPO) [On-Policy]. \todo[color=green]{citations}

  \subsubsection{Actor-Teacher} 
  The main idea in this approach is the agent improves its policy from direct feedback from a \emph{teacher}, this teacher can be an external system that provides rewards or a complete manual system where the teacher is an ``expert human'' providing demonstrations (which goes into Imitation Learning, see \ref{sec:il}). Some common approaches are
  \begin{itemize}
    \item \textbf{Direct Action Supervision}: where the teacher suggest actions directly to the agent.
    \item \textbf{Criticising Actions}: where the actions are suggested by the teacher.
    \item \textbf{Shaped Rewards}: the teacher modifies the reward signal dynamically to encourage desirable behaviours.
    \item \textbf{Curriculum Learning}: the teacher controls the difficulty of tasks over time to ease the learning of the agent.
  \end{itemize}
  
  \subsubsection{Hybrid Approaches}
  Mixing the two together is also a viability, where the characteristics from each can be combined benefit from different guarantees each provides \cite{qu2020combiningmodelbasedmodelfreemethods}
  On top of the model dependence, the RL approach can be of two flavours:

  \subsection{On-Policy}
  This approach means that the agent updates its policy from data generated by the current policy. So an agent can only learn from the actions it purposefully took under its latest policy. Which can restrict improvements, but means that no outdated or stale data can influence the present time actions or learning.
  This leads to more stable learning in stochastic environments \todo[color=green]{citation?} while ensuring policy consistency.

  However, it also means that the agent becomes sample inefficient, as fresh iterations are needed to learn and sharpen the current policy. On top of that, the learning rate slows down because the agent can't use older experience (only the latest policy), slowing training \cite{andrychowicz2020onpolicyRL}.

  \subsection{Off-Policy}
  In this case, the agent can utilise past experiences, regardless of the policy generated by the training. This recycling of information previously known makes the learning more efficient \cite{uehara2022reviewoffpolicyevaluationreinforcement}, as sample efficiency is increased.
  Having past experiences also means that the agent will need fewer exploration steps and less iterations because of these priors. Making this system efficient in deterministic environments. \todo[color=green]{not quite sure why, add here}
  
  This however, brings instability in stochastic environments, and the the policy diverging due to incorporating outdated and possibly uninformative (at least at the present) data. Therefore, some sort of importance of older steps should be kept track of, and maybe even decayed like the discount factor from MDPs (\ref{sec:mdp}). \cite{maroti2019rbed} \todo{check this evidence not sure, and link to earlier}
  \\\\
  Both of these approaches can be combined with either \textbf{Model-Based} or \textbf{Model-Free} approaches. Because at the end of the day, making a good model is about picking the correct trade-offs that is relevant to the scenario of the problem we are attempting to solve.

% Imitation Learning
\section{Imitation Learning (IL)}\label{sec:il}
  While traditional Reinforcement learning models focus on interactions with the environment to optimise a reward signal to find an optimal policy, these can be sample inefficient, or very challenging in high-dimensional tasks. So, another approach we can take in teaching robots is Imitation Learning (IL) where an agent learns directly from expert demonstrations \cite{attia2018globaloverviewimitationlearning}, potentially bypassing the need of extensive exploration such an adjacent RL system would need. Imitation Learning, or as some literature refers: \emph{Learning from Demonstrations (LfD)} \cite{ARGALL2009469}; is a form of \emph{Supervised Learning} \cite{hastie2009overview,cunningham2008supervised}, where the model is presented with labelled expert demonstrations and a model is learnt from those.
  The roots of this idea comes from humans and animals \cite{bakker1996robot} learning from observations to copy movements from carers \emph{(supervisors)} for survival This approach is beneficial for systems where the exploration of the action space is dangerous, expensive or inefficient. We can broadly classify IL into two main categories, \textbf{Behavioural Cloning} and \textbf{Inverse Reinforcement Learning} each addressing the challenge of learning from demonstration slightly differently.


\subsection{Behavioural Cloning (BC)}\label{sec:bc}
 Behavioural Cloning (BC) approaches achieve this goal by  mimicking the action given to it by training a model that predicts the function between the state and the actions taken  \cite{pomerlau1991neco.1991.3.1.88, ross2011reductionimitationlearningstructured}. The expert behaviour will be recorded as a set of demonstrations (or \emph{trajectories} for a moving robot) $\tau^* = \lbrace(s_i^*, a_i^*)\rbrace_{i = 0}^N$ (for $N$ total demonstrations). Where the demonstrations are state-action pairs and \emph{$*$} meaning \emph{expert or optimal}.
 Then using supervised learning methods and treating the demonstrations as the training data we can predict a policy $\pi_\theta\left(a | s\right)$ along with a loss function $L \left( a_i^*, \pi_\theta\left(s_i^*\right) \right)$. The loss function is typically Mean-Squared Loss (MSE) for continuous and Cross-Entropy for discrete actions.

 However, one big downside of this method is that the demonstrations are heavily coupled to the model, meaning slight deviations from the given behaviour in the learnt optimal policy might lead the agent into unfamiliar states. Which then can lead to compounding errors known as covariate shift, and should be controlled for stability of learning \cite{mehta2024stablebccontrollingcovariateshift}. 
 \label{para:covariate-shift}. Another issue a method like this faces is the adaptation problem. The model does not understand but copies. So, without shaped-rewards (like RL) and dubious quality demonstrations, the agent might not be very capable.
 
\subsection{Inverse Reinforcement Learning (IRL)}
As before in \ref{sec:bc}, this method will be provided the expert demonstrations $\tau^*$. Although, this time the model will assume the expert is acting according to some unknown reward function $R\left(s, a\right)$. And differently to BC, IRL methods aim to estimate this reward function which can later be used to derive the optimal policy. Allowing RL approaches to apply to a broader set of problems.

This also allows policies to be agent agnostic (to the extend of simulation training applying to the real world, though with caveats) as the reward function is more transferrable compared to the optimal policy \cite{russell1998learning}.

And an interesting side product is the reward function that the model learns. It can be extracted for downstream applications. While being more tolerant and robust to faulty demonstrations due to errors being -ideally- outliers in the data and the predicted reward can correctly discourage such actions. 

However, demonstration quality sensitivity is still present. As well as the solution complexity disproportionately grows compared to the problem size \cite{ARORA2021103500}. Each iteration is dominated by the complexity of solving the underlying MDP with the currently learnt reward function, which is polynomial in size of its parameters. Which are exponential in the number of dimensions of the state vector. On top of this as the problem size increases the sample complexity must also increase, meaning the expert should cover more trajectories in the training set for sufficient coverage of the state space and the optimal prediction of the underlying reward. And finally, verifying that correct reward is predicted is hard as it is inherently variable what is learnt.


\section{Demonstration Quality and Abundance}
A unique challenge of learning from demonstrations is that the quality of the provided information should be good, in terms of achieving optimal solutions, so then an agent can also infer the optimal policy. But also, there should be enough demonstrations so that the agent can generalise to more scenarios.

\subsubsection{Few-Shot Learning}\label{sec:few-shot}
This paradigm allows models to generalise behaviours from a limited number of demonstrations. This is done to overcome the problem of having to provide large collections of manually curated or at least verified data \cite{fewshotsurvey}. A usual approach is to use generative models (see \ref{sec:gail}) to create more data from the distributions of the data available. Main reason is to mitigate issues such as covariate shifts in the sampled data. An extreme end of this paradigm is \emph{one-shot} learning (such as \cite{vitiello2023one}) where the learning must be done on a single demonstration.

\section{Imitation or Reinforcement? - Hybrid Approaches}

\subsection{Offline Reinforcement Learning}
This ideas extends the \emph{data-driven} paradigm of IL and composes it with traditional RL approaches. As opposed to Online Reinforcement Learning, which iteratively collects experience and interacts with a given environment which is then used to improve the policy for the next episodes of iteration. This can be impractical due to data collection being costly (e.g. in robotics and healthcare domains) and sometimes dangerous (like autonomous driving). 

So offline RL, relies on previously collected data instead of fresh environmental interaction \cite{levine2020offlinereinforcementlearningtutorial}. With help from advancements in deep learning, it has been made possible to create generalisable \emph{decision making engines} \cite{levine2020offlinereinforcementlearningtutorial} if sufficient prior data can be obtained.

\subsection{Fighting Distributional Shifts: DAgger}
Another interesting solution to the problem mentioned above \ref{para:covariate-shift} is the Dataset Aggregation (DAgger) \cite{ross2011reductionimitationlearningstructured}. This is used to fight distribution shifts in standard BC \ref{sec:bc}. A policy is trained on expert demonstrations by because of the shift, there may be non-encountered states during deployment. So, DAgger iteratively collects new data; this new data, in the form or state-action pairs gets added to the training and gradually improves robustness of a policy against novel situations. Although, the constant expert supervision, while generating new data makes it costly for real-world applications; or where a algorithmic reward system is not in place.

\subsection{Generative Adversarial Imitation Learning (GAIL)}\label{sec:gail}

GAIL \cite{ho2016generativeadversarialimitationlearning}  takes inspiration form Generative Adversarial Networks (GANs) \cite{goodfellow2014generativeadversarialnetworks} and can create model-free algorithms for IL in robots. GANs are deep learning models for generative tasks, they are mainly given a distribution of data and can generate synthetic data from following that distribution. For IL, they are useful for creating useful samples for self-supervision.
% Computer Vision 
\section{Computer Vision in Robotics}
On top of the various training and learning algorithms, another cornerstone of fully autonomous robot movement is computer vision. Through the integration of visual sensors (and possibly others to support and reinforce this perception) and advanced image processing techniques, robots can be taught to identify surroundings and make informed decisions. 

Although autonomisation is possible without vision, having a generalisable view of an environment or task allows the robots actions to also be generic executors. Take for example a factory robot assembling cars. It can be efficiently made automatic without any complicated models, and with just a simple algorithm. However this means that the environment (i.e the car parts, and maybe the work-area layout) must be presented in identical configurations for each episodic repetition of the task.
As the focus in robotics is shifting towards generic dynamically moving robots, adapting to their environments, vision becomes an inevitable sending medium.

\subsection{The Camera Model}
Camera models are essential for vision. As us humans perceive the world in an analogue manner through light, the robot must also be able to interpret its surroundings the same way. A \emph{camera} is a device that captures light in a scene and a \emph{camera model} is therefore defined to be the how that analogue information is mapped onto a 2D coordinates in a mathematical manner \cite{zhang2021cameramodels}. 

An essential process when using a camera is calibration. This is required so that we can normalise what the robot ``sees'' and using some pre-defined criteria (such as a known object or pattern) so that we can be assured the information the camera provides is within a specified degree of confidence. This uncertainty range should be as low as possible as tasks like localisation, mapping and object interactions in robotics usually require precise camera measurements.

While calibrating we need to have some idea of physical properties of the camera (\emph{intrinsic}) and information about the mappings of the scene (\emph{extrinsic}).

\subsubsection{Intrinsic Parameters}
\label{subsubsec:intrinsic}

These cover the internal characteristics of the camera, and how the captured three dimensional (3D) world data will be projected down onto the two dimensional (2D) image plane.
Some important parameters are:

\begin{itemize}
  \item \textbf{Focal Length ($f$):} The distance netween the camera lens and the image sensor. Determines field of view (FOV) and required for scaling the scene.
  \item \textbf{Principal Point (c):} Point of intersection for the optical axis and image plane. Usually at the centre of the image.
  \item \textbf{Skew (s):} Non-orthogonality factor of the sensor axes of the camera. (Often assumed to be zero.)\todo{cite or remove?}
  \item \textbf{Distortion Coefficients:} Some parameters for distortion correction. Important for camera model systems like cameras with fish-eye lenses \cite{king1989history}
\end{itemize}

\missingfigure{maybe a photo with all of these on it if that is possible}

\subsubsection{Extrinsic Parameters}
Extrinsic parameters represent the physical placement of the camera in the scenes. Such as the position and orientation of the camera in the scene. Using these values we can map  3D representation of the world the camera sees (which is in world coordinates) into the camera coordinate system.
\\

\subsection {The Pin-Hole Camera}
One of the most foundational and widely used models to describe this calibration is the pin-hole (or the  \emph{projective}) camera model. 

\subsubsection{Mathematics Behind Pin-Hole}
The light passes through a single point, called the camera centre, $C$, before it is projected onto the 2D image plane (giving the name pin-hole). 
% // TODO\missingfigure{include classic image, steal from https://www.oreilly.com/library/view/programming-computer-vision/9781449341916/ch04.html#:~:text=The%20pin-hole%20camera%20model%20(or%20sometimes%20projective%20camera%20model,a%20dark%20box%20or%20room.
% }
A 3D point $\textbf{X}$ is projected onto image point $\textbf{x}$ using the equation:
\[\lambda \textbf{x} = P\textbf{X}\]
where $P$ is the a 3x4 matrix called the camera (or \emph{projection}) matrix and $\textbf{X}$ is 1x4 and has four elements in homogenous coordinates, \(\textbf{X} = [x, y, z, w]\) and $\lambda$ is the inverse depth of the 3D point. Which can be needed if we want all coordinates to be homogenous with the last value ($w$) normalised to $1$.

The Camera Matrix, P can will all calibration values for the camera. so:
\[P = K \left[R \mid t\right] \]

R is the (3x3) rotational matrix describing the orientation of the camera, and t a (3x1) translational vector describing the position of C.
Also the intrinsic calibration matrix, K, will encode camera attributes discussed above.
\[
  K = 
  \begin{bmatrix}
    \alpha f & s & c_x \\
    0 & f & c_y \\
    0 & 0 & 1
  \end{bmatrix}
\]

Where the values are from \ref{subsubsec:intrinsic} and $\alpha$ is the aspect ratio used for non-square pixel elements (usually safe to assume $a=1$).
% Active Vision 
\section{Active Vision (AV)}
    Physical robots, by actively controlling their cameras \cite{Aloimonos1988}, as well as using attention and focus mechanisms; can identify the relevant parts of an environment. This, by introducing another level of decision making in the form of: \textbf{where} and \textbf{what} to perceive, enables the robot to move and act more intelligently within the space provided. This approach is mostly beneficial in applications such as: autonomous navigation, multi-robot control and human-robot interactions. \cite{breazeal2001hri}.
    \todo[color=green]{more citations for the examples?}
    
  \subsection{Principles of Active Vision}
  Traditional static vision can be enhanced by incorporating camera and viewpoint optimisations into the perception systems. Some principles include:
  \begin{itemize}
    \item \textbf{Gaze Control:} This is the fundamental idea of active vision. Move the camera and the viewpoint, (ideally) to regions of interest.
    \item \textbf{Next-Best-View Planning:} Strategically adjusting the camera to maximise the visual information the robot acquires. They often use \emph{attention mechanisms} which ensure they focus on the specific items or areas \cite{Burusa_2024} while ignoring the other cluttering information in the scene. These attention mechanisms can be \emph{spatial}, focusing on regions of interest; or \emph{temporal}: used with tracking moving objects.
    \item \textbf{Task-Driven Perception:} Optimising the visual input to the task at hand by using methods like \emph{next-best-pose} to ensure the robot stays focused on task-specific manipulations, such as keeping its gripped in the correct pose before reaching an object.
  \end{itemize}

  Some combinations of these fundamental ideas, fitting them to the use-case means a robot can acquire better observations of the space instead of relying on prior human knowledge to strategically pre-place the camera where it would most benefit the task.

  \subsection{Back To Markov}
  \subsubsection{Partially Observable Markov Decision
  Processes (POMDPs)}\label{sec:pomdp}
  Adding on to the definitions above in \ref{sec:mdp}, we now require \emph{Partially Observable Markov Decision Processes (POMDPs)} 
  which is an extension of MDPs where it is defined as a 7-tuple \cite{thrun2002probabilistic,placed2023surveyactivesimultaneouslocalization}: 
  \[\langle S, A, \mathcal{Z}, \xi_S, \xi_{\mathcal{Z}}, E, \gamma \rangle \]
  
  in addition to earlier, the state transition function \( \xi_S ~\colon~ S \times A \rightarrow \Pi\left(S\right)\) and $\Pi\left(S\right)$ is the space of probability density functions (pdf) over $S$. The observation space $\mathcal{Z}$, and the conditional likelihood of making any of those observations \(\xi_{\mathcal{Z}} ~\colon~ S \rightarrow \Pi\left(Z\right)\) where $\Pi\left(\mathcal{Z}\right)$ is the space of pdfs over $\mathcal{Z}$.
  Differently to the fully observable case, agents here cannot reliably know their own true state. So, they maintain an internal \emph{belief} system (historic), $b_t\left(s_t\right)$ which represents the posterior probability over states at time $t$, given the available data collected up until that time. \(b_t\left(s_t\right) \triangleeq  \mathbb{P}\left(s_t \mid z_{1 \colon t}, a_{1 \colon t - 1}\right)\) where the given is the history, $h$.
  \\\\
  Introducing this belief system we can now model active tasks which rely on exploring the space for the optimal states and selecting an action that decreases the uncertainty in the belief state, leading to actively informed decision-making.

  
  \subsection{Active Vision in Practice}
  There are a number of ways to implement active vision in practice. The two main temporal aspects are \emph{synchronous} and \emph{asynchronous} vision-action models \cite{divyaHandEyeCoordsination}. Synchronous models need to make real-time decisions on the next action based on the current observation information. This can be challenging in practice as the sensing mediums must be well coordinated and the observation may not contain informative information yet, causing issues with capabilities of a model\label{sec:asynch-synch}

  A more realistic model is the asynchronous one. It is more human-like in the sense that decisions in movement and viewpoint usually do not occur simultaneously, although quickly we usually adjust our views then act. So this creates a nice vision to action pipeline that can be more effective for agents learning active vision policies.
  

  \subsubsection{RL for Active Perception}
    Framing this problem as a decision-making one, so: ``Where should the agent face to maximise reward?'' allows us to come up with a policy which can learn where to look next, optimise the pose of the camera to minimise uncertainty in decisions and adapt to different environments \cite{rothbucher2011,zhangembodied}. Which can then be modelled as a POMDP system to then be solved as classical RL, giving us a way to control the gaze during a tasks execution.

  \subsubsection{Predictive Control}
  Instead of allowing random exploration of the viewpoint space we can also train models that predict the next best view using techniques such as Entropy-based viewpoint selection and Bayesian optimisations with the use of prior information such as the knowledge of the scene, object models' geometry and symmetries \cite{dhami2023prednbvpredictionguidednextbestview3d,breyer2022closedloopnextbestviewplanningtargetdriven}
  \\\\
  With these implementation strategies we can use active learning for tasks such as:
  \begin{itemize}
    \item \textbf{Dexterous manipulation robots} (i.e. grasping robots), where a eye-in-hand camera (maybe) coupled with depth sensing can refine its understanding of its surroundings and actively adjust views to avoid occlusions. 
    \item \textbf{Autonomous navigation,} mobile robots can use approaches like Active SLAM \cite{s23198097}, to map their environments and refine their understanding of it constantly. Some examples may be aerial drones continuously exploring and learning more about different regions \cite{drones6040085}.
    \item \textbf{Human-robot interactions,} social robots can be made to keep eye contact or learn to track human mannerisms from gestures and expressions to make them more realistic in interactions.
  \end{itemize}
    
  \subsection{Challenges}
  Despite all the discussed possibilities, Active Vision still remains fairly theoretical and unexplored compared to other areas of robotics. Although, the ideas have been floating around in computing circles for almost three decades, there are several challenges that need to be overcome for active vision to be viable for all its possible applications. These issues include:
  \begin{enumerate}
    \item \textbf{Computational Complexity}
    \item \textbf{Sensor, Hardware and Environmental Limitations}
    \item \textbf{Integrations with other Sensing Mediums
    }
  \end{enumerate}

  Although, these cannot be classified as \emph{solved} there are some approaches in tackling this active vision perspective of robotics which we will review in the next chapter.

\chapter{Related Work}
\section{Active Vision Resaerch}

As active vision within robotics is an emerging area of research, there isn't a consensus on a single approach to tackle this issue. On top of this aspect, by the inherent nature of differing tasks requiring differing approaches, a general solution may not be possible for every task. However, using techniques highlighted earlier and building on top of these we can come up with frameworks that are competent and most importantly highly competitive compared to current static camera systems.

Here we will highlight some significant contributions that are similar in scope to what we are aiming to achieve in this project.

\section{Constraints in Active Vision}
The main constraint is that for effective and optimal learning large number of demonstrations must be fed into an agent. However, the manual aspect of generating ``expert'' demonstrations is not always possible. This naturally leads the literature to be mainly concerned with approaches that deal with few-shot learning (see \ref{sec:few-shot}) and self-supervision.

A direct by-product of this constraint is incorporating prior knowledge into the learning. Such as the information about an object in an \emph{object-centred} problem, meaning the task revolves around manipulation of specific object or objects. This allows to compensate for lack of large set of demonstrations. These could include object poses \cite{huang2018generalisedTPlearning, hu2023modelpredictiveoptimisation} or meta-learning policies from pre-trained models on desirable datasets \cite{finn2017oneshotvisualimitationlearning, mandi2022}.


\section{Object Priors}
Following the constraints, an active perception learning robot is usually given information about the task and the objects that are important to that task, that way extracting the relevant policies gives us a baseline on the prominence of such a policy.

\subsubsection{\emph{Robot See Robot Do}}
This is explored in \emph{Kerr et al.}, where the work revolves around teaching a robot to interact with manipulable objects, such as chests, drawers, glasses and toys \cite{kerr2024robotrobotdoimitating}. At its core this is a grasping task integrated with one-shot learning, however, with the important prior that the object's 4D model -recovered by 4D Differentiable Part Models (4D-DPM). This way the we can observe what the robot can synthesise the correct manipulation points from its viewpoint and generalise this to all given viewpoints. Although, this isn't immediately an active-vision robot, this is a good starting point in understanding to design systems that are using their perception to make meta-decisions that influence their movement in their main policies. 

\section{Semi-Active Vision}
The simplest idea in teaching a robot to see in a human-like manner is to train it on data directly generated by human interactions. This also means the subject specific information like priors can be inferred by the policy, instead of being explicitly provided by the researchers.

\emph{Chuang et al.} explore the idea of teaching a robot the task policy joint with the policy of moving a camera fitted arm \cite{chuang2024activevisionneedexploring}. Their setup includes a Virtual Reality (VR) Headset, which allows the demonstrator to move the AV camera on the robotic arm. This allows them to teach policies for tasks where the subject may be blocked by small static cameras around the scene, or the camera fitted in a eye-in-hand configuration being occluded or out-of-bounds due to orientation of the gripper with respect to the task. 

Although, their research is promising they acknowledge that active vision also brings some issues that need solving. Some important mentions are: operational and architectural complexity, though, they used similar architectures for all their different camera configurations, they note that a bespoke system for AV might benefit such a system. 

The expanded action space also poses an issue as the state space complexity explodes, and hint that decoupling the vision and the control system might help with the issue. Finally, they touch on distribution shifts being a big issue, this mainly stems from the earlier implicit subject information. As the model is not aware of targets it will learn what it can infer, so if the demonstrations do not contain enough variations in poses, generalising the  locations and other characteristics for an object become tougher. A possible strategy in solving that particular issue is generating augmented samples, to fight this covariate shift.

\section{Self-Supervision and Data Augmentation}
Another widely used strategy to mitigate providing vast amounts of manually crafted and hand labelled data is to subscribe to the idea of self-supervision. These are used to counteract the instability of reinforcement learning and the inefficiency of random exploration.\todo{link to earlier definition from cv/av}

\subsubsection{\emph{Making Imitation Learning Easy with Self-Supervision: MILES}}
Incorporating this into robot learning tasks usually takes the form of generating augmented movement trajectories. Which are simulated and sampled from the limited number of human trajectories (usually accepted to be expert behaviour) given to the agent. \emph{Papagiannis et al.} using the \emph{MILES} framework, simplifies the process by removing the human intervention aspects from highly repeatable supervision parts of the learning \cite{papagiannis2024milesmakingimitationlearning}. 

This mimics exploration-based RL learning systems where given a list of trajectories from the expert behaviour,the agent will move to a pose near the demonstration and attempt to move itself back to original trajectory. In the process creating an augmented path that eventually joins with the ``correct'' one. As it collects sensor data along the path (per waypoint) it therefore, creates augmented data that can now be used to aid its training. So, it is self-supervised in the way it collects data. As no interaction is needed to correct the agent back to starting position or other environmental resets (assuming the learning tasks doesn't manipulate the scene in a non-recoverable way),

This is achieved by training a separate policy for each task as a \emph{LSTM} (long-short term memory) network, based on Behavioural Cloning, which is a type of Recurrent Neural Network (RNN) which handles sequential data \cite{medsker2001recurrent}.This allows the policy to learn the gradually changing trajectory while remembering the steps taken in the past. On top of this no object pose priors are given to the network, meaning the networks learnt policies should be applicable to different object poses.

However, another important part of \emph{MILES} is that force sensors are also included in the decision making policies. Which is not a make-or-break addition, as they conclude, the force modality sometimes helps the system achieve better accuracy when coupled with vision, and sometimes not. While just force -without vision, in a somewhat expected manner- performs quite badly in any of the evaluation tasks they have chosen. 



% \cite{natarajan2021graspsynthesisnovelobjects}
\todo[color=green]{this (see comment in file) also kinf of goes under learing using heuristics but not really trajectory maybe move to next section and so some explanation if needed}

\section{Attention and Information Gain (IG)}
This is pivotal part of object-centric tasks. Because, if an agent knows what to focus on, then we can teach it a policy to learn to focus on specific subjects.

\subsubsection{\emph{Observe Then Act}}
Taking a third-person-view look to the classical grasping task, \emph{Wang et al.} aim to optimise this viewpoint based on the task goal \cite{wang2024observeactasynchronousactive}. They take a asynchronous viewpoint control approach (see \ref{sec:asynch-synch}); this separation of the camera systems and the motor actions over time. Leading to less need for coordination between the two section and instead, the model focuses more on task-specific movements and distribute this coordination throughout the task. 
They, again, follow a few-shot learning approach. The model comprises of two separate agents, a next-best-view (NBV) agent for optimal viewpoints and next-best-pose (NBP) for determining the gripper's action based on the previous agent's output. Active perceptions is achieved by alternating between sensor and motor action interfaces in each episode; which then leads to the learning of the tasks. These tasks are usually of sequential nature as the asynchronous approach works best with such systems, as they discuss.
\missingfigure{the neuron figure form the paper}

They follow a POMDP (\ref{sec:pomdp}) formulation to model the problem:
\[
  \langle \mathcal{O}, \mathcal{A}^c, P, \mathcal{A}^g, \mathcal{O}', P', R, \gamma \rangle
\]

where $\mathcal{O}$ and $\mathcal{O}'$ represent the observation spaces at times $t$ and $t'$ the NBV policy $\pi_v$, determines the camera viewpoint action $a_t^c \in \mathcal{A}^c$, given an observation $o_t \in \mathcal{O}$, and obtains a new observation $o_{t'} \in \mathcal{O}'$ through the transition probability $p\left(o_{t'} \mid o_t, a_t^c\right) \in P$. Finally, the NBP policy, $\pi_g$, then determines the gripper action $a^g_{t'} \in \mathcal{A}^g$ based on the new observation $o_{t'}$. then using the transition probability $p'\left(o_{t+1} \mid o_t, a^c_t\right) \in P'$ the next scene observation can be obtained and reward $r_{t+1} \in R$ will be provided. So, the model will be learning these policies: $\pi^*_v$ and $\pi^*_g$ then try to jointly maximise the return for the joint reward for this collective task: 
\[
  \pi_v^*, \pi_g^* = 
  arg~\underset{\pi_v, \pi_g}{max} 
  ~\mathbb{E}
  \left[
    \sum_{t=0}^{\inf}{\gamma^t R(o_t, a^c_t, a^g_t)}
  \right]
\]
\todo[color=green]{more information about the 3D voxels for environment normalisation?? this is more about fighting the caveats of few-shot learning not really fitting here, but might include if it comes in useful later}

Another important contribution here is the use of augmented trajectories again. Similar to \cite{papagiannis2024milesmakingimitationlearning}, demonstration trajectories are augmented in a \emph{viewpoint-aware} manner to expand the learning set by sampling the observations and discovering key frames. It does this while the viewpoint is allowed to shift, meaning the samples getting generated between time frame $T_t$ and $T-{t+1}$ (where $T$ is a trajectory) can have different views, which aids the camera policy in progressive movements. This is done for keyframes for both camera movement and gripper pose, for the two policies.

\subsubsection{Closed-Loop Next-Best-View Planning for Target-Driven Grasping}
Similar in idea, \emph{Breyer et al.} explore yet another grasping task (although first-person-view this time) this time. They augment classic grasp synthesis tasks that are mostly reliant on deep learning approaches, which suffer depending on the visual information it has available. Similar to \emph{Wang et al.}, they interact with the environment in an asynchronous manner within a fixed rate. They determine the best candidates for grasping then compute the next-best-view with its associated information gain. 
The information gain metric, allows the system to explore viewpoints in the neighbourhood of the current view and estimate what might be ideal for the grasping, bridging the rewards of the two policies. This again is made possible through the assumption that a task is object dependent hence the model receives bounding boxes and the geometry of the subject of the task.

However, the interesting takeaway from their system is an early stopping mechanism. Due to the nature of exploration multiple views can be scanned and many relevant ones can be identified by the system. Without stops, this algorithms may over explore and end up quite inefficient. These stopping conditions include: timeouts (and due to time-framing, limited number of policy updates), minimum thresholding on the information gain so that no unnecessary exploration is done where the estimated IG might be lower; and finally, convergence, when the Volumetric Grasp Network (VGN) \cite{breyer2021volumetricgraspingnetworkrealtime} outputs converge the exploration will stop. 

Therefore, the importance of their research is highlighted in the robustness of their system and the balance they managed to strike between exploration and exploitation. 

\todo[color=green]{talk about attention and active perception}

\subsection{Outline}
In summary, work in the field of active vision usually follows an attention metric which is guided by a reward system, usually depending on some priors. The need for priors, although, can be mitigated by providing more demonstrations, this complicates the test setups and operations. Therefore, a common approach is to create augmented data and broaden the horizon of the agent through the exploration of that synthetic space.

Combining these ideas shows us that active vision policy adjustments are not only possible during policy execution but also very promising way to advance the field of robot learning.



%//CHECK interesting paragraph not sure about it tho
% Another interesting point is the relevance of active vision for a task which isn't object-centred. 
% Does the use of active vision make sense to a robot where there isn't a defined subject? Or should the agent explore and find objects of importance?
% \todo{not sure about this paragraph, does this make sense maybe for the final report}

% \chapter{Project Plan}
  In this chapter, I will cover:
  \begin{enumerate}
    \item What the project entails and how the work will be divided until the end of the project's timeline.
    \item The ultimate evaluation plan for the entire project once the data is collected.
  \end{enumerate}

\section{Evaluation Plan}
As the project is an evaluation of the proposed framework upon adding active vision elements. It will have to be done on a pre-set collection of robot tasks. 
Firstly, these tasks will be attempted with conventional approaches as control. This will show us whether the improvements we suggest, in the form of extra active vision learning, has an affect on the success rate of the agents learning to do these tasks. % REMOVE -  plan was for interim report

\chapter{Early Work}
In this section I will cover:
  \begin{itemize}
    \item Initial research and practical explorations in 2D 
    \item Laying the groundwork for 3D environments
    \item Problems when transitioning into 3D environments
    \item Early experimentation on limitations of robot vision
  \end{itemize}

% Early work 2d
\section{Starting off with Two Dimensions}\todo[color=red]{shorten section, make pictuers smaler, this is too long I think}
Early steps I was advised to work on was, getting an idea of \emph{policy learning} in two dimensions (2D) to get an intuition and follow it on with expanding it to three dimensions (3D) from there.

\subsection{Structuring the Problem}\label{subsec:ew-2d-problem}
Following those ideas I wanted to create a simple task to train a policy to teach a block to move towards the other block, on simple 2D rendered canvas. Creating a path finding task, which essentially is a two dimensional reaching task.

\subsubsection{Problem Design} 
I started with pygame \cite{pygame}, a python library that can be used to create 2D canvas with artifacts in it. But before rendering there are some things we have to define. I wanted to create a small\emph{ish} canvas to simulate the game in, with my player and target. I chose to make a $800 \times 600$ canvas and defined my blocks to be $20 \times 20$. Skipping the small details, the simple game was rendered as Figure \ref{fig:initial-canvas}, where the agent (blue) attempts to get to the target (red).

\begin{figure}[h]
  \centering
  \includegraphics[width=0.2\textwidth]{assets/early-work/initial-canvas.png}
  \caption{Initial canvas $800 \times 600$ and $20 \times 20$ blocks}\label{fig:initial-canvas}
\end{figure}

As this was a simple game, for movement I defined a speed, which was initially $5$ pixels per press and mapped the movements to the arrow keys, and wrote a simple check that when the agent makes it to the target the game is complete.

\subsection{Solving the Problem: Learning}
To make a policy that would play the game, there was a few things I needed; firstly, the game needed a model to solve, I started with a classificiation approach, to classify the correct keys to press to move to the target.


\subsubsection{Classification}
Therefore, my model was fairly simple; the state, $S$ which is a \emph{4-tuple} $\langle float,~float,~float,~float \rangle$ that holds -in order- $x$-coordinate of agent, $y$-coordinate of agent, $x$-coordinate of target and the $y$-coordinate of target. And due to the keypress nature of the system, the Action, $\mathcal{A}$ was also a \emph{4-tuple} $\langle boolean,~boolean,~boolean,~boolean \rangle$ which was true or false to indicate if an arrow key was pressed, the positions in order are: \emph{Up, Down, Left} and \emph{Right}.

So, knowing the space, I created an expert policy, \(\pi_{demo}\), which gives the key combinations that must be pressed by interpolating where the target is with respect to the agent. Using their coordinates this was pretty simple. So, I randomly generated some starting positions and made the agent play the game many times and got snapshots of the states and stored them in a labelled structure for training a model later. \(data: list\left[State,~Action\right]\) where:\(
  State:~tuple\left[float,~float,~float,~flaot\right], 
  ~Action:~tuple\left[bool,~bool,~bool,~bool\right]
\)as before. Therefore, combining the two, I trained a linear neural network with 3 layers in PyTorch \cite{pytorch} predict the key presses (the action) given the coordinates of the blocks (state), which included the target's coordinates as well.

Without getting into implementation details, this worked pretty well. Every frame the state would be fed into the model and the model gives out the movement for that frame. Although, the movement was quite jagged because the classification was predicting key presses and the discrete actions were inherently not smooth; because a small change in a direction could prompt the model to change the  direction suddenly. A solution to this would be to get the direction change instead of a discrete movement direction. So, I pivoted to a regression model.

\subsubsection{Regression}
Similarly, as the scope changing I had to change my model slightly. State remained the same, but the action is now defined to have the type: \(Action: tuple[float, float]\), reflecting the change in $x$ and the change in $y$ (\(\delta x , \delta y\)).
Following a similar approach I now had a regression model, trained on the same data (also augmented to have coordinate labels)


\subsection{Extending the Problem: Vision}
To step the investigation in the direction of vision, now I wanted to do the same learning, however, without access to the coordinates of the blocks. Adding onto the regression interface, both because smooth movement is preferred and this reflects real life robot control in the form of continuous values to adjust motor velocities.

I started by simulating the training data, making the expert play the game and instead of saving the state as a \emph{4-tuple} of coordinates it was taking a screenshot of the canvas. As the canvas is an RGB image, the type of state now became: $State: list[Image, Action]$ where the image is a vector of size $3 \times 800 \times 600$, for the 3 RGB channels and the size of the canvas.

\subsubsection{Receptive Area}
Next step was feature extraction, my model had to \emph{understand} what it was seeing. I experimented with CNNs to be able to extract information from the scene and infer a relationship between state and action. The first issue I faced was with receptive field size. Mainly using a kernel size of $5$ and stride of $s$, so that by image was efficiently getting downscaled down the pipeline to learn information at smaller scales. 

However, due to the block size being too small compared to the total size of the canvas, CNNs that weren't deep enough were not extracting useful features from my scene. To hone the issue, I decided to increase the size of the blocks to $50 \times 50$ and after some experimentation and tuning parameters I landed on this model (Figure \ref{fig:cnn-5050}) that performed as well as, if not better, than the coordinate based regression model.

\subsubsection{Sampling in Phases}
Another issue I ran into was the this model was performing generally well in getting very close to the target but refusing to take the final small step to collide and successfully complete the task. I realised this was because of the random sample collection being generally far away from the target, by the nature of uniform random sampling and there being vastly more space away from the target. So, to fix this, rather than changing the model architecture I changed the way expert data was sampled.

In training, I added a parameter to the training dataset, and to the creation of the training data. I made sure I could control the places data was being generated and labelling them where they were getting generated with respect to the the target. Called these phases of generation and made sure i could control the ratio of training samples and which phase they were coming from during the training. \todo[color=blue]{can be shortened/removed}

See Figure \ref{fig:phase-regions}, for an explanation of this idea. Giving more weighting to a closer region to the target, allowed my model to not suffer from the distribution shifts in my data, allowed for the final push in the right direction for the agent to complete the task successfully.


\subsubsection{Returning to Small Blocks}
I also then got a model that was only one more convolutional layer deeper to work on the earlier smaller block size, however, the training time increase between the two were quite large for the same data sizes. Around one hour versus, four hours and change, for about 1000 episodes. Therefore, I continued this experimentation with the larger block size.\todo[color=green]{data? or not importtant enough?}

\subsection{Next Hurdle: Obstacles}
The rational next step was to add obstacles to simulate some ablations, which would mimic occlusions in 3D. Although, not visual barriers this was another hurdle the agent had to overcome. The expert data moved the agent using a path-finding algorithm, inspired by the $A^*$ algorithm \cite{cui2011based}, explores the cells around it with the heuristic of keeping the Euclidian distance of it and the target low. \todo[color=green]{pseudocode of this? appendix}

\subsubsection{Obstacle Generation}
I took a simple procedural checkered approach where the canvas was divided into a grid which has cells the size of the agent. Then a cell will either be an obstacle or not with some clever mechanics around ensuring at least a single path from the agent to the target and some uncertainty. Check Figure \ref{fig:obs-gen} for some examples.

\subsubsection{Getting Stuck}
I faced some issues with the model learning to avoid obstacles, the above policy for the no-obstacle case worked pretty well most of the time, however some select cases, such as there being an obstacle right between the agent and the target. 

After some experimentation, I thought this might be because of the way I modelled the prediction system. Every frame expects a new movement to be generated, per the state given. This inherently has no history about where the agent was moving from. Similarly, the training data is also not ordered in any way, and it is frame corresponds to movement. Therefore, my model was essentially learning the underlying policy of the Euclidean distance heuristic used in the path finding and not extracting useful information about trajectories. Some solutions might be to do planning, request demonstrations when stuck or do a RL approach. However, due to this being preliminary work about vision, and end-to-end learning, I stayed away from classical planning and exploration based approaches in favour or learning the scene through understanding the expert demonstration the agent is given

\subsubsection{Future of 2D: Sequence Models}
My next approach was to play around with a model that would take in multiple frames and output either one or multiple moves in order to have this idea of a history.I experimented briefly with a \emph{sequence-to-movement} model primitively replicating a RNN, where it kept a 10 frame buffer along with the movement to predict the next movement. This idea seemed promising, however, my initial approaches were not fruitful, and the model suffered from seeing to many padded empty episodes

% Early work 3d
\section{Transitioning to Three Dimensions}
The move to 3D was slow, the jump from \emph{pygame} to a full blown physics simulation was a big leap and it came with its problems.

\subsection{CoppeliaSim and RLBench}

CoppeliaSim is a mostly open source simulation program that provides full access to its features through an educational license and RLBench is an in-house (Imperial) tool adapted to plug into CoppeliaSim through its python API interface (PyRep \cite{}) \todo{cite pyrep}. 

There are many tutorials online to help learn how CoppeliaSim operates and designing some scenes \todo[color=green]{maybe cite some stuff here, not necessary}. However, RLBench is a harder beast to tame. Although the system is very smart and eliminates a lot of the manual work needed to be done before starting experimentation, it was now slightly out of date and I had many issues setting it up on my system.

\subsection{Environment Issues}\todo{entire section needs a reread and structuring}
One of the first large-scale issues I'd face in this project came here, quite early in its lifespan. I would soon learn robotics development is mostly done on Linux based machines and the journey to getting everything working would be quite long.

\subsubsection{Windows Setup}
I do most development work on Windows and WSL (Windows Subsystem for Linux) \cite{} and expecting this project to be fairly memory and graphic power hungry I though I'd setup everything on the Windows side. This caused a few issues with PyRep, this was becuase PyRep firstly expected to be running on Linux, and RLBench was having issues working.

\subsubsection{WSL Setup}
I thought this wouldn't be a problem, as WSL version 2 has been pretty good with GUI applications running on Linux and thought I could run Coppelia on there and still access any displays I might need. The translation layer between the operating systems might cause some slowing but there shouldn't be any major issues issue. Upon configuring everything as the installation guide suggested in RLBench and PyRep repositories. I had few issues linking object files downloaded with CoppeliaSim into the PyRep layer had some issues. After spending a lot of time researching issues, and coming across some online threads with similar issues (in different applications, so nothing was immediately applicable) I decided WSL must have been the problem and decided to move onto the next logical step.
 
\subsubsection{Linux Virtual Machine}
I had previously used Ubuntu a lot and even my WSL instance was running Ubuntu. As RLBench suggested Debian based systems, assuming they also developed it in Ubuntu, I created a Virtual Machine running it. After the entire setup process, I was finally able to get the instance running and finally managed to run one of the examples that gets downloaded when installing RLBench.

However, I got hit by another issues fairly quickly, the rendering of Virtual Box was definitely going through some sort of translation through Windows and then the GPU (not even sure, maybe it was all CPU rendered) however, everything was extremely sluggish, and running the non-primitive examples even crashed my virtual machine instances a few times. At this point I realised I had to settle for the real deal and got to partitioning by storage drive.

\subsubsection{Dual Booting}
I had some issues partitioning the drive Windows was already installed, as it had making that its home and spread all around the disk. I currently had no ways of backing up my data and erasing my disk to completely wipe then partitioning before installing windows again. So, I ripped a spare drive I had in my old laptop and booted Linux form there. After running through the same setup steps, I was finally properly running RLBench. With one caveat, CoppeliaSim constantly complained that I was missing some video encoding binaries, though seemed to work perfectly fine; even when I fixed the binaries it would work but every once in a while pop up saying I was missing them. There were other small annoyances like this throughout the entirety of the experimentation but at least now it was working.

\subsubsection{Campus Machines}
I knew that a solution to all of my problems might have been using a machine at the labs. I thought an issue with that may have been that RLBench requests a lot of binary linking and editing which might have required and the troubleshooting wouldn't be as easy as doing it on my machine. Which could be resolved if I requested a machine to use, but I though my personal machine had the appropriate powerful hardware for a machine learning project and wanted to use that. So jumping through all these steps tool about an entire week, but was well worth it at the end.


\subsection{Usability Issues}
One of the major instability issues I had was because RLBench is about 5 years old now and CoppeliaSim has moved on in some parts, and I was not able to get exactly the same version of the simulator they had as well as exactly the same version of the libraries used when developing RLBench.

\subsubsection{RLBench Codebase}
As I started using RLBench other issues started popping up, PyRep was randomly failing calls to CoppeliaSim due to errors raised in RLBench due to unexpected types and hitting exceptions such as \emph{``Should not be here''} which was especially frustrating. However, the solution was simple. Entirety of the RLBench source code is accessible, so instead of downloading it as a package I forked the repo and started fixing any issues as they started arising. Once trivial issues were getting fixed I also started using CoppeliaSim to make mockup tasks (to be outlined in the future \todo{linking!}) and create networks that would use the simulation to train.

\todo{mention RLBench and accompanying API PyRep are written for Coppelia v4.1.0, which is now almost 5 years old (https://manual.coppeliarobotics.com/en/versionInfo.htm), updating coppelia had issues with PyRep not working}

Around the time I started training some primitive models and learning more about these tools, I started hitting weirder errors, like the simulator constantly crashing, getting unexpected observation results (cameras such as overhead cameras  missing their input) and other weird issues. Spending even more time combing though some of the \emph{enourmous} codebase I would sometimes find issues to fix them, but it would routinely break some of the package binaries I had linked and just stop working with no way to reason or figure out how to fix it. At this point I decided to try some other tools.

\todo{add picture of the missing libraries thingy}

\subsubsection{PyBullet}\todo{add pybullet reference}
The first option was PyBullet, an open-source Python module for simulation, it wraps the C API of Bullet and provides a completely customisable simulation experience. Though, the GUI experience is lacking and almost everything is controller through this C API. 
Following some rough tutorials and combing through the manual for the module, I was able to put together fairly simple graphics, and shapes to start creating some tasks. Although, I came to a halting realisation soon after. Which was that without a complete robot learning suite at my disposal I would need to figure out a lot of the systems from scratch. Such as camera placements and movement systems, seamless task creation and linking, environment management and so on.


\subsection{Toolset Dilemma}\todo{needs a total rewrite this is more of a plan/rough draft}

Faced a lot of issues with RlBench, however, the time I spent on it was too long and learnt how to fix any issues when they came up. It also provided niceties like requesting demos and wiring environment and tasks fairly seamlessly.

On the other hand, PyBullet was customisable and I think overall ran better, coppeliasim (and especially my installation had a few issues which i couldnt seem to fix) but a lot of the ground work done by rlbench I would need to redo. 

So the main dilemma was, should I be wasting time making a comprehensive suite to fit my needs early on in the project and then cancontinue with the premise of the task, or stick with rlbench and solve issues as they arose. Once I fixed a lot of the OS dependent errors, newer versions of coppelia was for some reason not very happy with newer Ubuntu versions, I decided to stick with rlbench, and hoping any issue arising from this point on shouldn't be a system breaking one, and I was familiar enough with the rlbench codebase to at least attempt to fix anything at this point. So I went back to square one and got the work on rlbench.



I didn't do any meaningful work in terms of experimenting with learning or robots, but spent majority of my time setting up RLBench (quite tricky OS requirements and problems) and playing around to get myself familiar with the software. \todo{following to be deleted}



\section{First Steps in 3D: Reaching Task}
Landing back on RLBench and solving most of the issues, first steps were to create some 3d environments and tasks to test capabilities of simple policies on tasks that progressively get more difficult. So, the very first step was to directly translate the 2D task I played with earlier in \ref{subsec:ew-2d-problem}. This mean that I could create a simple reaching task, which not will happen in a 3D environment, with realistic physics simulation

\subsection{Creating the Task}
As discussed before, RLBench provides an intuitive method for quickly creating and validating tasks. \todo{link the tutoirals and the videeos maybe?}. Following the tutorials provided by the creator of RLBench and other resources online on CoppeliaSim \todo{link some copsim manuals or information here.} I could now create tasks for my use case.

Tasks are created using the \emph{`task builder'} CLI, which is a user facing interface that talks to the PyRep API which in turn controls CoppeliaSim. This allows the user to use GUI elements within the simulator to create objects, boundaries as well as edit their properties while observing what they look like in different camera view layers. Then to create scene wirings; the automatically created \verb|Task| class is used, where \emph{initialisation}, \emph{episode step} and other task related systems can be created for the scene object to then use to create demonstrations and train robot policies. See Figure \ref{fig:task-builder}.

\begin{figure}[htbp]
  \begin{subfigure}{0.48\linewidth}
    \centering
    \includegraphics[width=\linewidth]{assets/early-work/task-builder-cli-1.png}      \caption{Task builder CLI: creating a new task}
  \end{subfigure}%
  \hfill
  \begin{subfigure}{0.48\textwidth}
    \centering
    \includegraphics[width=\linewidth]{assets/early-work/task-builder-scene.png}
    \caption{Simulator: Graphical view of the new task}
  \end{subfigure}%
  
  \vspace{0.5cm}
  
  \begin{subfigure}{0.48\linewidth}
    \centering
    \includegraphics[width=\linewidth]{assets/early-work/task-builder-cli-2.png}
    \caption{Task builder CLI: running the simulator on new task}
  \end{subfigure}
  \hfill
  \begin{subfigure}{0.48\textwidth}
    \centering
    \includegraphics[width=\linewidth]{assets/early-work/task-builder-scene-hierarchy.png}
    \caption{Simulator: Scene Hierarchy of the objects in the new task}
  \end{subfigure}
  \caption{Creating a new task with `task builder'}\label{fig:task-builder}
\end{figure}\


The task I created is simple, see Figure \ref{fig:reach-no-obs}. I decided to use the \emph{Panda arm}, as that seemed to be standard and more importantly immediately supported by RLBench. It has 7 Degrees of Freedom (DoF) and a gripper -so the action space is a 8 dimensional vector. Then a red spherical target, which is not tangible, in the simulation it will be visually rendered but the arm will not collide or interact physically with it. Finally, added a proximity sensor to the target (which is not visually rendered) so the task can be immediately classified as completed by the system.



% //NOTE: relative paths for images, maybe add submodule once project is finished
\begin{figure}[htbp]
  \begin{subfigure}{0.48\linewidth}
    \centering
    \includegraphics[scale=0.4]{../fyp/assets/task-pics/reach-no-obs/random-front.png}      
    \caption{Front View}
  \end{subfigure}%
  \hfill
  \begin{subfigure}{0.48\linewidth}
    \centering
    \includegraphics[scale=0.4]{../fyp/assets/task-pics/reach-no-obs/random-top.png}
    \caption{Top View}
  \end{subfigure}
  \vspace{0.5em}
  \begin{subfigure}{1\linewidth}
    \centering
    \includegraphics[scale=0.5]{assets/early-work/random-scene-hierarchy.png}
    \caption{Scene Hierarchy}\label{fig:reach-no-obs-hierarchy}
  \end{subfigure}%
  \caption{Reaching Task with no obstacles}\label{fig:reach-no-obs}
\end{figure}

\subsection{Creating a Policy}

The next step was to create a policy network. As I am mainly working on imitation learning from demonstrations provided by the system, my network needs to be able to ingest these demonstrations and generate actions in the action space of the robot. \par

\fbox{  
  \begin{minipage}{\linewidth}
    \textbf{Aside on RlBench Demonstrations} \par
    Demonstrations are provided encapsulated in a \emph{Demo} class and contain a series of observations (belonging to a class \emph{Observation}). These include state information of the robot and the environment, containing information like the observations of the set cameras and sensors; which need to be configured when the environment is started. This modular approach means I can collect demonstrations, then selectively choose what to use, like ignore a camera or a sensor. \par 
    Demonstrations are created following waypoints. PyRep expects  \emph{Dummy} objects within the scene called ``waypointX'' where sequence number of the waypoint starting from $0$. Then he demonstration engine calculates trajectories to these waypoints in sequence. For example, in Figure \ref{fig:reach-no-obs-hierarchy}, we have `waypoint0' within the target, for which a trajectory will be calculated from the tip of the robot gripper.
  \end{minipage}
}

\noindent By default the demonstrations -which I will refer to as ``demos'' from now on- and specifically the cameras pre-placed in the scene, produce images of resolution $128 \times 128$ pixels. I binned this down to $64 \times 64$. This comes in handy as I can keep the processing power low and can adapt the policy for higher resolution cameras by scaling it up later on. Therefore, I created a simple network following this architecture outlined in Figure \ref{fig:policy-arch}. The full code can be found in \todo{add the code of the network to appendix}.

\begin{figure}[h]
  \centering
  \includegraphics[width=0.8\textwidth]{assets/early-work/cnn-encoder-policy-head.png}
  \caption{Simple Policy Network Architecture}\label{fig:policy-arch}
\end{figure}\todo{ius the quality bad? reuload or reexport the xmml is in assets}


\subsubsection{Tuning the Policy}\todo{can remove not sure??s}

\subsubsection{Policy Improvements}
\todo{maybe do some lr scheduling and data processing for the images for generalisation  and mention the shuffling etc of the data and that it doesn't change much}



\subsubsection{Testing Visual Limits}\todo[color=red]{not sure about these to be honest, maybe remove these??}


conditions are not always ideal, so wanted to do some quick checks to get concrete results on how the agent behaves when the target is placed on the fringes of its field of vision. So created the three tasks shown in Figure \ref{fig:no-obs-3-views}.

\begin{figure}[htbp]
  \begin{subfigure}{0.3\linewidth}
    \centering
    \includegraphics[scale=0.4]{../fyp/assets/demo-trials-no_obs/tasks/static-tasks-camera/initial-obs-side_l.png}      
    \caption{Left Side}
  \end{subfigure}
  \hfill
  \begin{subfigure}{0.3\textwidth}
    \centering
    \includegraphics[scale=0.4]{../fyp/assets/demo-trials-no_obs/tasks/static-tasks-camera/initial-obs-central.png}
    \caption{Central}
  \end{subfigure}
  \hfill
  \begin{subfigure}{0.3\linewidth}
    \centering
    \includegraphics[scale=0.4]{../fyp/assets/demo-trials-no_obs/tasks/static-tasks-camera/initial-obs-side_r.png}
    \caption{Right Side}
  \end{subfigure}%
  \caption{Three variations of the reach task with the obstacles sometimes out of view}\label{fig:no-obs-3-views}
\end{figure}

\subsection{Generalisability}
I had some static versions of this task where the target sphere was always in the same position, for which a single demonstration with long enough training is enough to learn the task. This is because without variation, Behavioural Cloning can be easily achieved by overfitting to the data. 

Therefore I created a dynamic version of the task, where the target is not randomly placed within view (not necessarily always fully within view but the centre will be, making sure the wrist camera can see it). This is achieved by using a \verb|SpawnBoundary| and randomly sampling the location of the target withing this for every new variation of the task, or for every new episode. Which can be seen in Figure \ref{fig:reach-no-obs}, the white bow being the boundary, which is not rendered in the simulation. This guarantees variety in demonstrations as well as helps us create a generalisable policy.

Running some tests with a tuned \todo{talk about tuning maybe?} policy, it was enough to see that, with the wrist camera having unobscured access to the target, it would easily learn to reach for it.


\subsection{Camera Limitations}\todo{reread and make sure it is a good starter segue to obs}

I was started experimenting limiting the learning to only the wrist mounted camera, which works well for this specific unobscured task, however, another interesting area of investigation is finding what kind of tasks benefit from different viewpoints, for example shoulder mounted or other static scene-observing cameras. 

Or more interestingly other sensors, such as depth or force sensors on robot hands or grippers. This relates back to the idea of human perception. To learn new tasks, we use all kinds of feedback from the environment that is reactive on our actions. Therefore, it is important to study shortcomings of individual sensors with respect to others to understand how policies with limited access to sensors can be made to overcome such challenges without necessarily adding more `observability' to a workspace.

\subsubsection{Multi Camera Mechanisms}
\todo{talk about `CamType' briefly}
\subsubsection{Policy Changes}
\todo{talk about the policy using differnt camera inputs to blend and make informed choices? maybe some plots here showing if it is better or not?}


\section{Reaching with an Obstacle}
So, to carry this investigation to the next level I introduced an obstacle placing mechanism as well as randomly placing the target behind the obstacle. This is to ensure that the agent doesn't learn where the target can be behind a wall from where the obstacle is. See Figure \ref{fig:reach-obs-random} for how this task looks and the check \todo{add appendix link} for the backend wiring of these tasks.

There are two versions of this task, I thought it might be interesting to randomise the object firstly dependently then independently on the obstacle. The `dependent' randomisation called \verb|ReachObs_Random| samples the obstacle, which in turn controls the spawn boundary of where the target can spawn in, meaning the target will always appear behind the obstacle albeit, edges of it can sometimes stick out. Conversely, the `independently' random version, called \verb|ReachObs_IndRandom|, this keeps the target spawn boundary fixed, meaning the target can be anywhere in the visible workspace, but it is not necessarily always covered by the obstacle. I can see that this potentially can be useful to keep the dataset a bit more diverse, and allow the wrist camera initially observe the target sometimes.

\begin{figure}[htpb] % htpb allows all placement
  \centering
  \begin{subfigure}{0.3\linewidth}
    \centering
    \includegraphics[scale=0.3]{../fyp/assets/task-pics/reach-obs/random-front.png}
    \caption{Front}
  \end{subfigure}
  \hfill
  \begin{subfigure}{0.3\linewidth}
    \centering
    \includegraphics[scale=0.3]{../fyp/assets/task-pics/reach-obs/random-side.png}
    \caption{Side (Left)}
  \end{subfigure}
  \hfill
  \begin{subfigure}{0.3\linewidth}
    \centering
    \includegraphics[scale=0.3]{../fyp/assets/task-pics/reach-obs/random-top.png}
    \caption{Top}
  \end{subfigure}
  \vfill
  \begin{subfigure}{0.45\linewidth}
    \centering
    \includegraphics[scale=0.5]{assets/early-work/obs-random-scene-hierarchy.png}
    \caption{`ReachObs\_Random' Scene Hierarchy}
  \end{subfigure}
  \hfill
  \begin{subfigure}{0.45\linewidth}
    \centering
    \includegraphics[scale=0.5]{assets/early-work/obs-ind-random-scene-hierarchy.png}
    \caption{`ReachObs\_IndRandom' Scene Hierarchy}
  \end{subfigure}
  \caption{Reaching Task with an Obstacle}\label{fig:reach-obs-random}
\end{figure}


\subsubsection{Wrist Camera Alone isn't Enough}

As expected, this is where the single wrist camera started showing its shortcomings. 

\todo{add graph of going around the obstacle but not quite reaching the target (wrist)}
\todo{show this with other combinations of cameras, comment on if the l/r without wrist can learn to go around the obstacle easily? maybe not, generalisation might be hard with no wrist}

The agent would easily move around the obstacle, however, would struggle to make the last steps in touching the target. This is mostly due to the fact that the demonstrations (which are provided by RlBench) are not necessarily pointing the wrist of the robot and hence the camera mounted there to look towards the target. This means that the wrist camera alone does not necessarily move towards the target, rather the robot learns to move behind the obstacle and nothing else.

From experiments I have realised that it learns to move around the obstacle easily, using simple behavioural cloning. However, getting the last nudge to actually reach the target is where it falls apart, especially in more realistic scenarios where the target is randomised behind the obstacle.

\subsubsection{Other Cameras}
Introducing the other cameras placed around in the environment, such as the \verb|left shoulder| or the \verb|right shoulder| cameras we can confirm that the wrist camera alone is not sufficient \todo{insert figure here with wrist vs others }, and the combination of wrist and other cameras are almost always the best as more coverage of the workspace guarantees less occlusions and more information the agent can work with to make decisions. It was clear that the wrist camera alone wasn't going to cut it unless it learnt to look towards the target.

\subsubsection{Implementing `Looking' into the Demonstrations}\todo{haven't done this yet}\label{ew-looking-at-target}
issues this is not easy might be working on this as a part of approach 2 later

Another way to make sure agent understands to look at the target is teaching it to actively seek out its target, either following previous works such as \todo{find some prior info tracking works add ref} or with attention mechanisms that figure out what is important in a task without prior object information \todo{maybe reference this later}

\section{Expanding the Task Space: Grasping Tasks}
Another task which is likely to suffer from lack of viewpoints is a grasping task. So I designed these tasks around the idea of grasping. Firstly, a simple version (Figure \ref{fig:grasp-simple}) which the agent learns to reach then grasp the cubic target. 

Main differences between this and the reaching task is that the target here is tangible, so on top of being rendered it is also set to be \emph{collidable}. Another major addition is the usage of the \emph{extension string} as seen in \ref{subfig:simple-zoom-actions}, this instructs the demonstration engine to insert certain moves within the movement of the simple trajectory. In this case \verb|open_gipper()| ensures the gripper is open, then a later waypoint will instruct it to close.

\todo{maybe some experiment resutls here?}

The more complicated counterpart, shown in Figure \ref{fig:grasp-move}, is a scenario where the cube needs to be picked up then moved to the target location (designated in green)

\begin{figure}[htpb] % htpb allows all placement
  \centering
  \begin{subfigure}{0.3\linewidth}
    \centering
    \includegraphics[scale=0.2]{../fyp/assets/task-pics/grasp/simple-front.png}
    \caption{Front}\label{subfig:simple-front}
  \end{subfigure}
  \hfill
  \begin{subfigure}{0.5\linewidth}
    \centering
    \includegraphics[scale=0.3]{../fyp/assets/task-pics/grasp/simple-front-zoom-gripper_actions.png}
    \caption{Zoomed, with gripper action}\label{subfig:simple-zoom-actions}
  \end{subfigure}
  \caption{Simple Grasping Task}\label{fig:grasp-simple}
\end{figure}

\begin{figure}[htpb] % htpb allows all placement
  \centering
  \begin{subfigure}{0.45\linewidth}
    \centering
    \includegraphics[scale=0.2]{../fyp/assets/task-pics/grasp/move-front.png} 
    \caption{Front}\label{subfig:grasp-move-front}
  \end{subfigure}
  \hfill
  \begin{subfigure}{0.45\linewidth}
    \centering
    \includegraphics[scale=0.2]{../fyp/assets/task-pics/grasp/move-top.png}
    \caption{Top}\label{subfig:grasp-move-top}
  \end{subfigure}
  \caption{Grasping then moving}\label{fig:grasp-move}
\end{figure}



\missingfigure{grasp pic and possibly the demo gifs}
\todo{add a picture with the cube grasped and the wrist camera view seen at that point}
Although, initially the wrist camera shouldn't pose any problems, as we advance through the task, especially after we have grasped something the wrist mounted camera becomes heavily obstructed and becomes unreliable so basing our decisions on this medium alone is not ideal.

\subsubsection{Observations}
What I got from these experiments was that the agent can benefit from understanding its surroundings at a higher level and more importantly remembering them. This is because once the camera becomes obstructed, as with \emph{Grasp Then Move}, even if the agent could do some exploration to find the target, it wouldn't be ideal due to the restricted view it has access to. So, observing the environment before, and remembering important parts will be vital for the later stages of tasks. I aim to explore some pre-policy visual exploration of the environment to be albe to address issues such as this one.


\chapter{View and Feature Combination Investigations}\label{ch:view-comb}
This chapter I will be:
\begin{itemize}
  \item Creating more complicated toy tasks in RLBench
  \item Propose various imitation learning agents to solve these tasks
  \item Evaluate said methods by:
  \begin{enumerate}
    \item Providing different views and features
    \item Augmenting the fusion of these features
    \item Understand how the agent `views' a scene and what information it benefits from for which tasks
    \item Compare the strengths and weaknesses of these methods
  \end{enumerate}
\end{itemize}

% Reaching (NoObs) tasks
\section{Experimenting with Reaching}

\subsection{Creating a Policy}
The next step was to create a policy network. As I am mainly working on imitation learning -and specifically behavioural cloning- from demonstrations provided by the system, my network needs to be able to ingest these demonstrations and generate actions in the action space of the robot. Following the earlier parameters of a demo, I created the following network, Figure \ref{fig:policy-arch}. This takes in the given wrist rgb image and extracts features, which are then interpreted into a $8$ dim vector of \textbf{float}s as an action. First $7$ corresponding to the $7$ joints of the \emph{Panda} robot, and holding a velocity value for them, while the final value is the gripper state, $0$ meaning closed and $1$ meaning open, which is clamped by the movement system under the hood.

\begin{figure}[h]
  \centering
  \includegraphics[width=0.6\textwidth]{assets/early-work/cnn-encoder-policy-head.png}
  \caption{Simple Policy Network Architecture}\label{fig:policy-arch}
\end{figure}\todo[color=blue]{ius the quality bad? reuload or reexport the xmml is in assets}

Along with the policy the second most important of any ML workload is the way the data us regularised, processed, and loaded into the system for training or testing.

\subsubsection{Data Processing}\todo{talk about rgb transforms? uniforming etc, or remove not sure}

\subsubsection{Data Loading}
I followed a simple flattening approach for loading the data into the system. Using PyTorch's \textbf{Dataset} and \textbf{DataLoader} classes I created a dataset that can take in a raw list of demonstration, then flattens its observations into a tensor of shape \(\langle 3,~64,~64 \rangle \) (permuted from the usual \(\langle 64,~64,~3 \rangle \) for images due to Torch conventions of convolutional networks and where they expect the channel  dimension). Then the dataset makes individual observation indexable along with their corresponding action labels. Types given as:\mintinline{python}|DemoObsDataset: tensor[tensor[3,  64, 64], tensor[8]]|. Then the loader can manage the shuffling and batching as usual. Initially I kept the data unshuffled, to keep the data in its sequential form. While keeping the batch size as the demo length. This is because currently I am trying to overfit the network to the single demonstration given to it to gauge how long to train my networks for

\subsection{Initial Observations}
Starting with the 3 static versions of the task, where the target is placed as shown in \ref{fig:no-obs-3-views} seen from the wrist cameras. I wanted to get an idea of how to tune the policy parameters. While understanding the relationship between training length and varying observability of the target.

\begin{figure}[htbp]
  \begin{subfigure}{0.3\linewidth}
    \centering
    \includegraphics[scale=0.4]{../fyp/assets/demo-trials-no_obs/tasks/static-tasks-camera/initial-obs-side_l.png}      
    \caption{Left Side}
  \end{subfigure}
  \hfill
  \begin{subfigure}{0.3\textwidth}
    \centering
    \includegraphics[scale=0.4]{../fyp/assets/demo-trials-no_obs/tasks/static-tasks-camera/initial-obs-central.png}
    \caption{Central}
  \end{subfigure}
  \hfill
  \begin{subfigure}{0.3\linewidth}
    \centering
    \includegraphics[scale=0.4]{../fyp/assets/demo-trials-no_obs/tasks/static-tasks-camera/initial-obs-side_r.png}
    \caption{Right Side}
  \end{subfigure}%
  \caption{Three variations of the reach task with the obstacles sometimes out of view}\label{fig:no-obs-3-views}
\end{figure}

I tested multiple epochs of training with the simple policy and recorded the final distance to the target at the end of their episode. The episode length is determined by the demo lengths, which I defaulted to the maximum, will also try mean, as keeping a static episode length doesn't make sense (especially on later tasks where the demonstration episodes can drastically vary in length)

The success of the tasks are wired to reaching the target in the simulator and will send a \emph{DONE} signal if it is reached. This happens around $0.12$ metres to the target. I have also observed the target will reach a very close distance but it won't trigger the detection in Coppelia, I think this must be a bounding box issue, wither the dummy objects that are doing the collision detection are missing each other, or the polling rate in the simulator is not frequent enough to detect this change. Either way, I added a way to count closeness into success if it were close enough.

\subsubsection{Static Tasks}
Testing on the static versions of the task, a simple policy with training around $20$ to $100$ epochs seems to do the job well, see Figure \ref{fig:rno-static}. And this is mostly because without variation in position simple Behavioural Cloning can be employed by overfitting to the data given. 

As I trained for longer it seemed to overfit early move really slowly at the start of the episode, wasting steps and ending up far from the target. Curious observation is \todo[color=purple]{} the central tasks behaves better at higher epochs which I believe is not necessarily because of visibility (the \emph{conv} features are not necessarily guiding anything with $1$ demo) but rather the centrality, as it is right above the gripper a simple downward bias allows the arm to easily get close to the target.

\begin{figure}[htpb] % htpb allows all placement
  \centering
  \includegraphics[scale=0.5]{assets/cam-comb/reach-no-obs/rno_static.png}
  \caption{Epoch experiments with the static tasks using a single demonstration}\label{fig:rno-static}
\end{figure}

\subsubsection{Placing Randomly}
To test generalisability, I created a dynamic version of the task, where the target is now randomly placed within view (not necessarily always fully within view, but at least some parts are visible). This is achieved by using a \verb|SpawnBoundary| and randomly sampling the location of the target withing this for every new variation of the task, or for every new episode. Which can be seen in Figure \ref{fig:reach-no-obs}, the white dotted box being the boundary. This boundary is not rendered in the simulation visually. This guarantees variety in demonstrations as well as helps us create a generalisable policy.

\begin{figure}[htpb] % htpb allows all placement
  \begin{subfigure}{0.50\linewidth}
    \centering
    \includegraphics[width=\linewidth]{assets/cam-comb/reach-no-obs/rno_random-dist.png}
    \caption{Average Final Distance to Target}\label{subfig:rno-random-dist}
  \end{subfigure}
  \hfill
  \begin{subfigure}{0.50\linewidth}
    \centering
    \includegraphics[width=\linewidth]{assets/cam-comb/reach-no-obs/rno_random-success.png}
    \caption{Success Rate (\%) for the $10$ Test Demos}\label{subfig:rno-random-success}
  \end{subfigure}
  \caption{Experiments with randomly placed target}\label{fig:rno-random}
\end{figure}

To run the random tests, shown in Figure \ref{fig:rno-random}, in a comparable manner I reused my set of demos that were created and saved earlier for this task for training. Then a set of 10 demos were randomly generated at the start and after the agent with the specific parameters were trained, I evaluated these policies against the test counterparts.

Looking at graph \ref{subfig:rno-random-dist}, we can see that providing more demonstrations helps the policy generalise better to random locations, where the sweet spots seems to be around 10 demos and around $500$ where the success rate (\ref{subfig:rno-random-success}) is quite high

\subsection{Camera Limitations}\todo[color=red]{reread and make sure it is a good starter segue to obs (maybe move these to after obs?), mention this under reachObs! MOVE}
I started this section by limiting the learning to only the wrist mounted camera, which works well for this specific unobscured task. Introducing some of the other RGB cameras, specifically the \verb|left shoulder| or the \verb|right shoulder| views, does not necessarily benefit the performance, see \ref{fig:rno-random-cams}. Conversely, we are increasing the training time by adding more channels to the convolutional layers, which can be considered a drawback.

\begin{figure}[htpb] % htpb allows all placement
  \begin{subfigure}{0.50\linewidth}
    \centering
    \includegraphics[width=\linewidth]{assets/cam-comb/reach-no-obs/rno_random-cams.png}
    \caption{Average Final Distance to Target}\label{subfig:rno-random-cams-dist}
  \end{subfigure}
  \hfill
  \begin{subfigure}{0.50\linewidth}
    \centering
    \includegraphics[width=\linewidth]{assets/cam-comb/reach-no-obs/rno_random-cam_success.png}
    \caption{Success Rate (\%) of the $10$ Demos}\label{subfig:rno-random-cams-success}
  \end{subfigure}
  \caption{Experimenting with multiple RGB cameras}\label{fig:rno-random-cams}
\end{figure}

Although the performance does not outright increase, we can clearly see that having a different point of view can sometimes drastically affect the quality of the learnt network and hence the action produced. \todo[color=purple]{}  \todo{need to actually talk about the numbers here refer to graph}

However, this is not to say all tasks will be immune to benefits from extra views. An important part of this project is to understand what is most important for a robot to observe in its given environment and task and how can it optimally leverage this data to solve a task better. And increasingly more complex tasks should benefit from abundance of information.

\subsection{Increasing the Toy Task Complexity}\todo{subsection to above?}
Complexity of tasks can be increased in two main ways:
\begin{enumerate}
  \item Introducing non-linearities to the environment to make a scene more challenging to traverse for an agent
  \item Increase the movement or the level of interaction of the task at hand
\end{enumerate}
I plan to do these mainly by introducing obstacles; which will guide me to understand what an agent needs to understand navigation. Secondly, I want to branch out to a grasping task, to increase the number of items of execution, to evaluate the capability of an agent to use its understanding to complete increasingly more complicated tasks.

\section{Reaching with an Obstacle}
To add onto the reaching task, I introduced an obstacle placing mechanism as well as randomly placing the target behind this obstacle, ensuring the agent doesn't learn where exactly target is by looking at just the obstacle\todo[color=purple]. See Figure \ref{fig:reach-obs-random} for how this task looks and the check \todo[color=green]{add appendix link, and code} for the backend wiring of the task.

\subsection{Creating the Task}
There are two versions of this task, I thought it might be interesting to randomise the object firstly dependently then independently on the obstacle. The `dependent' randomisation called \verb|ReachObs_Random| samples the obstacle, which in turn controls the spawn boundary of where the target can spawn in, meaning the target will always appear behind the obstacle albeit, edges of it can sometimes stick out. Conversely, the `independently' random version, called \verb|ReachObs_IndRandom|\todo{also add to appendix and link here}, keeps the target spawn boundary fixed, meaning the target can be anywhere in the visible workspace, but it is not necessarily always covered by the obstacle. I can see that this potentially can be useful to keep the dataset a bit more diverse, and allow the wrist camera initially observe the target sometimes.\todo[color=red]{use this somewhere, or hint back to it, maybe even to say there was no difference}

\begin{figure}[htpb] % htpb allows all placement
  \centering
  \begin{subfigure}{0.3\linewidth}
    \centering
    \includegraphics[scale=0.3]{../fyp/assets/task-pics/reach-obs/random-front.png}
    \caption{Front}
  \end{subfigure}
  \hfill
  \begin{subfigure}{0.3\linewidth}
    \centering
    \includegraphics[scale=0.3]{../fyp/assets/task-pics/reach-obs/random-side.png}
    \caption{Side (Left)}
  \end{subfigure}
  \hfill
  \begin{subfigure}{0.3\linewidth}
    \centering
    \includegraphics[scale=0.3]{../fyp/assets/task-pics/reach-obs/random-top.png}
    \caption{Top}
  \end{subfigure}
  \vfill
  \begin{subfigure}{0.45\linewidth}
    \centering
    \includegraphics[scale=0.5]{assets/early-work/obs-random-scene-hierarchy.png}
    \caption{`ReachObs\_Random' Scene Hierarchy}
  \end{subfigure}
  \hfill
  \begin{subfigure}{0.45\linewidth}
    \centering
    \includegraphics[scale=0.5]{assets/early-work/obs-ind-random-scene-hierarchy.png}
    \caption{`ReachObs\_IndRandom' Scene Hierarchy}
  \end{subfigure}
  \caption{Reaching Task with an Obstacle}\label{fig:reach-obs-random}
\end{figure}\todo[color=blue]{smaller?} 

\subsection{Experimenting with Views}\todo[color=red]{}

\subsubsection{Using same unchanged policy form ReachNoObs}
\todo{run the best params we have for the first one here with nothing changed}

\subsubsection{Understanding the Data Management}
\todo{mention switching to demo dataset}
\missingfigure{shuffling plots}
Talk about how the data is loaded and created and managed for the network, talk about shuffling mechanisms and what I've found etc. talk about \verb|shuffle_obs_in_demo| and the other one.

\subsection{Improvements}



\subsubsection{Wrist Camera Alone isn't Enough}
As expected this is where the single wrist camera started showing its shortcomings. The agent would easily move around the obstacle, however, would struggle to make the last steps in touching the target. This is mostly due to the fact that the demonstrations (which are provided by RLBench) are not necessarily pointing the wrist of the robot and hence the camera mounted there to look towards the target. This means that the behavioural cloning agent learns to to the swaying motion around a large grey body, however, is not aware of the obstacle, or even understand the task is related to reaching for the obstacle and depends on its visual cues. 

\todo{add graph of going around the obstacle but not quite reaching the target (wrist)}
\todo{show this with other combinations of cameras, comment on if the l/r without wrist can learn to go around the obstacle easily? maybe not, generalisation might be hard with no wrist}

From experiments I have realised that it learns to move around the obstacle easily, using simple behavioural cloning. However, getting the last nudge to actually reach the target is where it falls apart, especially in more realistic scenarios where the target is randomised behind the obstacle. For static placement behind the wall, the agent, expectedly is quite good. \todo{maybe explain or evidence this, plateaus aroudn the same distance value and watchig t heroot act comfirms this}

\subsubsection{Other Cameras}
So, we can confirm that the wrist camera alone is not sufficient \todo{ref}, and the combination of wrist and other cameras are almost always better as more coverage of the workspace guarantees less occlusions and more information the agent can work with to make decisions. It was clear that the wrist camera alone wasn't going to cut it unless it learnt to look towards the target.\todo[color=red]{looking at he target??}


\todo{talk about the policy using differnt camera inputs to blend and make informed choices? maybe some plots here showing if it is better or not?}
\todo{this is important for plan2 later, as that would depend on such a mechanism, hint and even link that from here}

\subsubsection{Implementing `Looking' into the Demonstrations}\todo{haven't done this yet}\label{ew-looking-at-target}
issues this is not easy might be working on this as a part of approach 2 later
Another solution might be to experiment with the demonstration system to make sure we are pointing the wrist camera (so, the hand of our robot) towards its target as a demonstration trajectory is calculated \todo{explain that this proved tricky and might not even be worth it}

If we can't implicitly encode the `looking' information through the demonstration that means we will have to inject this information into our agent some other way. Another way to make sure agent understands to look at the target is teaching it to actively seek out its target, either following previous works such as \todo{find some prior info tracking works add ref} where object priors are incorporated into the learning or with attention mechanisms that figure out what is important in a task without prior object information \todo{maybe reference this later}.  

\subsection{Attending on a Camera in Given Combination}
\todo{talk about the implementation of the }
\subsubsection{MultiCNN}
\todo{explaint the implementations, maybe connect to }
\subsubsection{SingleCNN}
I initially though to disconnect the convolutional network, thinking the information can be encoded per view and then fused together to assign better meaning to the given pose. However, with the same logic the scene is still the same scene and encoding features together means that the network will learn to \todo{find a nice way to say the network will learn its fusing in the conv layers. maybe make sure it can}
% Grasping tasks
\section{Expanding the Task Space: Grasping Tasks}
Another task which is likely to suffer from lack of viewpoints is a grasping task. I first designed
a simple version (Figure \ref{fig:grasp-simple}) which the agent learns to reach then grasp the cubic target. 

Main differences between this and the reaching task is that the target here is tangible, so on top of being rendered it is also set to be \emph{collidable}. Another major addition is the usage of the \emph{extension string} as seen in \ref{subfig:simple-zoom-actions}, this instructs the demonstration engine to insert certain moves within the calculated trajectory. In this case \verb|open_gipper()| ensures the gripper is open, then a later waypoint will instruct it to close. 

As the task complexity increases, its design complexity also increases. Also, without any prior knowledge about 3D simulators and 3D design, it took me quite a long time to hunt everything about CoppeliaSim, RLBench, and PyRep to put these together. One criticism I have on these tools is the documentation is all over the place. \todo[color=green]{too ranty? rewrite or remove}

The more complicated counterpart, shown in Figure \ref{fig:grasp-move}, is a scenario where the cube needs to be picked up then moved to the target location (designated in green).

\begin{figure}[htpb] % htpb allows all placement
  \centering
  \begin{subfigure}{0.3\linewidth}
    \centering
    \includegraphics[scale=0.2]{../fyp/assets/task-pics/grasp/simple-front.png}
    \caption{Front}\label{subfig:simple-front}
  \end{subfigure}
  \hfill
  \begin{subfigure}{0.5\linewidth}
    \centering
    \includegraphics[scale=0.3]{../fyp/assets/task-pics/grasp/simple-front-zoom-gripper_actions.png}
    \caption{Zoomed, with gripper action}\label{subfig:simple-zoom-actions}
  \end{subfigure}
  \caption{Simple Grasping Task}\label{fig:grasp-simple}
\end{figure}

\begin{figure}[htpb] % htpb allows all placement
  \centering
  \begin{subfigure}{0.45\linewidth}
    \centering
    \includegraphics[scale=0.2]{../fyp/assets/task-pics/grasp/move-front.png} 
    \caption{Front}\label{subfig:grasp-move-front}
  \end{subfigure}
  \hfill
  \begin{subfigure}{0.45\linewidth}
    \centering
    \includegraphics[scale=0.2]{../fyp/assets/task-pics/grasp/move-top.png}
    \caption{Top}\label{subfig:grasp-move-top}
  \end{subfigure}
  \caption{Grasping then moving}\label{fig:grasp-move}
\end{figure}\todo[color=blue]{reshape}


\todo{add a picture with the cube grasped and the wrist camera view seen at that point}

Initially the wrist camera shouldn't pose any problems. Although, I suspect as we advance through the task, especially after we have grasped something, wrist camera becoming heavily obstructed will render it unreliable so basing our decisions on this medium alone might not be ideal.


\todo[color=pink]{experiments here, or move this where I can get some data on this}
\subsubsection{Observations}
What I got from these experiments was that the agent can benefit from understanding its surroundings at a higher level, and more importantly remembering them. This is because once thek camera becomes obstructed, as with \textbf{Grasp Then Move}, even if the agent could do some exploration to find the target, it wouldn't be ideal due to the restricted view it has access to. So, observing the environment before, and remembering important parts will be vital for the later stages of tasks. I aim to explore some pre-policy visual exploration of the environment then feed this information forward, possibly when it might be needed. For example residual forwarding of data might be used later \todo{should I keep this here?}


% Reaching (Obs) tasks
\section{Reaching with an Obstacle}
Circling back to reaching, I introduced an obstacle placing mechanism as well as randomly placing the target behind this obstacle, ensuring the agent doesn't learn where exactly target is by looking at just the obstacle\todo[color=purple]. See Figure \ref{fig:reach-obs-random} for how this task looks and the check \todo[color=green]{add appendix link, and code} for the backend wiring of the task.

\subsection{Task}\label{subsec:ro-creating-the-task}
There are two versions of this task, I thought it might be interesting to randomise the object firstly dependently then independently on the obstacle. The `dependent' randomisation called \verb|ReachObs_Random| samples the obstacle, which in turn controls the spawn boundary of where the target can spawn in, meaning the target will always appear behind the obstacle albeit, edges of it can sometimes stick out. Conversely, the `independently' random version, called \verb|ReachObs_IndRandom|[color=green]{also add to appendix and link here}, keeps the target spawn boundary fixed, meaning the target can be anywhere in the visible workspace, but it is not necessarily always covered by the obstacle. I can see that this potentially can be useful to keep the dataset a bit more diverse, and allow the wrist camera initially observe the target sometimes.\todo[color=red]{use this somewhere, or hint back to it, maybe even to say there was no difference}

\begin{figure}[htpb] % htpb allows all placement
  \centering
  \begin{subfigure}{0.3\linewidth}
    \centering
    \includegraphics[scale=0.3]{../fyp/assets/task-pics/reach-obs/random-front.png}
    \caption{Front}
  \end{subfigure}
  \hfill
  \begin{subfigure}{0.3\linewidth}
    \centering
    \includegraphics[scale=0.3]{../fyp/assets/task-pics/reach-obs/random-side.png}
    \caption{Side (Left)}
  \end{subfigure}
  \hfill
  \begin{subfigure}{0.3\linewidth}
    \centering
    \includegraphics[scale=0.3]{../fyp/assets/task-pics/reach-obs/random-top.png}
    \caption{Top}
  \end{subfigure}
  \vfill
  \begin{subfigure}{0.45\linewidth}
    \centering
    \includegraphics[scale=0.5]{assets/early-work/obs-random-scene-hierarchy.png}
    \caption{`ReachObs\_Random' Scene Hierarchy}
  \end{subfigure}
  \hfill
  \begin{subfigure}{0.45\linewidth}
    \centering
    \includegraphics[scale=0.5]{assets/early-work/obs-ind-random-scene-hierarchy.png}
    \caption{`ReachObs\_IndRandom' Scene Hierarchy}
  \end{subfigure}
  \caption{Reaching Task with an Obstacle}\label{fig:reach-obs-random}
\end{figure}\todo[color=blue]{smaller?} 

\subsection{Running the Task}
There were a few interesting things of note happening on this task. 

I used the unchanged reaching policy from earlier\todo[color=green]{ref}. Unsurprisingly this did not perform amazingly, however, there are some important lessons to learn form what we are seeing here. Firstly, I adopted to observe the `minimum distance' to target this time \ref{subfig:ro-random-demo-mindist-10}, however the `final distance ' graph can be found in the appendix \todo[color=green]{link appendix}. 

\subsubsection{Why is `IndRandom' just worse?}
The experiments \todo[color=green]{appendix}, show that the `distance' graphs are translated upwards by roughly $0.1$m while the success rate is halved.

While they were trained on demos for their respective generation methods.They were evaluated using a shared a test set created using just \textbf{ReachObs\_Random}. This was a decision to be able to compare their results with each other and evaluate the differences of spawning techniques

This means that a policy trained on the independent generation is not as good at reaching the target successfully as often, but still manages to learn to get around the obstacle. Which tracks, as the independent task is easier, and predictions on a harder test set are tougher to fit.
This is likely due to to saved training demos for independent spawning being less occluded from the start. Training and testing the on the demos belonging to this task \ref{fig:ro-indrandom-demo-cams}, the results look a lot similar to what we expect from this model.


\begin{figure}[htpb] % htpb allows all placement
  \centering
  \begin{subfigure}{0.45\linewidth}
    \centering
    \includegraphics[width=\linewidth]{assets/cam-comb/reach-obs/ro_random-demo-mindist-10demos.png}
    \caption{Minimum Distance to Target}\label{subfig:ro-random-demo-mindist-10}
  \end{subfigure}
  \begin{subfigure}{0.45\linewidth}
    \centering
    \includegraphics[width=\linewidth]{assets/cam-comb/reach-obs/ro_random-demo-success-10demos.png}
    \caption{Success Rate (\%) of Task}\label{subfig:ro-random-demo-success-10}
  \end{subfigure}
  \caption{Policy trained on 10 demos, using the \emph{demo dataset} from \ref{subsec:grasp-data-loading-changes}}\label{fig:ro-random-demo-cams}
\end{figure}

\begin{figure}[htpb] % htpb allows all placement
  \centering
  \begin{subfigure}{0.45\linewidth}
    \centering
    \includegraphics[width=\linewidth]{assets/cam-comb/reach-obs/ro_indrandom-obs-mindist-10demos.png}
    \caption{Minimum Distance to Target}\label{subfig:ro-indrandom-demo-mindist-10}
  \end{subfigure}
  \begin{subfigure}{0.45\linewidth}
    \centering
    \includegraphics[width=\linewidth]{assets/cam-comb/reach-obs/ro_indrandom-demo-success-10demos.png}
    \caption{Success Rate (\%) of Task}\label{subfig:ro-indrandom-demo-success-10}
  \end{subfigure}
  \caption{Policy trained on 10 demos, using the \emph{demo dataset} from \ref{subsec:grasp-data-loading-changes}}\label{fig:ro-indrandom-demo-cams}
\end{figure}

\subsubsection{Dataset Differences}
I initially was planning on running the system with just the \emph{demo dataset}, however, one of the runs I accidentally included both. After reviewing the data, I realised the \emph{obs dataset} which prematurely flattens and forgets demo boundaries within the data seems to be performing better in this task; see \ref{subfig:ro-random-demo-success-10} and \ref{subfig:ro-random-obs-success-10}, more successes. Curiously though, the minimum distance to the target seems to be quite consistent, while the higher success, or course generally leads to a slightly smaller minimum distance.

My theory of why this happens is quite nuanced.\todo[color=purple]{} The \emph{obs dataset} here was trained using \verb|minibatch_size = 32|, the maximum episode length in these demos is $82$. This means that each epoch where the weights of the system are updated, we were exposing it to around one third of the data for a full episode. So, the first easy improvement is from extra training. The \emph{demo dataset} here was using a minibatch size of $1$, so for 10 demos, there is 10 updates per epoch, while the \emph{obs dataset} will do around $2.5$ more updates. 

The average episode for this task starts linearly getting near to the side of the obstacle, far away enough to not collide with it, then calculates a non-linear trajectory to curve inwards towards the target for the final bits of the movement.\todo[color=purple]{get a nice video of this for the presentation}. The non-linear part of the demo takes more steps than the linear part. Evident from the fully linear tasks like the grasp and the no obstacle reaching being around $40$ to $60$ steps. This means that the first $30$ ish steps are what the robot takes to move near the obstacle. Therefore, my running theory is that this system was performing better because the task was unintentionally compartmentalised into two sections. 1: Reaching to a side of the obstacle; and 2: Reaching towards the obstacle.

So, by unintentionally using this batch size we were updating the weights once to reach to a side of the obstacle then to the target. Finding this interesting, I ran more minibatch sizes to see if this was the case (see \ref{apx:Z-ro_random-obs-trials-success}), although, this doesn't seem to be too important once the number of epochs increase, at lower epochs the success rate decreases across the board for larger batch sizes. Although, this is not conclusive, and it might not be ideal nitpicking every demo to specifically train each network, I thought this was an interesting property of sequential batching, even if there is no mechanism to explicitly take care of sequencing in the policy.

\begin{figure}[htpb] % htpb allows all placement
  \centering
  \begin{subfigure}{0.45\linewidth}
    \centering
    \includegraphics[width=\linewidth]{assets/cam-comb/reach-obs/ro_random-obs-mindist-10demos.png}
    \caption{Minimum Distance to Target}\label{subfig:ro-random-obs-dist-10}
  \end{subfigure}
  \begin{subfigure}{0.45\linewidth}
    \centering
    \includegraphics[width=\linewidth]{assets/cam-comb/reach-obs/ro_random-obs-success-10demos.png}
    \caption{Success Rate (\%) of Task}\label{subfig:ro-random-obs-success-10}
  \end{subfigure}
  \caption{Policy trained on 10 demos, using the \emph{obs dataset} from \ref{subsec:reach-data-loading}}\label{fig:ro-random-obs-cams}
\end{figure}

\subsubsection{Wrist Camera Alone isn't Enough}
Although, success rate graph looks nice at first glance, there is a distinct lack of blue; which is the wrist camera. Looking at the minimum distance the wrist achieves (around slightly less than$0.45$) we can see what it does a very good job of going around to the side of the obstacle (as the target is around $0.5$ metres from the obstacle), however, would struggle to make the last steps in touching the target. This is mostly due to the fact that the demonstrations are not necessarily pointing the wrist of the robot towards the target. This means all the wrist camera sees once it is past the obstacle is the table (a static view) which means it will likely likely get confused here and as there are no distinct features to latch onto, it will just execute a random movement it synthesised from the demos It was generally the case that the demo provided would take a wide swing around the obstacle, and almost every wrist rollout I managed to witness, it would just keep slinging the arm away -which is why the final distance graphs were very skewed.\todo[color=green]{link appendix graph} The wrist combined views were also prone to these, however sometimes the assisting camera would sometimes pick something up and course-correct.

Therefore, combining wrist and other cameras are almost always better as more coverage of the workspace guarantees less occlusions and more information the agent can work with to make decisions. Without a way to `look' towards the target the wrist camera alone is not going to cut it. Combining these ideas I moved onto the next section.

\section{Camera Attention Based on Colour}\label{sec:reach-obs-naive-cam-attn}
First idea bordering on `active vision' was to selectively accept the encoding from a single camera depending on what it was seeing. So, not necessarily looking around, but accepting the inputs from one of the cameras it has access to, guided by some prior information. This requires at least two camera views to `blend' otherwise given only one camera, it will act like the old policy.

The problem this is meant to fix is the policies involving wrist cameras tend to focus too much on wrist information when it is not relevant. As we can see in \ref{fig:ro-random-demo-cams}, these policies tend to sit above the $0.4$ min distance area which means they are continually hitting against the obstacle. The obstacle is around $0.39 \approx 0.43$ metres from the obstacle depending on its generation. So, if we can attend to different cameras depending on occlusion, we might be able to fix this problem.

\subsubsection{The Colour Score}
Given a set of RGB views (normalised to be  in range \(\left[0, 1\right]\)) I want to score them and weight them according to their perceived \emph{utility}, the first metric that comes to mind is colour. For every view, $\mathcal{V}_i$, $i$ representing different cameras and every pixel $\left( u, ~v\right)$ we calculate its euclidean distance to the target's colour: \( d_i\left(u, ~v\right) = ||\mathcal{V}_i\left(u, ~v\right) - t_{colour}||\), this is to guarantee the per-pixel score is a positive real number. 


Then to isolate our target colour I introduced two hyperparameters: \textbf{tolerance} and \textbf{softness}. Tolerance, $\varepsilon_{tol}$, defines the acceptable range of differences of what we can classify as the \emph{target colour}; red in this case. Softness, $\varepsilon_{sof}$, which helps adjust the harshness of sigmoid which we are feeding this for a signal. If softness is low the values within the tolerance will be a strong signal ($\approx 1$) and values without will be near $0$. 

So, the final classification of our score is defined as \(m_i\left(u, ~v\right) = \sigma \left(\left(\varepsilon_{tol} - d_i\left(u, ~v\right)\right)\times \varepsilon_{sof} \right)\) where \(\sigma\left(z\right) = \frac{1}{1 + e^{-z}}\): the sigmoid function. 
Then for scoring the image, I pool these tensors. Initially testing \emph{mean} pooling, \(\mathbf{S}_i = \max_{1<=u<=W, 1<=v<=H}m_i(i, ~v)\), which essentially acts as a signal of ``is the target colour in view or not?''. Then used \emph{mean} pooling \(\mathbf{S}_i = {\frac{1}{WH}\sum_{u = 1}^{W}\sum_{v = 1}^{H}m_i\left(u, ~v\right)} + c\) to give an overall visibility of the target within a view -adjusted by a constant. This will capture more nuanced inherent information about ``how much of the screen is the target?'' which is information about distance and occlusions.

The decision to use euclidean distance and sigmoid was to keep the colour score differentiable. This mainly helps in capturing the dynamics of the colour interaction within our neural network. A second benefit is, if I want to make the colour to be a learnable parameter we will be able to get the gradient of the loss with respect to ${RBG}_{target}$ to achieve this. Which can eliminate the colour prior.

\subsubsection{Scoring the Weighted Features}
When we have the score per view, we can weight the scores of each view against each other
\( \tilde{\mathbf{S}}_{i, x} = \frac{\mathbf{S}_{i, x}}{\sum_{}^{k \in v} \mathbf{S}_{i, k} + 10^{-6}} \); scalar term $10^{-6}$ being a numerical stabiliser and preventing vanishing gradients while not adding too much bias.

The features extracted from the cameras are concatenated with their respective score \( \begin{bmatrix} \mathbf{F}_i \\ \tilde{\mathbf{S}_i} \end{bmatrix} \in \mathbb{R}^{d + 1}\) where $d$ is the feature dimensions encoded after the conv layers. This, in turn, is fed into a attention network, to guess what the most important parts of the gives views are. Outputs of the network are then fed through a temperature scaled \emph{softmax}, \(sm\left(s\right)_i = \frac{e^{s_i / \mathcal{T}}}{\sum_{j \in v}{e^{s_j / \mathcal{T}}}}\). I also included temperature scaling here to introduce entropy in the resulting probability distribution across the views. This way the network will not be strongly attending on only one view.

\subsubsection{Encouraging Attention}
Final part of the puzzle is tuning the view scoring network. I calculated the attention predictions using KL divergence, \(D_{KL}\left(P ~||~ Q\right) = \sum_{x \in \mathcal{X}} P\left(x\right) \log \frac{P\left(x\right)}{Q\left(x\right)}\). This is a measure of the probability distribution difference\todo[color=green]{reference}. So, augmenting the loss of the overall network to be \(loss = loss_{action} + \lambda_{attn} D_{KL}\left( \hat{W}_{attn} ~||~ \mathbf{S}_i\right)\) where $\hat{W}_{attn}$ is the predicted weights, encouraging them to stay close to the ground truth scores.

\subsection{Feature Encoder Coupling}
Final proposition for this network is to check whether disconnecting the feature encoders makes a difference. In theory, every view having its own feature extractor might allow the network to learn inherent pose and view information for that specific view. For example, the shoulder cameras.  Although, following the same logic, keeping a joint extractor might allow to policy to learn how different poses and information from those poses interact in tandem. As the workspace and hence the observations are quite similar.



% Depth Interfacing - under grasping
\section{Depth Interfacing}\label{sec:depth-interfacing}
Understanding distance through depth interfacing is an important part of perception. Continuing from the drawn parallels to humans, we place object in our fields of vision by our two eyes. Stereo-vision, allows us to process two slightly different poses of a target object to reinforce our understanding of where that object is in the environment around us. Other information such as lighting (and shadows) may unconsciously help us as well. The main takeaway is that understanding distance to an object goes a long way in firstly understanding how to approach an object. There are a few ways to achieve this in robotics. In RLBench specifically is either to use two cameras (with a known distance between the two cameras to adjust poses with known intrinsics). However, RGB cameras are not the only things we have access to. We also we have a depth sensor.

This is because,active vision is to be able to use minimal amounts of viewpoints. In this scenario, I want to be able justify to use the wrist camera accompanied by the wrist depth sensor to replace any other workspace cameras. Therefore, it is important to study shortcomings of individual sensors with respect to others to understand how policies with limited access to sensors can be proficient at using them.

\subsection{Grasping with Varying Depths}
A grasping task makes sense for this experiment. Because unlike the reaching tasks from earlier, the robot will need slightly more precision in executing its grasp and actually grabbing the target. The task performer needs to figure out where an object is before attempting to grab it. So, I created a modified version of the simple grasping task where the target object's distance and scale can be externally varied to observe the behaviours of agents in a controlled manner.

\begin{figure}[htpb] % htpb allows all placement
  \centering
  \begin{subfigure}{0.4\linewidth}
    \centering
    \includegraphics[width=0.3\linewidth]{assets/depth-interfacing/normal-size-grasp.png}
    \caption{Normal Target Size}\label{subfig:normal-grasp}
  \end{subfigure}
  \begin{subfigure}{0.4\linewidth}
    \centering
    \includegraphics[width=0.3\linewidth]{assets/depth-interfacing/smaller-grasp.png}
    \caption{Smaller Target Size}\label{subfig:small-grasp}
  \end{subfigure}
  \caption{Visualisation of the Depth Interfacing experiment task}\label{fig:di-task}
\end{figure}

Figure \ref{fig:di-task}, is the general setup I am planning on using to evaluate the depth sensor versus a multi-view setup. Initial observations from the side clearly indicate that these are two different targets and will require different reach lengths before the agent can attempt to grab them.

\begin{figure}[htpb] % htpb allows all placement
  \centering
  \begin{subfigure}{0.2\linewidth}
    \centering
    \includegraphics[width=\linewidth]{assets/depth-interfacing/normal-size-wrist.png}
    \caption{Normal RGB}\label{subfig:normal-rgb}
  \end{subfigure}
  \begin{subfigure}{0.2\linewidth}
    \centering
    \includegraphics[width=\linewidth]{assets/depth-interfacing/smaller-wrist.png}
    \caption{Smaller RGB}\label{subfig:small-rgb}
  \end{subfigure}
  \begin{subfigure}{0.20\linewidth}
    \centering
    \includegraphics[width=\linewidth]{assets/depth-interfacing/normal-depth.png}
    \caption{Depth Mask}\label{subfig:normal-depth}
  \end{subfigure}
  \begin{subfigure}{0.20\linewidth}
    \centering
    \includegraphics[width=\linewidth]{assets/depth-interfacing/smaller-depth.png}
    \caption{Smaller Mask }\label{subfig:small-depth}
  \end{subfigure}
  \caption{Wrist RGB and Depth Masks for the tasks}\label{fig:di-rgb-vs-depth}
\end{figure}

However, as seen in the comparison in \ref{subfig:normal-rgb} and \ref{subfig:small-rgb}, the RGB outputs look practically the same, and will very likely produce extremely similar features after extraction. A way to differentiate them would be to utilise the wrist depth mask in this encoding. As shown in \ref{subfig:normal-depth} and \ref{subfig:small-depth}, they now carry different features in those areas. In the depth mask the darker colours indicate closer objects, and the information is encoded as floats. 

\begin{figure}[htpb] % htpb allows all placement
  \centering
  \begin{subfigure}{0.2\linewidth}
    \centering
    \includegraphics[width=\linewidth]{assets/depth-interfacing/normal-l_rgb.png}
    \caption{Normal Left}\label{subfig:normal-l-shoulder}
  \end{subfigure}
  \begin{subfigure}{0.2\linewidth}
    \centering
    \includegraphics[width=\linewidth]{assets/depth-interfacing/normal-r_rgb.png}
    \caption{Normal Right}\label{subfig:normal-r-shoulder}
  \end{subfigure}
  \begin{subfigure}{0.2\linewidth}
    \centering
    \includegraphics[width=\linewidth]{assets/depth-interfacing/smaller-l_rgb.png}
    \caption{Smaller Left}\label{subfig:smaller-l-shoulder}
  \end{subfigure}
  \begin{subfigure}{0.2\linewidth}
    \centering
    \includegraphics[width=\linewidth]{assets/depth-interfacing/smaller-r_rgb.png}
    \caption{Smaller Right}\label{subfig:smaller-r-shoulder}
  \end{subfigure}
  \caption{Left and Right Shoulder RGB Cameras}\label{fig:di-lr-shoulder}
\end{figure}


Another possible differentiating factor is the shoulder cameras. These also provide extra information about the scene the agent can use to understand its 3D geometry, even if it isn't explicitly taught. Although these are also not perfect, as seen in \ref{subfig:smaller-l-shoulder}, this camera is completely obstructed by the robot and is not seeing the target. This is not immediately problematic, because a smart enough system might be able to reason that the target is visible on the right and not on the left, leading to devising the correct depth for it.

% \subsubsection{Adding Stereo Wrist Vision}\todo[color=blue]{can be removed not sure}
% A possibility to overcome this \emph{self-occlusion} is to introduce stereo vision at the wrist level. Adding 2 off-centre cameras to the gripper, likely to the left and right of the existing wrist camera, will fix the self-occlusion, and possibly create a better comparison to the \emph{Wrist RGB} and \emph{Wrist Depth} combination. However, this comes with a lot of rewiring of RLBench and with its likely hidden issues that will pop up later. I did not want to take on this large architectural change before continuing with main investigation of the project, however, if timings permit this will be an important addition to the testing suite.

% proposed policies and learnings
\section{Multi-Modal Policies}
\subsection{Deeper/Better Feature Understanding - RCNN}\todo{I feel like this section is talking about the wrong thing}
First part of any policy is extracting features from the available views. Experimentation up to this point was using a bespoke CNN architecture. However, I wanted to check if an off-the-shelf model would be able to extract more informative features.
\todo{describe a problem the resnet would fix}

Another thing I wanted to try was to include off-the-shelf models to help extract features and information from my views. As one of the problems I described faced earlier was from the agent not remembering the earlier information that it has seen, I believed an important part of increasing the workspace understanding would be to incorporate residual connections.

I experimented with various sizes of ResNet networks \todo[color=green]{reference}. Essentially replacing my feature extraction block with modified versions of \emph{resnet18, resnet34, and resnet50} \todo[color=green]{maybe not mention all, afterall the large ones will be ass when the image is so small}.

They had to be modified because of the size restriction I had on my views.\todo{explain why the kernel size had to be dropped}

I had to modify these, because the unmodified version using a main CNN of kernel size $7$ was too large and appropriate features were not being extracted from the views I was feeding. This was evident from observing the agent act in a test scenario where the arm would do nothing remotely similar to what the demo did, or even what the task is about.\todo{move this at the start of the evaluation chapter}. 

\todo[color=pink]{run the original confirm it does not work, if not get a diagram of kernel sizes 7 and 3 and an explanation why this might be the case.}


A reason these might not have worked well is because of one of my earlier constraints. The image sizes being \(64 \times 64\) pixels, might not be enough to extract meaningful information with a ResNet. This is likely due to its aggressive pooling between the layers and especially during the residual connection and the aggressiveness only increases in larger models. \todo[color=red]{not actually sure if this is right, fact check, wrote this some time ago} 

\subsubsection{increasing the camera view for experimentation}\todo[color=red]{larger image trials? expand the task to be lager 128 or even 224 seems common with resnet}

\subsection{Feature Level Fusion}\todo{really long intro, consolidate}
On top of just naively combining views (or modalities) and expecting the policy to understand what it means to `see' is a tall order. 

I have adopted a model-free approach in making these policies, and decided to use representations of information for predicting the next motion.
\todo{maybe more on latent representations and encoder models, here with refs}

The step after extracting representation is to learn how we can make the most of the information by incorporating the various views and sensors and seeing what combination of feature extraction and their fusion allows the most information gain.

So, after extracting features from available sensors. Using multiple feature extractors means that these features will need to be merged before making a final decision on movement.

If these extracted features can be thought of as encoded information, as with latent models, multiple extractors will lead to to many encodings. So, the next challenge becomes merging; and not just how but when to merge features from one view to another. I will explain their potential positives and investigate the consequences of each.\todo[color=green]{this might be a good segue or part in presentation}

\subsubsection{Early Fusion}
Fuse different modalities into a single input and run through the encoding model. This usually involves combining the raw data sources. Before any information is extracted or the data is modelled in any way.

Therefore, its advantages mostly lie in its simplicity. No dedicated processing has to be done per information stream. On the other hand, this can fall short of capturing complex interactions between these modalities; depending on the network and the semantic information each modality provides.

\subsubsection{Late Fusion}
This is the complete opposite of early fusion. Each modality will be run through its own model and predict a representation. The immediate advantage is that each model can learn a rich representation relating to only its given modality


\subsubsection{Intermediate Fusion}
This is the most common middle-ground. While separated feature learners can be used per desired modality, the representation extracted from them will be fused together. Then this aggregated latent representation can be passed through another model to get the final predictions. This is a tradeoff based system between early and late fusion. Most of the proposed systems will follow this idea.

\todo{i will argue separating the grasping network will be late fusion fuse at the actionn level and not separating it will be fully intermediate}

\section{Proposed Fusion Methods}
I designed a a poly-policy which can be instantiated in any fusion configuration to handle its given modalities differently before making a final movement prediction. \todo{make a diagram - consult the sheet}

\todo{add the fusingEncoder and the config to the appendix}
\subsection{Simple Stacking}
An example of early fusion, this is the naive method of just stacking the views on the channel dimension, and has clear disadvantages for this specific system.

Firstly, for the RGB cameras, there is definite misalignment, all these cameras have different poses and will disagree on what they see. Leading to features not lining up, leaving us with a non-optimal and more importantly a brittle\todo{explain brittle? or does the next senetence take care of it} policy. This could lead to unintended coupling of information which can lead to the agent making wrong choices during inference when the same views don't necessarily align during rollout.\todo{simple diagram of overlaid squares (arrow) model (arrow) representation} This will act as a baseline to compare other fusion strategies.

\subsection{Understadning Depth Separately}\label{subsec:policies-understand-depth-sep}
RGB provides a denser representation of the system and in a configuration with multiple RGB views the depth information may be drowned out. Therefore, an important first step was to learn the depth features separately and then make them influence the action. The idea here is to have a RGB encoder and a separate Depth encoder CNNs, then learn the best way to merge this data together. I propose three different methods to do this:

\begin{itemize}
  \item Concatenation of the features from each CNN \todo{add figure, explain more if needed}
  \item A second layer CNN, convolve the RGB outputs and the Depth outputs to learn joint features \todo{add figure, explain more if needed}
  \item A shallower secondary CNN to learn simple \emph{gating} from already extracted RGB and Depth features. \todo{add figure, explain more if needed}
  \item A Cross Attention Mechanism to attend to the important parts of from each extracted feature.
\end{itemize}
Which are ultimately decoded into an \emph{action}, $\hat{A}$.

\subsection{Feature-wise Linear Modulation}\todo{ref this paper arxiv1709.07871}
To take a step back from all the excessive RGB data. I wanted to understand how modulating similar views with each other to understand similar observations from each other's perspective. Allowing the network to learn and discover its own trends is a fine task, however, implicitly injecting cues from modality into another will, in theory, allow the learning to be more efficient.


A light-weight and quick way to bias network parameters with conditional data is Feature-wise Linear Modulation (FiLM).\todo{ref} 
A FiLM network learns to adaptively influence the weights of another neural network by learning $f_c$ and $h_c$ as functions on the conditional input $\mathbf{x}_i$. So:

\[
\text{FiLM} = \left( \mathbf{F}_{i, c} | \gamma_{i, c}, ~\beta_{i, c} \right) = \gamma_{i, c}\mathbf{F}_{i, c} + \beta_{i, c} \qquad  \gamma_{i, c} = f_c\left(\mathbf{x}_i \right), \beta_{i, c} = h_c \left( \mathbf {x}_i \right)
\]
My implementation involves \todo[color=green]{appendix link} creating a simple linear network to generate the two parameters. The singular linear network means the parameters will have shared weights, which makes sense as the modulation is being condition on a single view, and hence a dependence connection. I also initialise this network to be the identity modulation, meaning \(\gamma = 1 \text{and} \beta = 0\). \todo{why, not sure?}

Another inherent benefit is the network can be kept at a minimal size. And finally, a FiLM module can be plugged in at multiple different places in the network at different scales. \todo{explain more maybe}

I created 3 different modulation configurations: wrist RBG modulated with wrist depth\todo{diagram}, depth modulated wrist RGB\todo{diagram}, and both modulated with each other\todo{diagram}. 

Where the modulated features act as the latent encoding, which can be used to predict an action On top of the three configurations, I also FiLMed at different depths: early FiLM, where feature shapes are matched then modulated leading to a coarser scale modulation; also a finer version where the order of down-sampled learning and modulation are swapped.\todo{explain better maybe}


\subsection{Derivative Methods}
The final propositions include 4-way fusion models that take into account ideas from earlier.
\begin{itemize}
  \item Fully separated feature extractors, all concatenated before action prediction \todo{add figure}
  \item Pairwise FiLM between all modalities: wrist RGB and w Depth, then Left Shoulder RGB and Right Shoulder RGB; then concatenated before action prediction\todo{figre}
  \item Finally 4-way multi-modal cross-attention. \todo{add math and figure if needed}
\end{itemize}

\subsection{Temporal Understanding of Demonstrations}
The actions in my system are heavily dependent on the context of where the robot came from. And the demonstrations contain coherent and smooth trajectories. Although, I am preserving the sequential nature of the demos in training. There is no implicit underlying mechanism to enforce the agent will learn from this. Due to the partially observable setting of my system, and the robot not knowing any extra information about its state (model-free), it is important to be able to keep a historical understanding of what has happened, or what the agent believes is important for that moment in time.

So, I incorporated recurrent models into the latent representation. This is so that the embedding of the current states will depend on a history of the earlier observations or states.

A pro of RNNs, compared making a bespoke sequential model is they are not length constrained \todo{was an earlier problem in 2d, maybe mention}. The sequences can be of any length which will help model the demos I have with varying episode lengths. Som compared to 3D CNNs \todo{ref}, which can also incorporate spatial and temporal information \todo{ref}


Another theoretical advantage is that the long-term memory can be used to have a \todo{mamybe mention the remoark from the then move test, }

\subsubsection{Latent RNN Encodings of Movement Steps}
To encode timestep information into my already existing features, I added an Long-Short Term Memory (LSTM) network module near the end of the pipeline\todo{add figure}. Meaning the history will be kept in the latent space.

So, this simple idea of introducing temporal noise into the decision process, should in theory allow for more concise predictions that are grounded in what the agent has observed so far.

\subsection{ViT Encodings}\todo{kind of dabled with this but will not be testing it}
\todo[color=red]{not done much on this, but may be able to talk about how I would go about it or just remove}



\subsection{Proprioception}\todo{maybe talk about why I left it out earlier in the chapter}
Up until this point, I left proprioceptive data out of the training, mainly because I wanted to train on purely visual feedback to see what it could achieve without any other state information. Secondly, including state information about the robot, would not necessarily immediately help with the 

\subsection{Numerical Foundations}\todo{talk about the general math, like what are the shapes of the data, what is the loss minimising-maximising, and notation}

The policies above are trained in the same way and the only separation is the modular feature extraction blocks, as well as what data they require. All input and output is handled within the network and the policy interface exposed to the agent is compatible with all the ones proposed in this project.

\subsubsection{The Dataset}
A demonstration of episode length $T$ is defined as: \(\mathcal{D} := \{\langle o_i, ~a_i\rangle\}_{i = 1}^{T} \). Observation, \(o_i \in \mathcal{O}\) is defined as the available modalities to the system with shape \(\mathcal{O} := \mathbb{R}^{c \times W \times H}\). This corresponds the resolution of the sensor (\textbf{W}idth and \textbf{H}eight) also $c \in \left[1, 10\right]$ which correspond to the available modality data concatenated on their channel dimension. The order of $c$ is always:\todo{figure here with the stuff stacked}
\[
{RGB}_{wrists}, ~{RGB}_{{left}\_{shoulder}}, ~{RGB}_{{right}\_{shoulder}}, ~{Depth}_{wrist}
\] 
where the $RGB$ data is $3$ channels wide while $Depth$ is only 1. \(a_i \in \mathbb{R}^{8 \times 1}\) is the corresponding joint velocities at timestep $i$, which acts as our ground truth. 

The Data Loader, as discussed \todo{ref}, will provide a randomly shuffled batch of $b$ collated demos, along with extra information. \( \langle O^B, ~A^B, ~\left( P^B, ~l^B \right) \rangle\), where \(O^B \in \mathbb{R}^{ B \times c \times W \times H}\). The observation batch size, $B$ is equal to \({\sum_{ d \in \mathcal{D}^B}|d|}\), this is the sum of the length of all demos in this training batch. Similarly, \(A^B \in \mathbb{R}^{B \times 8}\) per joint velocity (7) and final grasp indicator, \(P^B \in \mathbb{R}^{B \times 7}\) per joint angle - the proprioception data. Finally, \(l^B \in \mathbb{Z}^{b}\), contains the original lengths of the individual demos.

Then during the forward pass modalities of interest will be extracted from $O^B$ and features will be extracted depending on what fusion flavour is picked.



\subsubsection{Fine-Grained per Time-Step Training}\todo[color=green]{math fixing}
The RNN policy is disconnected from the main system.\todo{link the FusionRNNPolicy appendix} Differently from above, the batched demonstrations will now respect the individual lengths of provided demonstrations. The loader will now emit: \( \langle {O'}^B, ~{A'}^B, ~\left( {P'}^B, ~l^B \right) \rangle\). The \emph{prime} ($'$) indicates a change in their batch dimesion, each now follow $\mathbb{R}^{b \times T_{max} \times \ldots}$ and  $T_{max}$ is the longest demonstration in this batch \(\max l^{B}\). We pad every demo to this length while keeping the original lengths $l^{B}$.

To fit the network to our data quite closely I opted to train the network with information at every step. Meaning every single step $i$ in a demo $mathcal{D}$ with episode length $T$, will have an action predicted with respect to $\mathcal{D}_{t < i}$ and will have an impact on the loss of the system. 

The loss will be calculated using the predicted actions, $\hat{A}$ and the ground truth labels, $A$; as \( \mathcal{L}^{action} = {loss}_{mse}\left(A_{x, t} - \hat{A}_{x, t}\right)\) for $x$ being the batch number in $1..b$ and $t$ the timestep in $1..T_{max}$. A bit mask \(M \in \{0, 1\}^{b \times T_{max}}, ~M_{x, t} = 1, ~t < l^B_x ~else ~0\) will be used to only extract the loss items lying withing the sequence lengths of demos. Finally, the loss per epoch will be normalised by the number of timesteps in the demos provided. \(loss = \frac{1}{\sum_{x, t}M_{x, t}}{\sum_{x = 1}^{b}\sum_{t = 1}^{T_{max}}M_{x, t} \mathcal{L}^{action}}\)

As before\todo{ref to earlier grasp}, the grasping loss is separated in the default configuration of this policy. Giving the final loss as: 
\[
  \mathcal{L}^{action} = \mathcal{L}^{pose} + \lambda_{grasp} \mathcal{L}^{grasp}
\]
where $\mathcal{L}^{pose}$ and $\mathcal{L}^{grasp}$ are calculated using the same masking method as above.

\missingfigure{updated diagram to convey the padding and unpacking and training the loss on every frame step}


\section{Moving to \emph{Active Vision}}
End-to-end models with absolutely zero fine-grained interactions are quite competent. However, in the chaotic and random environments an agent may be acting in. Planning and specific nudges may prove beneficial. Similar to the 

\todo{improve the grasping}

\todo{It might be worth scaling everthing to 128 by 128 next week and do a repeat run so that I can at least say that I have done it}

\todo{ruun grasp then move and confirm the suspicions of the policies with no memory failing.}



\chapter{Active Perception Policy Learning}\label{ch:appl}
In this section, by using what I have learnt so far, I will:
  \begin{itemize}
    \item Proposing some approaches to implementing active perception policies, mainly:
      \begin{enumerate}
        \item 3D Environmental reasoning policy
        \item View Uncertainty score policy
        % \item Human-like separate camera and gripper model for learning
      \end{enumerate}
    \item Discuss implementation details and ideas
  \end{itemize}

Contrary to the previous section, active vision entails moving the primary information source, as the robot control involves moving our robot (and by proxy its gripper) the cameras I will be mostly working with here will include some combination of the wrist mounted sensors, such as the RGB camera and the depth sensor.

% 1-
\section{First Approach: 3D Reasoning}\label{sec:appl-1}
A problem we have with RLBench demos is that the gripper does not understand to look at the target. Therefore, in an occluded scenario the system needs to actively search for poses to make the target most visible.

So, rather than changing the training of the agent, I want to optimise its movement after training on demos.Which is to say, a policy will be learnt as normal using BC. Then during its execution, the robot can use 3D reasoning and  planning to move to its optimal state before applying what its learnt from the demonstration, increasing the chances of the BC algorithm succeeding in completing the task.

\subsection{Proposed Approach}\label{subsec:appl-first-proposed}
What makes this system is the \emph{optimal pose predictor}, as we want the robot to place itself in a situation where the:
\begin{itemize}
  \item Object visibility is maximised
  \item Occlusions are minimised
\end{itemize}
all the while keeping the robot in a valid kinematic state as dictated by the physics in the simulator.


A simple system would follow the following decision process:
\missingfigure{Train -> evaluate visibility -> if POOR: runu viewpoint optimisation, move gripper there -> execute trained IL policy}

We can formalise this optimal pose system as follows:
\[
p^* = {argmax}_{p \in \mathcal{P}}
  \left[
    V \left( p; \mathcal{O} \right)
    - 
    \lambda C\left( p \right)
  \right]
\]

Where \(p^* \in \mathcal{P}\), is a (reachable) candidate gripper pose, represented in in \(\mathbb{R}^7\): \( \left[ x, ~y, ~z, ~qx, ~qy, ~qz, ~qw\right]\), as RLBench stores pose information as a list of 3D coordinates (\(x, ~y, ~z\)) and a quaternion (\(q = \langle x, ~y, ~z, ~w \rangle \)) to represent rotation. Which is a more compact way of representing rotations compared to $3\times3$ matrices.

$\mathcal{O}$ is the given observation. So, any  3D information about the target object and the scene respectively. They can be mesh data about the objects, or possibly point clouds. This approach will initially have to depend on prior knowledge about the target, then later I might be able to encode the target information to be learnt during training so that we can remove the dependence on providing priors.

$V$ is the visibility score for our target object. We need to define some scoring functions to asses the utility of our current pose within the scene.

And finally, $C$, is some sort of constraint or cost function to help shape the choices correctly, it might be a constraint to make sure candidate poses are reachable, or maybe a cost function determining whether a candidate pose is too far from the current configuration. $\lambda$ is just a tunable weight to adjust how much these limitations contribute to out optimal pose finding. To add, there maybe multiple cost functions so we can expand them as: \(\lambda_1 C_1 + \cdots +\lambda_n C_n \).


\subsection{Implementation}
Following the plan, the most important part to start with is our utility function, and a way of sampling poses. Then I can start devising constraints to help shape the policy.

\subsubsection{Scoring a Candidate Pose}
The very first step is to implement an utility function. The initial plan was to use a pretrained segmentation network. Such as MaskRCNN \todo{ref?}. The idea was to segment objects present in the scene and ideally get some bounding boxes of where the system thought the objects were. 

Initial trials proved this problematic. Because, these networks were mainly trained with real-life data and pictures the toy tasks that I created without proper graphics overlaying them, would commonly be misclassified or not be classified at all. 

Along with bounding boxes MaskRCNN also gives uncertainties as well as the label of what it thinks this item is. Which I initially thought I could use as a score for `seeing' the target. From the wrist camera it would almost always find the grippers (which are visible) but it was quite uncertain at marking my main target object, and the very rare times it did mark it it would not be certain what it was. From the many pre-trained labels it would guess between: `apple', `fire hydrant', and `traffic light'. This is unideal because in the fully autonomous training stage, the network does not necessarily understand about what its looking for, and it doesn't care what its looking for. This however made it hard to quantify the \emph{vision score} with the uncertainty because firstly it couldn't decide on what it was, meaning I couldn't extract just a label from the MaskRCNN. Secondly,  it would sometimes latch onto other things in the scene, such as the grippers I mentioned (mostly identified them as \todo{i cant remember check again?}). So I either had to label these by hand, which is inefficient, or find a way to have the network learn what its target is through an unsupervised (or at least semi-supervised manner). 

Training my own version also seemed infeasible. Both because I would need a lot of annotated data on my specific tasks, which though doable would be time consuming. However, the main issue was the time. Spending time to tune and create the perfect feature extractor would waste precious time for the project. Considering the goals of this project, I moved away from this.

This therefore lead me to pursue a simpler definition of utility. As before \todo{ ref cam attn} I was going to extract the colour. However, had plans of integrating the point cloud data and incorporate depth into this score.

Unlike before, I didn't use a simple distance colour metric, but used a colour range. This achieves the same effect of extracting the target colours. However, I tuned the HSV high and low specifically to be more accurate. 

Another addition is the use of the wrist point cloud. We can create a spatial mask from the colour range to extract the depth information that is specific to the target region. We can extract points $k$ from \( {pc}^{world}\) where \(m_k \in \{ \text{True, False} \}\) as \( \{ {pc}^{target} | m_k = \text{True} \} \) where $m$ comes from the aforementioned colour range extraction.

\subsubsection{Choosing a Candidate Pose}\label{sec:appl-first-choose-pose}
Simple idea was exploration. The main reason my \emph{non-active} policies struggled was because the system would refuse to point the gripper towards the target. So I wanted to sample some poses and check if a pose around my current pose would generate a better pose, graded by the utility.

So, I wanted to limit my sampling on the surface a unit sphere of radius $\mathbf{r}$ where the current position of the gripper $\mathbf{c}$ lies ont he surface of. Therefore, the sphere's centre, $mathbf{O}$ can be found as \(\mathbf{r} = || \mathbf{c} - \mathbf{O} || \).

Once we have that centre, we can sample \(\phi \sim U\left(0, 2\pi\right)\) abd \(u\sim U\left(0, 1\right)\) then set \(\cos \theta = 2u - 1 = \arccos\left(2u - 1\right)\). Using these values we create the unit vector \(\mathbf{u} = \left[ \sin\theta\cos\phi, ~\sin\theta\sin\phi, \cos\theta\right]\) and \(\mathbf{p} = \mathbf{c} + \mathbf{ru}\). Next, we pick the `facing' direction of our orientation, it made sense to always point towards the centre of the sphere. Once all these are set, we need to find the orthonormal basis to fix the roll around the  `facing' axis we have chosen.\todo{ref code, this is boring geometry} Once the full position and hte orientation is established, we can return this 3D matrix back into the pose and quaternion representation, so, we have a sample to test. We can use the \emph{inverse kinematics} of our robot to find out what configuration the joint angles need to be. And with a simple velocity controller, I arbitrarily chose a simple proportional control module system\todo{ref appendix}. We can move our robot to that location.

An important note, I arbitrarily chose the sphere centre to be right below (with a distance $\mathbf{r}$) the gripper position. The inverse kinematics system would frequently not be able to get to the joint configurations for many different radius configurations provided, so, for the final version of this system I adapted the pose sampling to only sample orientations. Rather than changing the position, it would change the orientation of the gripper with a random rotation, and then as before solve it with inverse kinematics. This meant that no position exploration would be happening but the agent would more robustly explore different orientations. 

I initially utilised two constraints in the pose selection, kinematic distance to the next pose, calculated by the absolute joint rotation differences. \(C_{jpos\_dist} = ||\mathbf{p}_{sampled} - \mathbf{p}_{current}||\). And secondly inverse kinematics solver. If it cannot be solved, \(C_{ik}\) blows up and that pose will not be considered. The ik solving discouraged movement because the score is calculated after moving. However, the distance could not immediately be added, because we need to move before getting a observation and scoring it. Therefore, I decided to take only the top $k$ of the samples from the sphere, encouraging only closer movements. And for the only-rotation approach, I just removed the distance restriction. This is because a simple rotation will be cheaper compared to full movement.

In later iterations, with a view estimator and other predictive scoring mechanisms, we can introduce the distance again, as then the entire score calculation can be done without requiring movement.

\subsubsection{Collison Avoidance}
A final heuristic I added was using action smoothing and back-off if we were getting too close to an object. As we now have access to the projected point cloud from the wrist. And removing the the target from this point cloud.\todo{usign the above calculations} By sampling for the closest $k$ points in the point cloud I estimate the centre of the closest obstacle. And set a repulsion direction away from it.
Then with some extra parameters: \textbf{clearance} and \textbf{blend} is used to calculate the distance thresholds and how to smoothly transition from full repulsion to the full action. As the wrist camera can see the tips of the grippers, I made sure to ignore these, they are about $0.895$ metres from the camera, so I checked distances above this and below the clearance to act on this information. \todo{see how its done in appendix}


\subsubsection{Action Loop}
There are two actions loops. I equipped the \emph{active agent} with one of the good policies from the earlier chapter. Essentially, there would be an \emph{active action} and a \emph{BC action}. The toggle between the two settings will depend on the utility score and an arbitrary limit to how long the agent can explore will be placed, so that it doesn't get stuck somewhere forever.

I initially started in active mode, which means that agent would continuously search at the start. This is a bad system because the agent is no where close to the target nor the obstacle. And with only the rotational exploration, it wouldn't be able to move. So the next heuristic was that, the active portion would only kick in once we were close enough to the obstacle, and the vision score was low. The closeness was not euclidean distance but a z-height range, so that slamming into the obstacle form above does not trigger the active mode, which again would get stuck. The final decision system is given here \todo{figure:}
  
% 2- (barely exists)
\section{Second Approach: View Emsembling}
\todo{cool name not to do with it yet, explain what this is meant to achieve, from proposal selection}

\subsection{Proposed Approach}
\todo{math foundations, add other subheading for limitations such as adding movement within rlbench demos??}

\subsection{Implementation}
\todo{need to make something here not sure what yet but I need at least one thing to compare the main thing}
% 3 - (does not exist)
% \section{Third Approach: Separate Cam and Hand Control}
\todo{explain what this is}
Initial idea from my supervisor was to train a policy -using reinforcement learning- to train a policy that directly controls the camera, for any task and object. Then when the main policy is trained using imitation learning on the specific task from the humuan demonstrations, both policies can be deployede to make sure the robot is optimally obsereving areas of the task that might be of interest.

Even though this was the main objective of the project from the get-go, without the previous ground work and the clearer understadning of comparably simpler policies, this would have been very hard to start working on. This is why 

\subsection{Proposed Approach}
\todo{propose some plan, maybe talk about placing a second movable camera in the workspace}


\subsection{Implementation}
Due to its complexity, and frankly, running out of the limited time we are allowed for this project, I was not able to iterate on this approach to the same quality as the ones above, and produce a system that works using this plan.

Hence, this will not be in the final comparison, but being the main motivator of the project, this is what this project will naturally evolve into, if development were to be furthered in this area.

\todo{even if i get a shitty implementation of this, which is not happening i dont think, I will just say not good enough to add to results etc}
  

\chapter{Evaluation}
Here I will evaluate:
\begin{enumerate}
  \item Findings from the FDC investigations
  \item Compare the proposed active vision approaches to non-active systems from before
\end{enumerate}\todo{split later if needed}

\section{Evaluation Setup}
The general setup included training a set of policies with their inherent hyperparameters, which will be discussed during their specific sections.

\subsection{Collected Data}
The data I chose to collect to reason about these policies are:
\begin{itemize}
  \item \textbf{Camera Type} A misnomer, indicates which sensors or combinations of sensors are being used in the policy to make decisions from \({RGB}_{ \{wrist, ~left\_shoulder, ~right\_shoulder \} }\) and \(Depth_{wrist}\).
  \item \textbf{Final Distance} (to target) all of my tasks are target centric, so final distance within the allowed episode length a measure of competence.
  \item \textbf{Minimum Distance} (to target) as above. The difference is reasoning where a policy might be falling apart, seeing the discrepancies between \emph{min} and \emph{final}
  \item \textbf{Success Rate} All the tasks had slightly differing success criteria, however this is an important measure for competency again. This was mostly measured as a count out of $n$, for $n$ being the length of the test demonstrations. The term ``test demonstrations'' means that a set of demos were recorded with the target and the environment in a fixed state. Loading these back allows us to hae a comparable test between policies and their variables.
  \item Other policy specific hyperparameters, will be discussed when relevant
\end{itemize}

All the data collection is done on a machine equipped with a RTX 3090 with 24GB of VRAM and an Intel 13700KF running Ubuntu 24.04.2 LTS using RLBench and PyRep version 4.1.0 run within a Python 3.11.12 virtual environment. The requirements can be found in the project repository (\ref{subsec:conc-materials}).

\subsection{Reproducibility and Verification}
All the random aspects that can be controlled, be it \emph{numpy} and \emph{pytorch} random choices or random `live' demos. Seeds are sets for both the libraries, as well as the \verb|DataLoader| seeds being fixed for the entirety of the policies tested here.
The demos per task are pre-made and saved in the project repository. \todo{ref the final deliverables bit}. I repeated all the tests with $5$ different seeds for the data shuffling, to observe if training on a different (but controlled) ordering of the demos affects the final policy.

Also, each test task has an associated `testing' and `training' demonstration set associated with it. Depending on its parameters these will be found in the project repository (see \ref{subsec:conc-materials}) under \verb|'src/3d/kup-bench/data/test/*demo(s)/_'| and \verb|'src/3d/kup-bench/data/*demo(s)/_'| where `\_' indicated the task name. All the evaluation carried out here uses corresponding task's `training' set (always the first 10 to keep training times manageable) then does the tests on the `test'. Tests with multiple configuration parameters will have clearly indicated folder names in this directory. RLBench is capable of loading these demos into the simulator with some of my library changes implemented in \verb|~/kup-bench/rlbench|. `Loading a demo' means the objects are initialised exactly as they are when the demo was first created, and does not mean the trajectories will be used. I used this as a method of exactly repeating environment configurations.

\section{Testing Task Environments}
Two main branches of tasks I followed in this project are grasping and occluded reaching tasks.

\subsection{Reaching with an Obstacle - \textbf{ReachObs\_Random}}\todo[color=red]{write something here}

\textbf{ReachObs\_IndRandom} is not discussed here in depth, due to the similarity of the tasks, and the independent version not being any more interesting. Tests were run on this and any interesting ones are included here \todo{add appendix if needed}

\subsection{Grasping}
These two grasping tasks are to assess the contextual 3D understanding of the policy: learning depth information and adapting to differing sizes of targets; as well as learning the workspace before attempting a task \todo{grasp then move, not sure if I have time for this, but If i do defo interesting}

\subsection{Grasp and Depth Understanding - \textbf{Vision\_Random}}
This task is quite similar to the depth interfacing tests from earlier.\todo{ref DI}. The main difference is that the placement of the targets are random within he workspace, for \emph{training}, \emph{control} and \emph{test} sets. The test is repeated with different sizes of targets, \textbf{normal} and \textbf{smaller}. \textbf{normal} sets the scale of the target to $1$ and \textbf{smaller} sets it to $0.5$. The training and the control sets are the same size while the test will be the the other. The reasons for this is to assess:
\begin{enumerate}
  \item The effectiveness of the configuration in solving the task it is immediately trained for
  \item Assess the information extraction from the various views by comparing the discrepancy between control and test.
\end{enumerate}
I created two separate variations of this specific task. One where the training and control set is set to be the \textbf{normal} size and we evaluate that on \textbf{smaller} blocks. The next variation is flipping this configuration: control and training are scaled down, while test is normal sized. Mainly to investigate whether the information wew train on has any affect on the policy proficiency.

\subsubsection{Important Note!}
The demos saved for all the configurations are tested and checked to appear correctly. However, I have realised that the specifically the \textbf{saved smaller testing demos} which is used in the place of the control or the test, would spawn three of the targets smaller than they should be. Indices  \(0, ~5, ~6\)  will spawn smaller and hence look smaller. I will remove these when evaluating the policies.

I have not figured out the reasoning for this, it seems to randomly happen, I speculate that it is an issue with how I control the sizes of targets. The target size is set during the start of an episode, so this means the \verb|Environment| object from PyRep must have initialised it wrong when the demos were being saved. This is not an isolated incident and sometimes happens while resetting the environment. However, the ideal part is if it is encoded in the saved demos, it is at least controlled. The \textbf{smaller} training set is confirmed to spawn correctly. The evaluation of the results will consider this.

\subsection{Carrying the Target - \textbf{Grasp\_ThenMove}}\todo{havent run this yet, might remove it depending on what I will do today}

\section{Parameters}

\subsection{Agent Parameters}
Along with a policy, there is a high level \verb|Agent| class that manages the data ingestion and other high-level control. There are 3 main configurations it can be in:
\begin{itemize}
  \item \verb|is_grasp|, meaning the grasping head is separated from the pose and action predictor.
  \item \verb|is_proprio|, allowing proprioceptive data will be included in the pipeline
\end{itemize}
Each of these also have a corresponding \verb|_opts| field, so modules can be fine controlled further down the policy network. They were kept as defaults I landed on during experiments, and will be disclosed when where necessary. Each agent can also be of \emph{RNN} flavour meaning, the final state encoding will be passed through a LSTM network for temporal reasoning.

For the grasping tasks the agents were configured to use \verb|is_grasp| and \verb|is_prorio| only used during proprioception results, and not anytime before.

\subsection{Policy Parameters}
Most policies were trained using the configuration in Table \ref{tab:eval-training-params}. Any other setting that was enabled will be discussed in their respective sections. The \emph{italics} are variables that are inserted during  the test. The \emph{SEED} was locked to \( \left[ 3790, ~1901, ~4248, ~6689, ~7653 \right] \) which was a randomly generated set. The \emph{Epochs} was a variable range depending on the test and policy type.

\begin{table}[ht]
\centering
  \begin{tabular}{|| c | c ||}
  \hline
  Parameter & Value \\
  \hline
  minibatch\_size & 10 \\
  lr & $1\times 10^{-3}$ \\
  shuffle\_data & \texttt{True} \\
  dataset\_to\_use & \texttt{`demo'} \\
  epochs & \emph{Epochs} \\
  lock\_loader\_seed & \emph{SEED} \\
  \hline
  \end{tabular}\caption{Default Training parameters}\label{tab:eval-training-params}
\end{table}

\section{Results}\todo{remove}



\section{Naive Colour Attention Results}
This is the results relating to the proposed naively active approaches from \ref{sec:reach-obs-naive-cam-attn}. The numbers here are from the reaching test (\ref{eval-setup-reach-obs}).
The variations of this policy I tested included:
\begin{itemize}
  \itemsep0em
  \item Separate or joint Feature Encoders (\verb|is_multi_cnn|)
  \item \emph{mean} or \emph{max} pooling for the colour score (\verb||)
  \item $\lambda_{attn}$ loss parameter that scales the KL divergence of the predicted 
  weights.
\end{itemize}
These were tested over multiple RGB camera combinations, namely:
\begin{enumerate}
  \itemsep0em
  \item Wrist, Left Shoulder and Right Shoulder
  \item Wrist and Right Shoulder
  \item Wrist and Left Shoulder
  \item Left and Right Shoulder
\end{enumerate}
No `single' camera was tested, as the idea of this policy is to investigate the interaction between cameras and their features. They were tested for increasing epochs in \(\left[100, ~200, ~500, ~1000, ~2000\right]\) to observe longer training trends. Finally, I collected the attention weights for the different cameras during the episodes and separated them between when they happen in the episode. This is to investigate whether the weights change from above or below the obstacle.


\subsection{Observations}
The main takeaway from this task was how it favoured longer epochs of training. The most success and lowest distances to target were recoded in the $500$ to $1000$ range, then the success rate starts decreasing again, meaning overtraining and no more benefit.\todo[color=purple]{} The very likely reason for this is that, the demonstration trajectories for this task are a lot more varied due to the obstacle. More importantly the choice to move around the obstacle, there are demos that force the arm to go to the left go back and down, or left then down. This high variation means the agent needs longer time to average out motions in strict Behavioural Cloning settings, and takes longer.


\begin{figure}[htpb]
  \centering
  \begin{subfigure}{\linewidth}
    \centering
    \includegraphics[width=0.6\linewidth]{assets/evaluation/cam-attn/ro_random-cam_attn-max-mindist.png}
    \caption{Minimum Distance (`max' pool)}\label{subfig:cam-attn-max-mindist-cnn}
  \end{subfigure}

  \begin{subfigure}{\linewidth}
    \centering
    \includegraphics[width=0.6\linewidth]{assets/evaluation/cam-attn/ro_random-cam_attn-mean-mindist.png}
    \caption{Mean}\label{subfig:cam-attn-mean-mindist-cnn}
  \end{subfigure}
  \caption{Minimum distances reached per pooling method}\label{fig:cam-attn-mindist-cnn}
\end{figure}

\begin{figure}[htpb]
  \centering
  \begin{subfigure}{0.45\linewidth}
    \centering
    \includegraphics[width=0.9\linewidth]{assets/evaluation/cam-attn/ro_random-cam_attn-max-success.png}
    \caption{Minimum Distance (`max' pool)}\label{subfig:cam-attn-max-success-cnn}
  \end{subfigure}
  \hfill
  \begin{subfigure}{0.45\linewidth}
    \centering
    \includegraphics[width=0.9\linewidth]{assets/evaluation/cam-attn/ro_random-cam_attn-mean-success.png}
    \caption{Minimum Distance (`mean' pooling)}\label{subfig:cam-attn-mean-success-cnn}
  \end{subfigure}
  \caption{Minimum distances reached per pooling method}\label{fig:cam-attn-success-cnn}
\end{figure}

\subsection{Separate Feature Encoders}
This parameter was more of a test to see if it interacted well in any way. Theoretically both approaches are sound. 

This task is quite unforgiving and errors add up quickly, if the robot gets stuck on the obstacle, it rarely manages to save itself. Evident from the quick increases in minimum distance between epochs $100$ and $200$ for single-CNNs, which capture early fitting due to information abundance, but get confused trying to generalise due to the feature encodings keeping a lot of this uncertainty.

Conversely, multi-CNNs learn late. Makes sense as more passes allow each CNN (now seeing only one frame not all) to learn slowly. Therefore, the multi-CNN policies have a more decisive dip in minimum distance reinforcing the idea that training for longer allows them to capture pose specific information that is helpful to the system in solving this task. Multi-CNN interacts interestingly with \emph{mean} pooling, see Figure \ref{subfig:cam-attn-mean-mindist-cnn}. Seeing the spiky and jagged Single-CNN graphs we can confirm that merging all views and convoluting them likely confuses the feature extraction and a lot of uncertainty creeps into the features, especially because the view poses are so different.


\subsection{\emph{Mean} and \emph{Max} Pooling}
The differences here are extremely subtle. Both work fairly similarly in terms of the minimum distance they can reach per epoch, excluding `max' and the shoulder cameras\todo{talk about the attention weights about this}. A stark difference, however, is how jagged and uncertain \emph{max} is with lower epochs. This is due to max likely sending a similar signal between all views, if more than one view can see the target, they will likely have very similar weights. Which can add indecisiveness, especially when both shoulders are active.

\emph{Mean} is not necessarily any better when single-CNNs are used. Again, due to uncertainty creeping in, but from the view understanding.

Though, its performance immediately starts good when it is paired with multi-CNN feature encoder. This makes the data less spread out (\ref{subfig:cam-attn-mean-mindist-cnn}) and the policy being capable of moving past the target, even from the start. However, the drawback is, the performance does not get much better for shoulder cameras as training goes on. This is now likely due to how \emph{mean} operates, it relates the score magnitude to depth essentially, more area covered the better. The shoulder cams being at a fixed depth, don't allow them to participate in this benefit. So \emph{max} with Multi-CNN combination is generally more successful than \emph{mean} with combinations using shoulder cams.

\subsection{Attention Loss Weights ($\lambda_{attn}$)}
This didn't seem to make a change at all. However, looking at the loss curves \todo{add a loss curve here, fighting for vram currently}, some potential issues may be that the one loss term is drowning the other one out and scaling it up might not matter, rerun depending on this.

\subsection{Remarks}
I thought this policy would be able to fix the shortcomings of the naive policies by selectively attending on a view, even if the views are static. These shortcomings included the final reach toward the obstacle not being executed well. Non-wrist cameras help with the first half of the motion while the wrist camera should be able to generalise to the final reaching motion. However the problem is the demos don't force the gripper to `look' towards the target. So, the wrist camera cannot get the benefits from a \emph{mean} pooled configuration, which will include intrinsic depth data however, the agent does not learn to face the target. This led to the attention weights heavily weighting a single view during rollout. Which counteracts this policy's theoretical benefits as attention will be mainly focused on a single target.



\section{ResNet}
I decided with various sizes of ResNet systems\todo[color=green]{reference}. Essentially replacing my feature extraction block with modified versions of \emph{resnet18, resnet34, and resnet50} \todo[color=green]{maybe not mention all, afterall the large ones will be ass when the image is so small}.

added a lot to training time while not improving the results in any way i decided to leave the resnets for my own feature extractors. Maybe I should have put some resiudual connections??

A reason these might not have worked well is because of one of my earlier constraints. The image sizes being \(64 \times 64\) pixels, might not be enough to extract meaningful information with a ResNet. This is likely due to its aggressive pooling between the layers and especially during the residual connection and the aggressiveness only increases in larger models. \todo[color=red]{not actually sure if this is right, fact check, wrote this some time ago} 



\section{Naive Colour Attention - Results}
\section{Recurrent Models - Results}
Not much better in terms of numbers however the model size is lower?? not sure experiment with this the big ass 4 way attention is the most expensive currently and this might be a contender,

talk about the shorter training to get comparable results, smoother trajectories learn occlusions robustly (?? i have not observed this i don't think)


This lead to a more robust and confident architecture. The movements of my policy felt less spiky, no more sudden jerks and more calculated reaching. It did still absolutely overfit to the average position, and was not much better at grasping, but the episodic movements were cleaner and I believe enhancing the grasping branch of the network would allow it to be more successful \todo{what the fuck am i talking about, summarise and clean up}
\chapter{Conclusion}
I will conclude the report here with:
\begin{itemize}
  \item Final thoughts
  \item Declarations
  \item Future work
\end{itemize}

\section{Concluding Remarks}
In this work I personally explored the steep learning curve of getting into Robot Learning (Chapter \ref{ch:early-work}). Depending heavily on 3D simulation frameworks, such as CoppeliaSim and Bullet, quickly made me realise that prior information is not only useful for an active vision task; but also crucial for a project of this scale. Here I pivoted the main goals of the project to align it with learning personally alongside my agent.

I proposed simple, yet capable behaviourally cloned policies for the tasks I created, then iterated on them with various fusion policies to find the sweet spot of what a robot can extract from a scene, with limited visibility, both in terms of information: what fits in the scene; and in terms of resolution of what fits in the sensor. I have uncovered some surprising results, regarding when data dilution and extraneous features play a huge role in shaping a policy and how small low-level changes in a modality can make or break the proficiency of a system.

Finally, in Chapter \ref{ch:appl}, by using information from my findings about feature extraction workspace understanding of my robot, I proposed two separately active learning systems. First,depending heavily on information priors, such as task geometry and colours. This system was tougher to manage, it required extreme fine tuning  per-task and without the information it could use to effectively plan, it was quite helpless. My second proposition, however promising, was unfortunately not materialised. This was mainly due to the constraints I was working with within the simulator architecture. Although, not closed down, significant changes that this proposition required were not feasible in the timeline of this project along with all th prior work. Therefore, I leave the execution and evaluation of this system as an exercise to the reader.

In conclusion, although I may not have hugely interesting results to share, I learnt a lot about three dimensional data management and modelling as well as robotics in general.

\section{Declarations}
  \subsection{Sustainability and Ethical Considerations}
  Although maybe not immediately related to the outcome or the goals of the project, I have used a lot of GPU resources which in turn uses a lot of energy. Although, my personal contribution might be insignificant; I sincerely believe that nature is sacred and it is our moral obligation to understand the value in it and protect it. With advancements in Machine Learning and Artificial Intelligence, reliance on raw compute power and hence data centres is growing rapidly. Much of which are still using fossil fuels. I think the environmental impact of training models should be heavily considered. Again, I do not believe the dent I might have caused is relevant, yet no act is insignificant to nature. So, to say, I believe any research in the AI community should incorporate their long term environmental impacts as well as other ethical concerns, alongside innovation and performance.

  \subsection{Use of Generative AI}
  Generative AI is an important tool in modern times and is a tool that should be mastered for efficient working. 
  I declare that I have used Generative AI tools in this project in line with Imperial College's guidance and principles. Generative AI has not in any way been involved in the intellectual material produced in this project such as the report or the code. However, it was used in areas such as:
  \begin{itemize}
    \item Creating appealing and clear plots using \verb|matplotlib| and \verb|seaborn| \todo{ref these?} using data I have collected.
    \item Formatting the bibliography and helping generate BiBTeX style bibliographies where it wasn't specifically provided by an author; as well as formatting these references.
    \item General help with LaTeX figure and section formatting, with occasional questions about compilation errors.
  \end{itemize}

  \subsection{Project Materials}
  The materials on this project are available here:
  \todo[color=red]{Update the readme of the repo to explain things}
  \begin{itemize}
    \item Codebase: \todo{insert link to public repo}
    \item Report: \todo{insert link to public repo}
  \end{itemize}

\section{Future Work}
This project started with the vision of creating an active vision policy with an articulating arm mounted camera separate from the arm what was to be used for grasping and reaching. However, during the timeline of the project I identified that, to get to that level of complexity I had to understand the basics and pivoted to a more experimental side of understanding how 3D simulators, complex machine learning models, and most importantly robot understanding works with those models.

So, a good continuation this project paves the road to, is expanding in the area of \emph{full-active} vision, which I will coin to mean active perception separated from the action needed to complete the task. This should create more human-like capabilities in solving tasks and allow humans to better teach such agents how to act.

\bibliographystyle{unsrt}
\bibliography{bibs/intro, bibs/background, bibs/rel-work, bibs/general}

\appendix

\chapter{Learning Robot Learning}
\begin{figure}[h]
  \centering
  \includegraphics[width=0.5\textwidth]{assets/early-work/cnn-diagram.png}
  \caption{One of the best models for the agent reaching task}\label{fig:cnn-5050}
\end{figure}


\begin{figure}[htbp]
  \centering
  \includegraphics[width=0.5\textwidth]{assets/early-work/regions.png}
  \caption{Illustration of phases around the target, 0 would be green, 1 would be blue and 2 would be the rest of the canvas in this example. }\label{fig:phase-regions}
\end{figure}


\begin{figure}[htbp]
  \centering
  \begin{subfigure}{0.45\textwidth}
      \centering
      \includegraphics[width=0.6\linewidth]{assets/early-work/obs-gen1.png}
      \caption{Example 1}
  \end{subfigure}%
  \hfill
  \begin{subfigure}{0.45\textwidth}
      \centering
      \includegraphics[width=0.6\linewidth]{assets/early-work/obs-gen2.png}
      \caption{Example 2}
  \end{subfigure}

  \vspace{0.5cm}

  \begin{subfigure}{0.45\textwidth}
      \centering
      \includegraphics[width=0.6\linewidth]{assets/early-work/obs-gen3.png}
      \caption{Example 3}
  \end{subfigure}%
  \hfill
  \begin{subfigure}{0.45\textwidth}
      \centering
      \includegraphics[width=0.6\linewidth]{assets/early-work/obs-gen4.png}
      \caption{Example 4}
  \end{subfigure}
  \caption{The 2 by 2 grid of some example obstacle generations}\label{fig:obs-gen}
\end{figure}




\begin{figure}[htpb] % htpb allows all placement
  \centering
  \includegraphics[scale=0.3]{assets/early-work/missing-libs.png}
  \caption{}\label{fig:missing-libs}
\end{figure}

\begin{listing}[H]
  \begin{minted}[fontsize=\small, bgcolor=gray!10, linenos]{python}

  obs_config = ObservationConfig().set_all(True) 
  enabled_config = CameraConfig(
    rgb=True, depth=False, mask=False, point_cloud=False, image_size=(64, 64)
    render_mode=RenderMode.OPENGL,
  )
  disabled_config = CameraConfig(
    rgb=False, depth=False, mask=False, render_mode=RenderMode.OPENGL)

  obs_config.wrist_camera = enabled_config ## example: enabling a cam/sensor
  obs_config.front_camera = disabled_config

  env = Environment(
    action_mode = MoveArmThenGripper(
      arm_action_mode=JointVelocity(), 
      gripper_action_mode=Discrete()
    ),
    dataset_root = '' if live_demos else 'PATH/TO/YOUR/DATASET',
    obs_config = obs_config, headless = False
  )
  env.launch() ## start the simulator
  \end{minted}
  \caption{Standardised environment launching}\label{lst:env-setup}
\end{listing}

\chapter{View and Feature Combinations}
\begin{figure}[h]
  \centering
  \includegraphics[width=0.6\textwidth]{assets/early-work/cnn-encoder-policy-head.png}
  \caption{Simple Policy Network Architecture}\label{fig:policy-arch}
\end{figure}

\begin{figure}[htpb]
  \centering
  \includegraphics[width=0.7\linewidth]{assets/cam-comb/policies/general-diagram.png}
  \caption{Main Skeleton of the proposed policy structure}\label{fig:policies-skeleton-idea}
\end{figure}


\chapter{First Appendix RENAME}
% \section{General Framework Contributions}


\section{Reach with No Obstacles}

\section{Grasping}

\begin{figure}[H] 
  \centering
  \includegraphics[scale=0.6]{assets/cam-comb/grasp-simple/Z-tuning-normal-old-policy.png}
  \caption{Final Distance for Current Policy (less linear layers in MLP gripper head)}
\end{figure}

\begin{figure}[H] 
  \centering
  \includegraphics[scale=0.6]{assets/cam-comb/grasp-simple/Z-tuning-normal-old-policy-success.png}
  \caption{Success Count for Current Policy (less linear layers in MLP gripper head)}
\end{figure}

\begin{figure}[H] 
  \centering
  \includegraphics[scale=0.6]{assets/cam-comb/grasp-simple/Z-tuning-normal-old-policy-lambda-dist-hist-hue-cams.png}
  \caption{}
\end{figure}


\begin{figure}[H] 
  \centering
  \includegraphics[scale=0.5]{assets/cam-comb/grasp-simple/Z-tuning-normal-old-policy-lambda-dist-hist.png}
  \caption{}
\end{figure}

\begin{figure}[H] 
  \centering
  \includegraphics[scale=0.6]{assets/cam-comb/grasp-simple/Z-tuning-normal-old-policy-retry-success.png}
  \caption{}
\end{figure}

\begin{figure}[H]
  \centering
  \includegraphics[scale=0.6]{assets/cam-comb/grasp-simple/Z-tuning-normal-old-policy-retry-dist-hist.png}
  \caption{hist}
\end{figure}

\begin{figure}[H]
  \centering
  \includegraphics[scale=0.6]{assets/cam-comb/grasp-simple/Z-tuning-normal-old-policy-retry-dist-kde.png}
  \caption{KDE}
\end{figure}\todo{maybe remmove, not sure what this really is saying}



\section{Reach with Obstacles}

\begin{figure}[H]
  \centering
  \includegraphics[scale=0.6]{assets/cam-comb/reach-obs/Z-ro_random-Final-dist.png}
  \caption{Final Distances}
\end{figure}

\begin{figure}[H]
  \centering
  \includegraphics[scale=0.6]{assets/cam-comb/reach-obs/Z-ro_random-Minimum-dist.png}
  \caption{Minimum Distances}
\end{figure}

\begin{figure}[H]
  \centering
  \includegraphics[scale=0.6]{assets/cam-comb/reach-obs/Z-ro_random-success.png}
  \caption{Success Rate}
\end{figure}

\begin{figure}[htpb]
  \centering
  \includegraphics[scale=0.6]{assets/cam-comb/reach-obs/Z-ro_random-obs-dist.png}
  \caption{Reach `Random' and `IndRandom' Final Distannce}
\end{figure}

\begin{figure}[htpb]
  \centering
  \includegraphics[scale=0.6]{assets/cam-comb/reach-obs/Z-ro_random-obs-mindist.png}
  \caption{Reach `Random' and `IndRandom'}
\end{figure}

\begin{figure}[htpb]
  \centering
  \includegraphics[scale=0.6]{assets/cam-comb/reach-obs/Z-ro_random-obs-trials-success.png}
  \caption{Running \emph{obs dataset} with different batch sizes}\label{apx:Z-ro_random-obs-trials-success}
\end{figure}


\end{document}