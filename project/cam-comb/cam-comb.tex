\chapter{Feature Combination and Dependence Investigations}
This chapter I will be:
\begin{itemize}
  \item Creating more complicated toy tasks in RLBench
  \item Propose various imitation learning agents to solve these tasks
  \item Evaluate said methods by:
  \begin{enumerate}
    \item Providing different views and features
    \item Augmenting the fusion of these features
    \item Understand how the agent `views' a scene and what information it benefits from for which tasks
    \item Compare the strengths and weaknesses of these methods
  \end{enumerate}
  \item Use learnings to guide proposed active policies in the next chapter \todo{maybe remove this}
  
  \todo{there are commented notes from earlier here, check the cam-combs tex file}
  % \item a good vision score policy without intrinsic information about the simulator.
  %   \item simple colour checking with a threshold?? -> shit but all that I could get working 
  %   \item  reay projectrion and checking if it lands on target
  %     \item  prior info about target?? -> using CAD models I can do RBG(-D) pose estimation
  %     \item extract prior information form demonstration?
  % \item segmentation masks to pull bits out
  %   \item  How do we know which segmentation mask is my target?
\end{itemize}

\todo{in the vision experiments section add the condlusion from `plots.ipynb'  talk about why wrist is not generalising and the possible fixes by adding depth, }


\todo{move the depth interfacing here}
% Reaching tasks
\section{Experimenting with Reaching}

\subsection{Creating a Policy}
The next step was to create a policy network. As I am mainly working on imitation learning -and specifically behavioural cloning- from demonstrations provided by the system, my network needs to be able to ingest these demonstrations and generate actions in the action space of the robot. Following the earlier parameters of a demo, I created the following network, Figure \ref{fig:policy-arch}. This takes in the given wrist rgb image and extracts features, which are then interpreted into a $8$ dim vector of \textbf{float}s as an action. First $7$ corresponding to the $7$ joints of the \emph{Panda} robot, and holding a velocity value for them, while the final value is the gripper state, $0$ meaning closed and $1$ meaning open, which is clamped by the movement system under the hood.

\begin{figure}[h]
  \centering
  \includegraphics[width=0.6\textwidth]{assets/early-work/cnn-encoder-policy-head.png}
  \caption{Simple Policy Network Architecture}\label{fig:policy-arch}
\end{figure}\todo[color=blue]{ius the quality bad? reuload or reexport the xmml is in assets}

Along with the policy the second most important of any ML workload is the way the data us regularised, processed, and loaded into the system for training or testing.

\subsubsection{Data Processing}\todo{talk about rgb transforms? uniforming etc, or remove not sure}

\subsubsection{Data Loading}
I followed a simple flattening approach for loading the data into the system. Using PyTorch's \textbf{Dataset} and \textbf{DataLoader} classes I created a dataset that can take in a raw list of demonstration, then flattens its observations into a tensor of shape \(\langle 3,~64,~64 \rangle \) (permuted from the usual \(\langle 64,~64,~3 \rangle \) for images due to Torch conventions of convolutional networks and where they expect the channel  dimension). Then the dataset makes individual observation indexable along with their corresponding action labels. Types given as:\mintinline{python}|DemoObsDataset: tensor[tensor[3,  64, 64], tensor[8]]|. Then the loader can manage the shuffling and batching as usual. Initially I kept the data unshuffled, to keep the data in its sequential form. While keeping the batch size as the demo length. This is because currently I am trying to overfit the network to the single demonstration given to it to gauge how long to train my networks for

\subsection{Initial Observations}
Starting with the 3 static versions of the task, where the target is placed as shown in \ref{fig:no-obs-3-views} seen from the wrist cameras. I wanted to get an idea of how to tune the policy parameters. While understanding the relationship between training length and varying observability of the target.

\begin{figure}[htbp]
  \begin{subfigure}{0.3\linewidth}
    \centering
    \includegraphics[scale=0.4]{../fyp/assets/demo-trials-no_obs/tasks/static-tasks-camera/initial-obs-side_l.png}      
    \caption{Left Side}
  \end{subfigure}
  \hfill
  \begin{subfigure}{0.3\textwidth}
    \centering
    \includegraphics[scale=0.4]{../fyp/assets/demo-trials-no_obs/tasks/static-tasks-camera/initial-obs-central.png}
    \caption{Central}
  \end{subfigure}
  \hfill
  \begin{subfigure}{0.3\linewidth}
    \centering
    \includegraphics[scale=0.4]{../fyp/assets/demo-trials-no_obs/tasks/static-tasks-camera/initial-obs-side_r.png}
    \caption{Right Side}
  \end{subfigure}%
  \caption{Three variations of the reach task with the obstacles sometimes out of view}\label{fig:no-obs-3-views}
\end{figure}

I tested multiple epochs of training with the simple policy and recorded the final distance to the target at the end of their episode. The episode length is determined by the demo lengths, which I defaulted to the maximum, will also try mean, as keeping a static episode length doesn't make sense (especially on later tasks where the demonstration episodes can drastically vary in length)

The success of the tasks are wired to reaching the target in the simulator and will send a \emph{DONE} signal if it is reached. This happens around $0.12$ metres to the target. I have also observed the target will reach a very close distance but it won't trigger the detection in Coppelia, I think this must be a bounding box issue, wither the dummy objects that are doing the collision detection are missing each other, or the polling rate in the simulator is not frequent enough to detect this change. Either way, I added a way to count closeness into success if it were close enough.

\subsubsection{Static Tasks}
Testing on the static versions of the task, a simple policy with training around $20$ to $100$ epochs seems to do the job well, see Figure \ref{fig:rno-static}. And this is mostly because without variation in position simple Behavioural Cloning can be employed by overfitting to the data given. 

As I trained for longer it seemed to overfit early move really slowly at the start of the episode, wasting steps and ending up far from the target. Curious observation is \todo[color=purple]{} the central tasks behaves better at higher epochs which I believe is not necessarily because of visibility (the \emph{conv} features are not necessarily guiding anything with $1$ demo) but rather the centrality, as it is right above the gripper a simple downward bias allows the arm to easily get close to the target.

\begin{figure}[htpb] % htpb allows all placement
  \centering
  \includegraphics[scale=0.5]{assets/cam-comb/reach-no-obs/rno_static.png}
  \caption{Epoch experiments with the static tasks using a single demonstration}\label{fig:rno-static}
\end{figure}

\subsubsection{Placing Randomly}
To test generalisability, I created a dynamic version of the task, where the target is now randomly placed within view (not necessarily always fully within view, but at least some parts are visible). This is achieved by using a \verb|SpawnBoundary| and randomly sampling the location of the target withing this for every new variation of the task, or for every new episode. Which can be seen in Figure \ref{fig:reach-no-obs}, the white dotted box being the boundary. This boundary is not rendered in the simulation visually. This guarantees variety in demonstrations as well as helps us create a generalisable policy.

\begin{figure}[htpb] % htpb allows all placement
  \begin{subfigure}{0.50\linewidth}
    \centering
    \includegraphics[width=\linewidth]{assets/cam-comb/reach-no-obs/rno_random-dist.png}
    \caption{Average Final Distance to Target}\label{subfig:rno-random-dist}
  \end{subfigure}
  \hfill
  \begin{subfigure}{0.50\linewidth}
    \centering
    \includegraphics[width=\linewidth]{assets/cam-comb/reach-no-obs/rno_random-success.png}
    \caption{Success Rate (\%) for the $10$ Test Demos}\label{subfig:rno-random-success}
  \end{subfigure}
  \caption{Experiments with randomly placed target}\label{fig:rno-random}
\end{figure}

To run the random tests, shown in Figure \ref{fig:rno-random}, in a comparable manner I reused my set of demos that were created and saved earlier for this task for training. Then a set of 10 demos were randomly generated at the start and after the agent with the specific parameters were trained, I evaluated these policies against the test counterparts.

Looking at graph \ref{subfig:rno-random-dist}, we can see that providing more demonstrations helps the policy generalise better to random locations, where the sweet spots seems to be around 10 demos and around $500$ where the success rate (\ref{subfig:rno-random-success}) is quite high

\subsection{Camera Limitations}\todo[color=red]{reread and make sure it is a good starter segue to obs (maybe move these to after obs?), mention this under reachObs! MOVE}
I started this section by limiting the learning to only the wrist mounted camera, which works well for this specific unobscured task. Introducing some of the other RGB cameras, specifically the \verb|left shoulder| or the \verb|right shoulder| views, does not necessarily benefit the performance, see \ref{fig:rno-random-cams}. Conversely, we are increasing the training time by adding more channels to the convolutional layers, which can be considered a drawback.

\begin{figure}[htpb] % htpb allows all placement
  \begin{subfigure}{0.50\linewidth}
    \centering
    \includegraphics[width=\linewidth]{assets/cam-comb/reach-no-obs/rno_random-cams.png}
    \caption{Average Final Distance to Target}\label{subfig:rno-random-cams-dist}
  \end{subfigure}
  \hfill
  \begin{subfigure}{0.50\linewidth}
    \centering
    \includegraphics[width=\linewidth]{assets/cam-comb/reach-no-obs/rno_random-cam_success.png}
    \caption{Success Rate (\%) of the $10$ Demos}\label{subfig:rno-random-cams-success}
  \end{subfigure}
  \caption{Experimenting with multiple RGB cameras}\label{fig:rno-random-cams}
\end{figure}

Although the performance does not outright increase, we can clearly see that having a different point of view can sometimes drastically affect the quality of the learnt network and hence the action produced. \todo[color=purple]{}  \todo{need to actually talk about the numbers here refer to graph}

However, this is not to say all tasks will be immune to benefits from extra views. An important part of this project is to understand what is most important for a robot to observe in its given environment and task and how can it optimally leverage this data to solve a task better. And increasingly more complex tasks should benefit from abundance of information.

\subsection{Increasing the Toy Task Complexity}\todo{subsection to above?}
Complexity of tasks can be increased in two main ways:
\begin{enumerate}
  \item Introducing non-linearities to the environment to make a scene more challenging to traverse for an agent
  \item Increase the movement or the level of interaction of the task at hand
\end{enumerate}
I plan to do these mainly by introducing obstacles; which will guide me to understand what an agent needs to understand navigation. Secondly, I want to branch out to a grasping task, to increase the number of items of execution, to evaluate the capability of an agent to use its understanding to complete increasingly more complicated tasks.

\section{Reaching with an Obstacle}
To add onto the reaching task, I introduced an obstacle placing mechanism as well as randomly placing the target behind this obstacle, ensuring the agent doesn't learn where exactly target is by looking at just the obstacle\todo[color=purple]. See Figure \ref{fig:reach-obs-random} for how this task looks and the check \todo[color=green]{add appendix link, and code} for the backend wiring of the task.

\subsection{Creating the Task}
There are two versions of this task, I thought it might be interesting to randomise the object firstly dependently then independently on the obstacle. The `dependent' randomisation called \verb|ReachObs_Random| samples the obstacle, which in turn controls the spawn boundary of where the target can spawn in, meaning the target will always appear behind the obstacle albeit, edges of it can sometimes stick out. Conversely, the `independently' random version, called \verb|ReachObs_IndRandom|\todo{also add to appendix and link here}, keeps the target spawn boundary fixed, meaning the target can be anywhere in the visible workspace, but it is not necessarily always covered by the obstacle. I can see that this potentially can be useful to keep the dataset a bit more diverse, and allow the wrist camera initially observe the target sometimes.\todo[color=red]{use this somewhere, or hint back to it, maybe even to say there was no difference}

\begin{figure}[htpb] % htpb allows all placement
  \centering
  \begin{subfigure}{0.3\linewidth}
    \centering
    \includegraphics[scale=0.3]{../fyp/assets/task-pics/reach-obs/random-front.png}
    \caption{Front}
  \end{subfigure}
  \hfill
  \begin{subfigure}{0.3\linewidth}
    \centering
    \includegraphics[scale=0.3]{../fyp/assets/task-pics/reach-obs/random-side.png}
    \caption{Side (Left)}
  \end{subfigure}
  \hfill
  \begin{subfigure}{0.3\linewidth}
    \centering
    \includegraphics[scale=0.3]{../fyp/assets/task-pics/reach-obs/random-top.png}
    \caption{Top}
  \end{subfigure}
  \vfill
  \begin{subfigure}{0.45\linewidth}
    \centering
    \includegraphics[scale=0.5]{assets/early-work/obs-random-scene-hierarchy.png}
    \caption{`ReachObs\_Random' Scene Hierarchy}
  \end{subfigure}
  \hfill
  \begin{subfigure}{0.45\linewidth}
    \centering
    \includegraphics[scale=0.5]{assets/early-work/obs-ind-random-scene-hierarchy.png}
    \caption{`ReachObs\_IndRandom' Scene Hierarchy}
  \end{subfigure}
  \caption{Reaching Task with an Obstacle}\label{fig:reach-obs-random}
\end{figure}\todo[color=blue]{smaller?} 

\subsection{Experimenting with Views}\todo[color=red]{}

\subsubsection{Using same unchanged policy form ReachNoObs}
\todo{run the best params we have for the first one here with nothing changed}

\subsubsection{Understanding the Data Management}
\todo{mention switching to demo dataset}
\missingfigure{shuffling plots}
Talk about how the data is loaded and created and managed for the network, talk about shuffling mechanisms and what I've found etc. talk about \verb|shuffle_obs_in_demo| and the other one.

\subsection{Improvements}



\subsubsection{Wrist Camera Alone isn't Enough}
As expected this is where the single wrist camera started showing its shortcomings. The agent would easily move around the obstacle, however, would struggle to make the last steps in touching the target. This is mostly due to the fact that the demonstrations (which are provided by RLBench) are not necessarily pointing the wrist of the robot and hence the camera mounted there to look towards the target. This means that the behavioural cloning agent learns to to the swaying motion around a large grey body, however, is not aware of the obstacle, or even understand the task is related to reaching for the obstacle and depends on its visual cues. 

\todo{add graph of going around the obstacle but not quite reaching the target (wrist)}
\todo{show this with other combinations of cameras, comment on if the l/r without wrist can learn to go around the obstacle easily? maybe not, generalisation might be hard with no wrist}

From experiments I have realised that it learns to move around the obstacle easily, using simple behavioural cloning. However, getting the last nudge to actually reach the target is where it falls apart, especially in more realistic scenarios where the target is randomised behind the obstacle. For static placement behind the wall, the agent, expectedly is quite good. \todo{maybe explain or evidence this, plateaus aroudn the same distance value and watchig t heroot act comfirms this}

\subsubsection{Other Cameras}
So, we can confirm that the wrist camera alone is not sufficient \todo{ref}, and the combination of wrist and other cameras are almost always better as more coverage of the workspace guarantees less occlusions and more information the agent can work with to make decisions. It was clear that the wrist camera alone wasn't going to cut it unless it learnt to look towards the target.\todo[color=red]{looking at he target??}


\todo{talk about the policy using differnt camera inputs to blend and make informed choices? maybe some plots here showing if it is better or not?}
\todo{this is important for plan2 later, as that would depend on such a mechanism, hint and even link that from here}

\subsubsection{Implementing `Looking' into the Demonstrations}\todo{haven't done this yet}\label{ew-looking-at-target}
issues this is not easy might be working on this as a part of approach 2 later
Another solution might be to experiment with the demonstration system to make sure we are pointing the wrist camera (so, the hand of our robot) towards its target as a demonstration trajectory is calculated \todo{explain that this proved tricky and might not even be worth it}

If we can't implicitly encode the `looking' information through the demonstration that means we will have to inject this information into our agent some other way. Another way to make sure agent understands to look at the target is teaching it to actively seek out its target, either following previous works such as \todo{find some prior info tracking works add ref} where object priors are incorporated into the learning or with attention mechanisms that figure out what is important in a task without prior object information \todo{maybe reference this later}.  

\subsection{Attending on a Camera in Given Combination}
\todo{talk about the implementation of the }
\subsubsection{MultiCNN}
\todo{explaint the implementations, maybe connect to }
\subsubsection{SingleCNN}
I initially though to disconnect the convolutional network, thinking the information can be encoded per view and then fused together to assign better meaning to the given pose. However, with the same logic the scene is still the same scene and encoding features together means that the network will learn to \todo{find a nice way to say the network will learn its fusing in the conv layers. maybe make sure it can}
% Grasping tasks
\section{Expanding the Task Space: Grasping Tasks}
Another task which is likely to suffer from lack of viewpoints is a grasping task. I first designed
a simple version (Figure \ref{fig:grasp-simple}) which the agent learns to reach then grasp the cubic target. 

Main differences between this and the reaching task is that the target here is tangible, so on top of being rendered it is also set to be \emph{collidable}. Another major addition is the usage of the \emph{extension string} as seen in \ref{subfig:simple-zoom-actions}, this instructs the demonstration engine to insert certain moves within the calculated trajectory. In this case \verb|open_gipper()| ensures the gripper is open, then a later waypoint will instruct it to close. 

As the task complexity increases, its design complexity also increases. Also, without any prior knowledge about 3D simulators and 3D design, it took me quite a long time to hunt everything about CoppeliaSim, RLBench, and PyRep to put these together. One criticism I have on these tools is the documentation is all over the place. \todo[color=green]{too ranty? rewrite or remove}

The more complicated counterpart, shown in Figure \ref{fig:grasp-move}, is a scenario where the cube needs to be picked up then moved to the target location (designated in green).

\begin{figure}[htpb] % htpb allows all placement
  \centering
  \begin{subfigure}{0.3\linewidth}
    \centering
    \includegraphics[scale=0.2]{../fyp/assets/task-pics/grasp/simple-front.png}
    \caption{Front}\label{subfig:simple-front}
  \end{subfigure}
  \hfill
  \begin{subfigure}{0.5\linewidth}
    \centering
    \includegraphics[scale=0.3]{../fyp/assets/task-pics/grasp/simple-front-zoom-gripper_actions.png}
    \caption{Zoomed, with gripper action}\label{subfig:simple-zoom-actions}
  \end{subfigure}
  \caption{Simple Grasping Task}\label{fig:grasp-simple}
\end{figure}

\begin{figure}[htpb] % htpb allows all placement
  \centering
  \begin{subfigure}{0.45\linewidth}
    \centering
    \includegraphics[scale=0.2]{../fyp/assets/task-pics/grasp/move-front.png} 
    \caption{Front}\label{subfig:grasp-move-front}
  \end{subfigure}
  \hfill
  \begin{subfigure}{0.45\linewidth}
    \centering
    \includegraphics[scale=0.2]{../fyp/assets/task-pics/grasp/move-top.png}
    \caption{Top}\label{subfig:grasp-move-top}
  \end{subfigure}
  \caption{Grasping then moving}\label{fig:grasp-move}
\end{figure}\todo[color=blue]{reshape}


\todo{add a picture with the cube grasped and the wrist camera view seen at that point}

Initially the wrist camera shouldn't pose any problems. Although, I suspect as we advance through the task, especially after we have grasped something, wrist camera becoming heavily obstructed will render it unreliable so basing our decisions on this medium alone might not be ideal.


\todo[color=pink]{experiments here, or move this where I can get some data on this}
\subsubsection{Observations}
What I got from these experiments was that the agent can benefit from understanding its surroundings at a higher level, and more importantly remembering them. This is because once thek camera becomes obstructed, as with \textbf{Grasp Then Move}, even if the agent could do some exploration to find the target, it wouldn't be ideal due to the restricted view it has access to. So, observing the environment before, and remembering important parts will be vital for the later stages of tasks. I aim to explore some pre-policy visual exploration of the environment then feed this information forward, possibly when it might be needed. For example residual forwarding of data might be used later \todo{should I keep this here?}



% Depth Interfacing - under grasping
\section{Depth Interfacing}\label{sec:depth-interfacing}
Understanding distance through depth interfacing is an important part of perception. Continuing from the drawn parallels to humans, we place object in our fields of vision by our two eyes. Stereo-vision, allows us to process two slightly different poses of a target object to reinforce our understanding of where that object is in the environment around us. Other information such as lighting (and shadows) may unconsciously help us as well. The main takeaway is that understanding distance to an object goes a long way in firstly understanding how to approach an object. There are a few ways to achieve this in robotics. In RLBench specifically is either to use two cameras (with a known distance between the two cameras to adjust poses with known intrinsics). However, RGB cameras are not the only things we have access to. We also we have a depth sensor.

This is because,active vision is to be able to use minimal amounts of viewpoints. In this scenario, I want to be able justify to use the wrist camera accompanied by the wrist depth sensor to replace any other workspace cameras. Therefore, it is important to study shortcomings of individual sensors with respect to others to understand how policies with limited access to sensors can be proficient at using them.

\subsection{Grasping with Varying Depths}
A grasping task makes sense for this experiment. Because unlike the reaching tasks from earlier, the robot will need slightly more precision in executing its grasp and actually grabbing the target. The task performer needs to figure out where an object is before attempting to grab it. So, I created a modified version of the simple grasping task where the target object's distance and scale can be externally varied to observe the behaviours of agents in a controlled manner.

\begin{figure}[htpb] % htpb allows all placement
  \centering
  \begin{subfigure}{0.4\linewidth}
    \centering
    \includegraphics[width=0.3\linewidth]{assets/depth-interfacing/normal-size-grasp.png}
    \caption{Normal Target Size}\label{subfig:normal-grasp}
  \end{subfigure}
  \begin{subfigure}{0.4\linewidth}
    \centering
    \includegraphics[width=0.3\linewidth]{assets/depth-interfacing/smaller-grasp.png}
    \caption{Smaller Target Size}\label{subfig:small-grasp}
  \end{subfigure}
  \caption{Visualisation of the Depth Interfacing experiment task}\label{fig:di-task}
\end{figure}

Figure \ref{fig:di-task}, is the general setup I am planning on using to evaluate the depth sensor versus a multi-view setup. Initial observations from the side clearly indicate that these are two different targets and will require different reach lengths before the agent can attempt to grab them.

\begin{figure}[htpb] % htpb allows all placement
  \centering
  \begin{subfigure}{0.2\linewidth}
    \centering
    \includegraphics[width=\linewidth]{assets/depth-interfacing/normal-size-wrist.png}
    \caption{Normal RGB}\label{subfig:normal-rgb}
  \end{subfigure}
  \begin{subfigure}{0.2\linewidth}
    \centering
    \includegraphics[width=\linewidth]{assets/depth-interfacing/smaller-wrist.png}
    \caption{Smaller RGB}\label{subfig:small-rgb}
  \end{subfigure}
  \begin{subfigure}{0.20\linewidth}
    \centering
    \includegraphics[width=\linewidth]{assets/depth-interfacing/normal-depth.png}
    \caption{Depth Mask}\label{subfig:normal-depth}
  \end{subfigure}
  \begin{subfigure}{0.20\linewidth}
    \centering
    \includegraphics[width=\linewidth]{assets/depth-interfacing/smaller-depth.png}
    \caption{Smaller Mask }\label{subfig:small-depth}
  \end{subfigure}
  \caption{Wrist RGB and Depth Masks for the tasks}\label{fig:di-rgb-vs-depth}
\end{figure}

However, as seen in the comparison in \ref{subfig:normal-rgb} and \ref{subfig:small-rgb}, the RGB outputs look practically the same, and will very likely produce extremely similar features after extraction. A way to differentiate them would be to utilise the wrist depth mask in this encoding. As shown in \ref{subfig:normal-depth} and \ref{subfig:small-depth}, they now carry different features in those areas. In the depth mask the darker colours indicate closer objects, and the information is encoded as floats. 

\begin{figure}[htpb] % htpb allows all placement
  \centering
  \begin{subfigure}{0.2\linewidth}
    \centering
    \includegraphics[width=\linewidth]{assets/depth-interfacing/normal-l_rgb.png}
    \caption{Normal Left}\label{subfig:normal-l-shoulder}
  \end{subfigure}
  \begin{subfigure}{0.2\linewidth}
    \centering
    \includegraphics[width=\linewidth]{assets/depth-interfacing/normal-r_rgb.png}
    \caption{Normal Right}\label{subfig:normal-r-shoulder}
  \end{subfigure}
  \begin{subfigure}{0.2\linewidth}
    \centering
    \includegraphics[width=\linewidth]{assets/depth-interfacing/smaller-l_rgb.png}
    \caption{Smaller Left}\label{subfig:smaller-l-shoulder}
  \end{subfigure}
  \begin{subfigure}{0.2\linewidth}
    \centering
    \includegraphics[width=\linewidth]{assets/depth-interfacing/smaller-r_rgb.png}
    \caption{Smaller Right}\label{subfig:smaller-r-shoulder}
  \end{subfigure}
  \caption{Left and Right Shoulder RGB Cameras}\label{fig:di-lr-shoulder}
\end{figure}


Another possible differentiating factor is the shoulder cameras. These also provide extra information about the scene the agent can use to understand its 3D geometry, even if it isn't explicitly taught. Although these are also not perfect, as seen in \ref{subfig:smaller-l-shoulder}, this camera is completely obstructed by the robot and is not seeing the target. This is not immediately problematic, because a smart enough system might be able to reason that the target is visible on the right and not on the left, leading to devising the correct depth for it.

% \subsubsection{Adding Stereo Wrist Vision}\todo[color=blue]{can be removed not sure}
% A possibility to overcome this \emph{self-occlusion} is to introduce stereo vision at the wrist level. Adding 2 off-centre cameras to the gripper, likely to the left and right of the existing wrist camera, will fix the self-occlusion, and possibly create a better comparison to the \emph{Wrist RGB} and \emph{Wrist Depth} combination. However, this comes with a lot of rewiring of RLBench and with its likely hidden issues that will pop up later. I did not want to take on this large architectural change before continuing with main investigation of the project, however, if timings permit this will be an important addition to the testing suite.

% proposed policies and learnings
\section{Multi-Modal Policies}
\subsection{Deeper/Better Feature Understanding - RCNN}\todo{I feel like this section is talking about the wrong thing}
First part of any policy is extracting features from the available views. Experimentation up to this point was using a bespoke CNN architecture. However, I wanted to check if an off-the-shelf model would be able to extract more informative features.
\todo{describe a problem the resnet would fix}

Another thing I wanted to try was to include off-the-shelf models to help extract features and information from my views. As one of the problems I described faced earlier was from the agent not remembering the earlier information that it has seen, I believed an important part of increasing the workspace understanding would be to incorporate residual connections.

I experimented with various sizes of ResNet networks \todo[color=green]{reference}. Essentially replacing my feature extraction block with modified versions of \emph{resnet18, resnet34, and resnet50} \todo[color=green]{maybe not mention all, afterall the large ones will be ass when the image is so small}.

They had to be modified because of the size restriction I had on my views.\todo{explain why the kernel size had to be dropped}

I had to modify these, because the unmodified version using a main CNN of kernel size $7$ was too large and appropriate features were not being extracted from the views I was feeding. This was evident from observing the agent act in a test scenario where the arm would do nothing remotely similar to what the demo did, or even what the task is about.\todo{move this at the start of the evaluation chapter}. 

\todo[color=pink]{run the original confirm it does not work, if not get a diagram of kernel sizes 7 and 3 and an explanation why this might be the case.}


A reason these might not have worked well is because of one of my earlier constraints. The image sizes being \(64 \times 64\) pixels, might not be enough to extract meaningful information with a ResNet. This is likely due to its aggressive pooling between the layers and especially during the residual connection and the aggressiveness only increases in larger models. \todo[color=red]{not actually sure if this is right, fact check, wrote this some time ago} 

\subsubsection{increasing the camera view for experimentation}\todo[color=red]{larger image trials? expand the task to be lager 128 or even 224 seems common with resnet}

\subsection{Feature Level Fusion}\todo{really long intro, consolidate}
On top of just naively combining views (or modalities) and expecting the policy to understand what it means to `see' is a tall order. 

I have adopted a model-free approach in making these policies, and decided to use representations of information for predicting the next motion.
\todo{maybe more on latent representations and encoder models, here with refs}

The step after extracting representation is to learn how we can make the most of the information by incorporating the various views and sensors and seeing what combination of feature extraction and their fusion allows the most information gain.

So, after extracting features from available sensors. Using multiple feature extractors means that these features will need to be merged before making a final decision on movement.

If these extracted features can be thought of as encoded information, as with latent models, multiple extractors will lead to to many encodings. So, the next challenge becomes merging; and not just how but when to merge features from one view to another. I will explain their potential positives and investigate the consequences of each.\todo[color=green]{this might be a good segue or part in presentation}

\subsubsection{Early Fusion}
Fuse different modalities into a single input and run through the encoding model. This usually involves combining the raw data sources. Before any information is extracted or the data is modelled in any way.

Therefore, its advantages mostly lie in its simplicity. No dedicated processing has to be done per information stream. On the other hand, this can fall short of capturing complex interactions between these modalities; depending on the network and the semantic information each modality provides.

\subsubsection{Late Fusion}
This is the complete opposite of early fusion. Each modality will be run through its own model and predict a representation. The immediate advantage is that each model can learn a rich representation relating to only its given modality


\subsubsection{Intermediate Fusion}
This is the most common middle-ground. While separated feature learners can be used per desired modality, the representation extracted from them will be fused together. Then this aggregated latent representation can be passed through another model to get the final predictions. This is a tradeoff based system between early and late fusion. Most of the proposed systems will follow this idea.

\todo{i will argue separating the grasping network will be late fusion fuse at the actionn level and not separating it will be fully intermediate}

\section{Proposed Fusion Methods}
I designed a a poly-policy which can be instantiated in any fusion configuration to handle its given modalities differently before making a final movement prediction. \todo{make a diagram - consult the sheet}

\todo{add the fusingEncoder and the config to the appendix}
\subsection{Simple Stacking}
An example of early fusion, this is the naive method of just stacking the views on the channel dimension, and has clear disadvantages for this specific system.

Firstly, for the RGB cameras, there is definite misalignment, all these cameras have different poses and will disagree on what they see. Leading to features not lining up, leaving us with a non-optimal and more importantly a brittle\todo{explain brittle? or does the next senetence take care of it} policy. This could lead to unintended coupling of information which can lead to the agent making wrong choices during inference when the same views don't necessarily align during rollout.\todo{simple diagram of overlaid squares (arrow) model (arrow) representation} This will act as a baseline to compare other fusion strategies.

\subsection{Understadning Depth Separately}\label{subsec:policies-understand-depth-sep}
RGB provides a denser representation of the system and in a configuration with multiple RGB views the depth information may be drowned out. Therefore, an important first step was to learn the depth features separately and then make them influence the action. The idea here is to have a RGB encoder and a separate Depth encoder CNNs, then learn the best way to merge this data together. I propose three different methods to do this:

\begin{itemize}
  \item Concatenation of the features from each CNN \todo{add figure, explain more if needed}
  \item A second layer CNN, convolve the RGB outputs and the Depth outputs to learn joint features \todo{add figure, explain more if needed}
  \item A shallower secondary CNN to learn simple \emph{gating} from already extracted RGB and Depth features. \todo{add figure, explain more if needed}
  \item A Cross Attention Mechanism to attend to the important parts of from each extracted feature.
\end{itemize}
Which are ultimately decoded into an \emph{action}, $\hat{A}$.

\subsection{Feature-wise Linear Modulation}\todo{ref this paper arxiv1709.07871}
To take a step back from all the excessive RGB data. I wanted to understand how modulating similar views with each other to understand similar observations from each other's perspective. Allowing the network to learn and discover its own trends is a fine task, however, implicitly injecting cues from modality into another will, in theory, allow the learning to be more efficient.


A light-weight and quick way to bias network parameters with conditional data is Feature-wise Linear Modulation (FiLM).\todo{ref} 
A FiLM network learns to adaptively influence the weights of another neural network by learning $f_c$ and $h_c$ as functions on the conditional input $\mathbf{x}_i$. So:

\[
\text{FiLM} = \left( \mathbf{F}_{i, c} | \gamma_{i, c}, ~\beta_{i, c} \right) = \gamma_{i, c}\mathbf{F}_{i, c} + \beta_{i, c} \qquad  \gamma_{i, c} = f_c\left(\mathbf{x}_i \right), \beta_{i, c} = h_c \left( \mathbf {x}_i \right)
\]
My implementation involves \todo[color=green]{appendix link} creating a simple linear network to generate the two parameters. The singular linear network means the parameters will have shared weights, which makes sense as the modulation is being condition on a single view, and hence a dependence connection. I also initialise this network to be the identity modulation, meaning \(\gamma = 1 \text{and} \beta = 0\). \todo{why, not sure?}

Another inherent benefit is the network can be kept at a minimal size. And finally, a FiLM module can be plugged in at multiple different places in the network at different scales. \todo{explain more maybe}

I created 3 different modulation configurations: wrist RBG modulated with wrist depth\todo{diagram}, depth modulated wrist RGB\todo{diagram}, and both modulated with each other\todo{diagram}. 

Where the modulated features act as the latent encoding, which can be used to predict an action On top of the three configurations, I also FiLMed at different depths: early FiLM, where feature shapes are matched then modulated leading to a coarser scale modulation; also a finer version where the order of down-sampled learning and modulation are swapped.\todo{explain better maybe}


\subsection{Derivative Methods}
The final propositions include 4-way fusion models that take into account ideas from earlier.
\begin{itemize}
  \item Fully separated feature extractors, all concatenated before action prediction \todo{add figure}
  \item Pairwise FiLM between all modalities: wrist RGB and w Depth, then Left Shoulder RGB and Right Shoulder RGB; then concatenated before action prediction\todo{figre}
  \item Finally 4-way multi-modal cross-attention. \todo{add math and figure if needed}
\end{itemize}

\subsection{Temporal Understanding of Demonstrations}
The actions in my system are heavily dependent on the context of where the robot came from. And the demonstrations contain coherent and smooth trajectories. Although, I am preserving the sequential nature of the demos in training. There is no implicit underlying mechanism to enforce the agent will learn from this. Due to the partially observable setting of my system, and the robot not knowing any extra information about its state (model-free), it is important to be able to keep a historical understanding of what has happened, or what the agent believes is important for that moment in time.

So, I incorporated recurrent models into the latent representation. This is so that the embedding of the current states will depend on a history of the earlier observations or states.

A pro of RNNs, compared making a bespoke sequential model is they are not length constrained \todo{was an earlier problem in 2d, maybe mention}. The sequences can be of any length which will help model the demos I have with varying episode lengths. Som compared to 3D CNNs \todo{ref}, which can also incorporate spatial and temporal information \todo{ref}


Another theoretical advantage is that the long-term memory can be used to have a \todo{mamybe mention the remoark from the then move test, }

\subsubsection{Latent RNN Encodings of Movement Steps}
To encode timestep information into my already existing features, I added an Long-Short Term Memory (LSTM) network module near the end of the pipeline\todo{add figure}. Meaning the history will be kept in the latent space.

So, this simple idea of introducing temporal noise into the decision process, should in theory allow for more concise predictions that are grounded in what the agent has observed so far.

\subsection{ViT Encodings}\todo{kind of dabled with this but will not be testing it}
\todo[color=red]{not done much on this, but may be able to talk about how I would go about it or just remove}



\subsection{Proprioception}\todo{maybe talk about why I left it out earlier in the chapter}
Up until this point, I left proprioceptive data out of the training, mainly because I wanted to train on purely visual feedback to see what it could achieve without any other state information. Secondly, including state information about the robot, would not necessarily immediately help with the 

\subsection{Numerical Foundations}\todo{talk about the general math, like what are the shapes of the data, what is the loss minimising-maximising, and notation}

The policies above are trained in the same way and the only separation is the modular feature extraction blocks, as well as what data they require. All input and output is handled within the network and the policy interface exposed to the agent is compatible with all the ones proposed in this project.

\subsubsection{The Dataset}
A demonstration of episode length $T$ is defined as: \(\mathcal{D} := \{\langle o_i, ~a_i\rangle\}_{i = 1}^{T} \). Observation, \(o_i \in \mathcal{O}\) is defined as the available modalities to the system with shape \(\mathcal{O} := \mathbb{R}^{c \times W \times H}\). This corresponds the resolution of the sensor (\textbf{W}idth and \textbf{H}eight) also $c \in \left[1, 10\right]$ which correspond to the available modality data concatenated on their channel dimension. The order of $c$ is always:\todo{figure here with the stuff stacked}
\[
{RGB}_{wrists}, ~{RGB}_{{left}\_{shoulder}}, ~{RGB}_{{right}\_{shoulder}}, ~{Depth}_{wrist}
\] 
where the $RGB$ data is $3$ channels wide while $Depth$ is only 1. \(a_i \in \mathbb{R}^{8 \times 1}\) is the corresponding joint velocities at timestep $i$, which acts as our ground truth. 

The Data Loader, as discussed \todo{ref}, will provide a randomly shuffled batch of $b$ collated demos, along with extra information. \( \langle O^B, ~A^B, ~\left( P^B, ~l^B \right) \rangle\), where \(O^B \in \mathbb{R}^{ B \times c \times W \times H}\). The observation batch size, $B$ is equal to \({\sum_{ d \in \mathcal{D}^B}|d|}\), this is the sum of the length of all demos in this training batch. Similarly, \(A^B \in \mathbb{R}^{B \times 8}\) per joint velocity (7) and final grasp indicator, \(P^B \in \mathbb{R}^{B \times 7}\) per joint angle - the proprioception data. Finally, \(l^B \in \mathbb{Z}^{b}\), contains the original lengths of the individual demos.

Then during the forward pass modalities of interest will be extracted from $O^B$ and features will be extracted depending on what fusion flavour is picked.



\subsubsection{Fine-Grained per Time-Step Training}\todo[color=green]{math fixing}
The RNN policy is disconnected from the main system.\todo{link the FusionRNNPolicy appendix} Differently from above, the batched demonstrations will now respect the individual lengths of provided demonstrations. The loader will now emit: \( \langle {O'}^B, ~{A'}^B, ~\left( {P'}^B, ~l^B \right) \rangle\). The \emph{prime} ($'$) indicates a change in their batch dimesion, each now follow $\mathbb{R}^{b \times T_{max} \times \ldots}$ and  $T_{max}$ is the longest demonstration in this batch \(\max l^{B}\). We pad every demo to this length while keeping the original lengths $l^{B}$.

To fit the network to our data quite closely I opted to train the network with information at every step. Meaning every single step $i$ in a demo $mathcal{D}$ with episode length $T$, will have an action predicted with respect to $\mathcal{D}_{t < i}$ and will have an impact on the loss of the system. 

The loss will be calculated using the predicted actions, $\hat{A}$ and the ground truth labels, $A$; as \( \mathcal{L}^{action} = {loss}_{mse}\left(A_{x, t} - \hat{A}_{x, t}\right)\) for $x$ being the batch number in $1..b$ and $t$ the timestep in $1..T_{max}$. A bit mask \(M \in \{0, 1\}^{b \times T_{max}}, ~M_{x, t} = 1, ~t < l^B_x ~else ~0\) will be used to only extract the loss items lying withing the sequence lengths of demos. Finally, the loss per epoch will be normalised by the number of timesteps in the demos provided. \(loss = \frac{1}{\sum_{x, t}M_{x, t}}{\sum_{x = 1}^{b}\sum_{t = 1}^{T_{max}}M_{x, t} \mathcal{L}^{action}}\)

As before\todo{ref to earlier grasp}, the grasping loss is separated in the default configuration of this policy. Giving the final loss as: 
\[
  \mathcal{L}^{action} = \mathcal{L}^{pose} + \lambda_{grasp} \mathcal{L}^{grasp}
\]
where $\mathcal{L}^{pose}$ and $\mathcal{L}^{grasp}$ are calculated using the same masking method as above.

\missingfigure{updated diagram to convey the padding and unpacking and training the loss on every frame step}


\section{Moving to \emph{Active Vision}}
End-to-end models with absolutely zero fine-grained interactions are quite competent. However, in the chaotic and random environments an agent may be acting in. Planning and specific nudges may prove beneficial. Similar to the 

\todo{improve the grasping}

\todo{It might be worth scaling everthing to 128 by 128 next week and do a repeat run so that I can at least say that I have done it}

\todo{ruun grasp then move and confirm the suspicions of the policies with no memory failing.}


